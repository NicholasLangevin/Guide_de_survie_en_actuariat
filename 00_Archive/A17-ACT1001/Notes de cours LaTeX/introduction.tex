Les math�matiques financi�res sont utilis�es dans plusieurs contextes : 
	\begin{itemize}
		\item{d�p�t (\emph{deposit} \;}
		\item{emprunt (\emph{loan} \;}
		\item{pr�t hypoth�caire (\emph{mortgage} \;}
		\item{obligation (\emph{bond} \;}
		\item{investissement.}
	\end{itemize}	
\p
\begin{remarque}
\hl{Par d�faut, $t$ mesure le temps en ann�es, $C$ repr�sente le Capital et $i$ repr�sente le taux d'int�r�t.}
\end{remarque}

Note : Il est possible de donner des dates auquel cas il peut �tre question d'ann�es et de mois ($\frac{1}{12}$) ou d'ann�es et de jours ($\frac{1}{365}$).
\paragraph{}

\begin{remarque}
\hl{La $t\up{e}$ ann�e va de $(t-1)$ � $t$}. Par d�faut, les calculs sont donc bas� sur la fin de l'ann�e.
\end{remarque}

\subsection*{Exemple}
Du 1\up{er} mars au 1\up{er} juin, il y a...
\begin{itemize}
	\item{si on compte en mois, il y a 3 mois ou $\frac{1}{4}$ an.}
	\item{si on compte en jours, $31+30+31=92$ jours ou $\frac{92}{365}$ ans.}
\end{itemize}

\p
Il existe plusieurs sortes de taux, notamment : 
\begin{itemize}
	\item{TIOL : Taux inter-bancaire offert � Londres (\emph{London Interbank Overnight Rate})}
	\item{Taux de base des banques (\emph{prime rate)})}
	\item{Taux cible de la R�serve F�d�rale (\emph{Federal Reserve Target Rate})}
	\item{Taux hypoth�caire (\emph{Mortgage Rate})}
\end{itemize}

\p
\subsection*{Taux de rendement et taux d'int�r�t}
En g�n�ral, on parle de \textbf{taux d'int�r�t} (\emph{interestrate}) quand il est fix� et connu d'avance au moins pour une certaine p�riode. Sinon, on parle plut�t d'un \textbf{taux de rendement} (\emph{Rate of Return}), lequel ne se mesure seulement qu'� {\itshape posteriori}. \hl{il faut donc faire attention � la terminologie et � la notation}.