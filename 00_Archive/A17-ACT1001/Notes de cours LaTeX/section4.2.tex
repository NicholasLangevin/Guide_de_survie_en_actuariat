% d�but section 4.2
\subsection{Obligation remboursable par anticipation : dates optionnelles de remboursement}
en anglais : \emph{Callable bond}

C'est l'�metteur de l'obligation qui d�cide des moments o� le remboursement anticip� peut �tre fait. $C$ peut d�pendre de la date choisie.

\bcattention Le taux de rendement � l'�ch�ance (RAE) d�pend du vrai $T$ et du vrai $C$.
\p
Si \blue{$Fr = Cj$}, tout est �quivalent.
\p
Si \red{$Fr < Cj$}, les coupons sont \emph{petit} et l'�metteur aura avantage � attendre. Donc, on calculera le prix (le plus petit) et le taux $j$ (le plus petit) en supposant le remboursement � l'�ch�ance.
\p
Si \darkgreen{$Fr > Cj$}, les coupons sont \emph{gros} et l'�metteur aura avantage � se d�p�cher.
\p
\bcattention \underline{Notes}
\begin{enumerate}[label=\checked]
\item Calculs de $P$ et $j$ se font en supposant un remboursement d�s que possible.
\item Ces observations supposent que la valeur de remboursement est toujours $C$.
\item Sinon, il faut faire les calculs pour chaque $C$ possibles et prendre le plus petit $P$ (ou $j$).
\end{enumerate}
% fin section 4.2