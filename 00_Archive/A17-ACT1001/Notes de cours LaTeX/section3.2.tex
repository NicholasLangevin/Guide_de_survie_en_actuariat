% section 3.2 math fin

\bcattention $j_t = j \forall\ t$ et $K_t = K \quad$ pour $t= 1,2,...,n$

\begin{empheq} [box=\fcolorbox{black}{green}]{equation}
L = K \ax{\angln j}
\end{empheq}

Donc $OB_0 = L \quad, OB_n = 0$

\underline{$OB_t$ avec la m�thode \textbf{r�trospective}}
\begin{align*}
OB_t = L(1+j)^t - K \sx{\angl{t} j} \quad t = 1,2,...,n
\end{align*}


\underline{$OB_t$ avec la m�thode \textbf{prospective}}
\begin{align*}
OB_t = K \ax{\angl{n-t} j} \quad t = 0,1,2,...,n-1
\end{align*}

\begin{equation*}
K_T = nK
\end{equation*}

\begin{align*}
PR_t 	& = OB_{t-1} - OB_t \\
	& = K(\ax{\angl{n-t+1} j} - \ax{\angl{n-t} j} \\
	& = K(1+j)^{-(n-t+1)} \\
	& = Kv^{n-t+1} \rightarrow \text{progression g�om�trique}\\
\end{align*}

\begin{align*}
I_t 	& = K_t - PR_t \\
	& = K(1 - v^{n-t+1}) \\
\end{align*}

\begin{align*}
I_T	& = K_T - PR_T \\
	& = nK - L \\
\end{align*}

Si $K_t = K_{t+1}$ et $j_t = j_{t+1} = j$, alors

\begin{equation*}
PR_{t+1} = PR_t(1+j)
\end{equation*}
\begin{proof}
\begin{align*}
PR_{t+1}		& = K - OB_t j \\
			& = K - (OB_{t-1}(1+j) - K) j \\
			& = K + Kj - OB_{t-1} j(1+j) \\
			& = K(1+j) - OB_{t-1} j(1+j) \\
			& = (K - OB_{t-1}j)(1+j) \\
			& = PR_t(1+j) \\
\end{align*}
\end{proof}

\bcinfo Au Canada, les taux sont mesur� $i$ ou $i^{(2)}$.

Supposons $N$ ann�es de versements mensuels, $n = 12N$.

\begin{align*}
L	& = K \ax{\angl{12N} j} \\
	& = 6K \ax{\angl{2N} i^{(2)}/2}[(6)] \\
	& = 12K \ax{\angl{N} i}[(12)] 
\end{align*}

\bcinfo Au �tats-Unis, $i^{(12)}$

\begin{equation*}
L = K \ax{\angl{12N} i^{(12)}/12}
\end{equation*}

�a peut para�tre moins cher, mais il y a des frais $F$!

Si on veut trouver le vrai taux,
\begin{equation*}
L - F = K \ax{\angl{12N} j^{(12)}/12}
\end{equation*}
\p
Supposons un emprunt initial (par exemple, un hypoth�que) de $L$ sur $N$ ann�es. Le taux initial $i^{(2)}$ fix� pour $n$ ann�es. Les versmements mensuels sont �gaux. On sait aussi que le pr�t initial est contract� � la banque A.

\begin{align*}
L = 6K \ax{\angl{2N} i^{(2)}/2}[(6)] \\
K = \frac{L}{6 \ax{\angl{2N} i^{(2)}/2}[(6)]} \\
\end{align*}


Au bout de $n$ ann�es, est-ce qu'on change pour la banque B? On choisit $i_A^{(2)}$ ou $i_B^{(2)}$ ?

Pour r�pondre � cette question, calculons les 2 soldes : 

\begin{align*}
OB_{12n} = 6K \ax{\angl{2(N-n)} i^{(2)}/2}[(6)]
\end{align*}
Mais il faut aussi consid�rer les frais $F$ pour changer...

\begin{enumerate}
\item Supposons qu'on peut financer $F$ dans le nouveau pr�t � la banque B. On va comparer les versements : si $K_A \le K_B$, �a ne sert � rien de changer!
	\begin{enumerate}
	\item 
	\begin{equation*}
	OB_{12n}  = 6K_A \ax{\angl{2(N-n)} i_A^{(2)}/2}[(6)]
	\end{equation*}
	\item
	\begin{equation*}
	OB_{12n} + F = 6K_B \ax{\angl{2(N-n)} i_B^{(2)}/2}[(6)] \\
	\end{equation*}
	\end{enumerate}
	
\item Supposons que les frais $F$ \underline{ne sont pas financ�s} (contexte � l'am�ricaine).
\begin{align*}
OB_{12n}		& = 6K_B \ax{\angl{2(N-n)} i_B^{(2)}/2}[(6)] \\
OB_{12n} - F	& = 6K_B \ax{\angl{2(N-n)} i_{B^*}^{(2)}/2}[(6)] \\
\end{align*} 	
\end{enumerate}

\subsubsection*{Exemple}
$
\begin{aligned}
N =25	&\quad n = 5	&\quad i=6\%	&\quad L = 100000\\
\end{aligned}
$

Trouver le montant des paiements mensuels �gaux : 
\begin{align*}
100000	& = 12K \ax{\angl{25} 6\%}[(12)] \\
... \\
k		& = 634,620844 \$ \\
\end{align*}

Solde : 

\begin{align*}
OB_{60}	& = 12K \ax{\angl{20} 6\%}[(12)] \\
		& = .... \\
		& = 89 725,429508\$ \\
\end{align*}

Options : 
\begin{enumerate}
\item renouveler � $i_A = 4\%$ 
\item ou refinancer � $i_B = 3,75\%$ mais p�nalit� de $2000\$$
\end{enumerate}

\begin{enumerate}
	\item 
\begin{align*}
OB_{60} 	& = 12K_A \ax{\angl{20} 4\%}[(12)] \\
		& = ... \\
	K_A	& = 540,343367
\end{align*}

\begin{align*}
Ob_{60}	& = K_B \ax{\angl{20} 3,75\%}[(12)] \\
		& = ... \\
K_B		& = 540,829142 \\
\end{align*}
	
	\item
\begin{align*}
OB_{60}	& = 12K_B^* \ax{\angl{20} 3,75\%}[(12)] \\
		& = ... \\
K_B^*	& = 529,036793 \\
\end{align*}

\begin{align*}
OB_{60} - 2000		& = 12 K_B^* \ax{\angl{20} i_B^*}[(12)] \\
				& = ... \\
		i_B^*	& = 4,016598\% \\  
\end{align*}	
\end{enumerate}

