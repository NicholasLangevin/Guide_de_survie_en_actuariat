% !TEX encoding = UTF-8 Unicode
% LaTeX Preamble
% Author : Gabriel Crépeault-Cauchon
% Last update : 10/10/2019 by Alec James van Rassel
%
% ---------------------------------------------
% BEGINNING OF PREAMBLE
% ---------------------------------------------
%
%%% -----------------------------
%%% Encoding packages
%%% -----------------------------
\usepackage[utf8]{inputenc}
\usepackage[T1]{fontenc}
\usepackage{babel}
\usepackage{lmodern}

%%% -----------------------------
%%% Options for URLs and links
%%% -----------------------------
\usepackage{hyperref}
\hypersetup{colorlinks = true, urlcolor = red!60!black, linkcolor = red!60!black}

%%% -----------------------------
%%% Document policy (uncomment only one)
%%% -----------------------------
%	\usepackage{concrete}
%	\usepackage{mathpazo}
%	\usepackage{frcursive} %% permet d'écrire en lettres attachées
 \usepackage{aeguill}
% 	\usepackage{mathptmx}
%	\usepackage{fourier} 

%%% -----------------------------
%%% Math configuration
%%% -----------------------------
\usepackage{amsmath,amsthm,amssymb,latexsym,amsfonts}
\usepackage{empheq}
\usepackage{numprint}
\usepackage{dsfont} 	% Pour avoir le symbole du domaine Z

%% -----------------------------
%% tcolorbox configuration
%% -----------------------------
\usepackage{tcolorbox}
\tcbuselibrary{xparse}
\tcbuselibrary{breakable}

%%
%% Coloured box "definition" for definitions
%%
\newcounter{def}[section]
\DeclareTColorBox{definition}{ o }				% #1 parameter
{
	colframe=blue!60!green,colback=blue!5!white, % color of the box
	breakable, 
	pad at break* = 0mm, 						% to split the box
	title = {#1},
	after title = {\large \hfill \faBook}
}
%%
%% Coloured box "algo" for algorithms
%%
\newtcolorbox{algo}[ 1 ]
{
	colback = blue!5!white,
	colframe = blue!75!black,
	fonttitle = \bfseries,title=#1
}
%%
%% Coloured box "formula" for formulas
%%
\newtcolorbox{formula}[ 1 ]
{
	colback = green!5!white,
	colframe = green!75!black,
	fonttitle = \bfseries,title=#1
}
%%
%% Coloured box "exemple" for examples
%%
\newcounter{ex}[section]
\DeclareTColorBox{exemple}{ o }	% #1 parameter
{
	colframe = green!20!black,
	colback = green!5!white, % color of the box
	breakable, 
	pad at break* = 0mm, % to split the box
	before title = 
	{
		\textbf{Exemple \stepcounter{ex} \arabic{chapter}.\arabic{section}.\arabic{ex} }
	},
	IfValueTF = {#1}{title= {#1}}{title= \hphantom},
	after title = {\large \hfill \faWrench}
}
\DeclareTColorBox{note}{ o }
    {colframe=black,
     colback=white,
     sharp corners,
     pad at break*=0mm,
     IfValueTF={#1}{title={#1}, fonttitle=\bfseries}{title=Note, fonttitle=\bfseries}}

%% -----------------------------
%% Graphics and pictures
%% -----------------------------
\usepackage{graphicx}
\usepackage{pict2e}
\usepackage{tikz}
%
% Special symbols packages
%
\usepackage[tikz]{bclogo}
%% -----------------------------
%% Fontawesome for special symbols
%% -----------------------------
\usepackage{fontawesome}

%% -----------------------------
%% insert pdf pages into document
%% -----------------------------
\usepackage{pdfpages}

%% -----------------------------
%% Color configuration
%% -----------------------------
\usepackage{color, soulutf8, colortbl}

%
%	Colour definitions
%
\definecolor{darkpastelpurple}{rgb}{0.59, 0.44, 0.84}
\definecolor{darkgreen}{rgb}{0.0, 0.2, 0.13}		
\definecolor{tocColor}{HTML}{8A2507}	
\definecolor{burntorange}{rgb}{0.8, 0.33, 0.0}		
\definecolor{burntsienna}{rgb}{0.91, 0.45, 0.32}		
\definecolor{ao(english)}{rgb}{0.0, 0.5, 0.0}		% ACT-2003
\definecolor{amber(sae/ece)}{rgb}{1.0, 0.49, 0.0} 	% ACT-2003
\definecolor{green_rectangle}{RGB}{131, 176, 84}		% ACT-2004
\definecolor{red_rectangle}{RGB}{241,112,113}		% ACT-2004
\definecolor{blue_rectangle}{RGB}{83, 84, 244}		% ACT-2004
\definecolor{amethyst}{rgb}{0.6, 0.4, 0.8}
\definecolor{amethyst-light}{rgb}{0.6, 0.4, 0.8}

%% -----------------------------
%% Mathematics table
%% -----------------------------
\usepackage{array}   		% for \newcolumntype macro
%% -----------------------------
%% Tabular column type configuration
%% -----------------------------
\newcolumntype{C}{>{$}c<{$}} % math-mode version of "l" column type
\newcolumntype{L}{>{$}l<{$}} % math-mode version of "l" column type
\newcolumntype{R}{>{$}r<{$}} % math-mode version of "l" column type
\newcolumntype{f}{>{\columncolor{green!20!white}}p{1cm}}
\newcolumntype{g}{>{\columncolor{green!40!white}}m{1.2cm}}
\newcolumntype{a}{>{\columncolor{red!20!white}$}p{2cm}<{$}}	% ACT-2005
% configuration to force a line break within a single cell
\usepackage{makecell}
%
%	Useful shortcuts for colored text
%
\newcommand{\orange}{\textcolor{orange}}
\newcommand{\red}{\textcolor{red}}
\newcommand{\cyan}{\textcolor{cyan}}
\newcommand{\blue}{\textcolor{blue}}
\newcommand{\green}{\textcolor{green}}
\newcommand{\darkgreen}{\textcolor{darkgreen}}
\newcommand{\purple}{\textcolor{magenta}}
\newcommand{\yellow}{\textcolor{yellow}}

%% -----------------------------
%% Enumerate environment configuration
%% -----------------------------
%
% Custum enumerate & itemize Package
%
\usepackage{enumitem}
%
% French Setup for itemize function
%
\frenchbsetup{StandardItemLabels=true}
%
% Change default label for itemize
%
\renewcommand{\labelitemi}{\faAngleRight}

%%%
%%%	Mathematics shortcuts
%%%
\usepackage{cancel}

\newcommand{\reels}{\mathbb{R}}
\newcommand{\entiers}{\mathbb{Z}}
\newcommand{\naturels}{\mathbb{N}}
\newcommand{\eval}{\biggr \rvert}
\newcommand{\derivee}[1]{\frac{\partial}{\partial #1}}
\newcommand{\prob}[1]{\Pr \left( #1 \right)}	% probabilité entre parenthèses
\newcommand{\esp}[1]{\mathrm{E} \left[ #1 \right]} 
\newcommand{\variance}[1]{\mathrm{Var} \left( #1 \right)}
\newcommand{\covar}[1]{\mathrm{Cov} \left( #1   \right)}
\newcommand{\laplace}{\mathcal{L}}
\newcommand{\matr}[1]{\mathbf{#1}} % Notation matricielle
\DeclareMathOperator{\Tr}{Tr}
\newcommand{\fgp}{\mathcal{P}}
\DeclareMathOperator{\Adj}{Adj}
\newcommand{\indic}[1]{\mathds{1}_{\{ #1 \}}}
\newcommand{\VaR}[2][k]{\mathrm{VaR}_{#1}{\left( #2 \right)}}
\newcommand{\TVaR}[2][k]{\mathrm{TVaR}_{#1}{\left( #2 \right)}}
\newcommand{\hatY}{\hat{Y}}
\newcommand{\hatbeta}{\hat{\beta}}
\newcommand{\sumn}{\sum_{i=0}^n}
%
%	Actuarial notation packages
%
\usepackage{actuarialsymbol}
\usepackage{actuarialangle}
%
%	Matrix notation for math symbols (\bm{})
%
% 	NOTE: 
%	+ à enlever éventuellement, j'ai ajouté la macro \matr{} à la place.
\usepackage{bm}

%
% 	To indicate equation number on a specific line in align environment
%
\newcommand\numberthis{\addtocounter{equation}{1}\tag{\theequation}}

% Other shortcuts
\newcommand{\p}{\paragraph{}}
\newcommand{\n}{\newline}

% source : https://tex.stackexchange.com/questions/112576/math-mode-in-tabular-without-having-to-use-everywhere

%
%	Retire l'indentation automatique de Latex
%
\setlength{\parindent}{0pt}

%
%	Utilisé pour la page couverture
% 
\usepackage[absolute]{textpos} % Textblock environement
\usepackage{anyfontsize} % Avoir un gros titre
\usepackage{titling} % Avoir un gros titre
\usepackage{changepage} % ajustwidth environement

%
% Pour afficher du code
%
\usepackage{listings}

\definecolor{codegray}{gray}{0.9}
\newcommand{\code}[1]{\colorbox{codegray}{\texttt{#1}}}

\definecolor{insideBlackTerminal}{RGB}{33,33,33}
% Set Language
% \lstset{
%     language={bash},
%     basicstyle=\small\ttfamily\color{white}, % Global Code Style
%     captionpos=b, % Position of the Caption (t for top, b for bottom)
%     extendedchars=true, % Allows 256 instead of 128 ASCII characters
%     tabsize=2, % number of spaces indented when discovering a tab 
%     columns=fixed, % make all characters equal width
%     keepspaces=true, % does not ignore spaces to fit width, convert tabs to spaces
%     showstringspaces=false, % lets spaces in strings appear as real spaces
%     breaklines=true, % wrap lines if they don't fit
%     frame=single, % draw a frame at the top, right, left and bottom of the listing
%     numberstyle=\tiny\ttfamily, % style of the line numbers
%     % commentstyle=\color{red}, % style of comments
%     % keywordstyle=\color{red}, % style of keywords
%     % stringstyle=\color{red}, % style of strings
%     backgroundcolor = \color{insideBlackTerminal},
%     rulecolor=\color{red}
% }

\usepackage{lstlinebgrd}
\definecolor{grayComment}{HTML}{8D90B8}
\lstset{
	language = R,                     % the language of the code
	basicstyle = \ttfamily, % the size of the fonts that are used for the code
% 	numbers = left,                   % where to put the line-numbers
% 	numberstyle = \color{blue},  % the style that is used for the line-numbers
% 	stepnumber = 1,                   % the step between two line-numbers. If it is 1, each line will be numbered
	numbersep = 5pt,                  % how far the line-numbers are from the code
	backgroundcolor = \color{white},  % choose the background color. You must add \usepackage{color}
	linebackgroundcolor = \color{white},
	showspaces = false,               % show spaces adding particular underscores
	showstringspaces = false,         % underline spaces within strings
	showtabs = false,                 % show tabs within strings adding particular underscores
	frame = single,                   % adds a frame around the code
	rulecolor = \color{black},        % if not set, the frame-color may be changed on line-breaks within not-black text (e.g. commens (green here))
	tabsize = 2,                      % sets default tabsize to 2 spaces
	captionpos = b,                   % sets the caption-position to bottom
	breaklines = true,                % sets automatic line breaking
	breakatwhitespace = false,        % sets if automatic breaks should only happen at whitespace
%   keywordstyle=\color{functionR},      % keyword style
	commentstyle = \color[HTML]{9F0808},  %\color[HTML]{8D90B8},   % comment style
%   stringstyle=\color[HTML]{1D9507},      % string literal style
	moredelim = **[is][\color{grayComment}]{@}{@}, % couleur manuel
	literate = %
		{à}{{\`a}}1
		{é}{{\'e}}1
		{è}{{\`e}}1
} 

%%% -----------------------------
%%% Cheatsheets customization
%%% -----------------------------

%%% -----------------------------
%%% Variable definition
%%% -----------------------------
%\def\auteur{Gabriel Crépeault-Cauchon / Nicholas Langevin}
%\def\BackgroundColor{white}
%
%%% -----------------------------
%%% Margin and layout
%%% -----------------------------
%% Determine the margin for cheatsheet
%\usepackage[landscape, hmargin=1cm, vmargin=1.7cm]{geometry}
%\usepackage{multicol}
%
%%% Remove automatic indentation after section/subsection title.
%%\setlength{\parindent}{0cm}
%
%% Save space in cheatsheet by removing space between align environment and normal text.
%\usepackage{etoolbox}
%\newcommand{\zerodisplayskips}{%
%  \setlength{\abovedisplayskip}{0pt}%
%  \setlength{\belowdisplayskip}{0pt}%
%  \setlength{\abovedisplayshortskip}{0pt}%
%  \setlength{\belowdisplayshortskip}{0pt}}
%\appto{\normalsize}{\zerodisplayskips}
%\appto{\small}{\zerodisplayskips}
%\appto{\footnotesize}{\zerodisplayskips}
%
%%% -----------------------------
%%% Section Font customization
%%% -----------------------------
%\usepackage{sectsty}
%\sectionfont{\color{\SectionColor}}
%\subsectionfont{\color{\SubSectionColor}}
%
%%% -----------------------------
%%% Footer/Header Customization
%%% -----------------------------
%\usepackage{lastpage}
%\usepackage{fancyhdr}
%\pagestyle{fancy}
%
%%
%% Header
%%
%\fancyhead{} 	% Reset
%\fancyhead[L]{Aide-mémoire pour~ \cours ~(\textbf{\sigle})}
%\fancyhead[R]{\auteur}
%
%%
%% Footer
%%
%\fancyfoot{}		% Reset
%\fancyfoot[R]{\thepage ~de~ \pageref{LastPage}}
%\fancyfoot[L]{\href{https://github.com/gabrielcrepeault/latex-template}{\faGithub \ gabrielcrepeault/latex-template}}
%
%%
%% Page background color
%%
%\pagecolor{\BackgroundColor}


















































% ---------------------------------------------
% END OF PREAMBLE
% ---------------------------------------------
