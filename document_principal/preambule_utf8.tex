% !TEX encoding = UTF-8 Unicode
% LaTeX Preamble
% Author : Gabriel Crépeault-Cauchon
% Last update : 08/09/2019
% ---------------------------------------------
% BEGINNING OF PREAMBLE
% ---------------------------------------------
% Encoding packages
\usepackage[utf8]{inputenc}
\usepackage[T1]{fontenc}
\usepackage{babel}
\usepackage{lmodern}

% HYPERREF (URL's and Link options)
\usepackage{hyperref}
\hypersetup{colorlinks = true, urlcolor = red!60!black, linkcolor = red!60!black}

% POLICY (choose one of them)
%	\usepackage{concrete}
%	\usepackage{mathpazo}
%	\usepackage{frcursive} %% permet d'écrire en lettres attachées
 \usepackage{aeguill}
% 	\usepackage{mathptmx}
%	\usepackage{fourier} 

% Mathematics configuration
\usepackage{amsmath,amsthm,amssymb,latexsym,amsfonts}
\usepackage{empheq}
\usepackage{numprint}
\usepackage{dsfont} 


% Tcolorbox config
\usepackage{tcolorbox}
\tcbuselibrary{xparse}
\tcbuselibrary{breakable}

% Définition Boite pour exemple
\newcounter{ex}[section]
\DeclareTColorBox{exemple}{ o }% #1 parameter
{colframe=green!20!black,colback=green!5!white, % color of the box
breakable, pad at break*=0mm, % to split the box
before title = {\textbf{Exemple \stepcounter{ex} \arabic{chapter}.\arabic{section}.\arabic{ex} }},
IfValueTF = {#1}{title= {#1}}{title= \hphantom},
after title = {\large \hfill \faWrench}
}

%% Définition boite pour définition
\newcounter{def}[section]
\DeclareTColorBox{definition}{ o }% #1 parameter
{colframe=blue!60!green,colback=blue!5!white, % color of the box
breakable, pad at break*=0mm, % to split the box
before title = {\textbf{Définition \stepcounter{def} \arabic{chapter}.\arabic{section}.\arabic{def} }},
title = {#1},
after title = {\large \hfill \faBook}
}

\DeclareTColorBox{note}{ o }
    {colframe=black,
     colback=white,
     sharp corners,
     pad at break*=0mm,
     IfValueTF={#1}{title={#1}, fonttitle=\bfseries}{title=Note, fonttitle=\bfseries}}


% Graphics and picture import Packages
\usepackage{graphicx}
\usepackage{pict2e}

% insert PDF package
\usepackage{pdfpages}

% Color package
\usepackage{color, soulutf8, colortbl}

% Mathematics table
\usepackage{array}   % for \newcolumntype macro
\newcolumntype{L}{>{$}l<{$}} % math-mode version of "l" column type

% usefull shortcut for colored text
\newcommand{\orange}{\textcolor{orange}}
\newcommand{\red}{\textcolor{red}}
\newcommand{\cyan}{\textcolor{cyan}}
\newcommand{\blue}{\textcolor{blue}}
\newcommand{\green}{\textcolor{green}}
\newcommand{\darkgreen}{\textcolor{darkgreen}}
\newcommand{\purple}{\textcolor{magenta}}
\newcommand{\yellow}{\textcolor{yellow}}

% Colors define
\definecolor{darkgreen}{RGB}{37, 128, 40}
\definecolor{tocColor}{HTML}{8A2507}

% Custum enumerate & itemize Package
\usepackage{enumitem}
% French Setup for itemize function
\frenchbsetup{StandardItemLabels=true}

% Mathematics shortcut
\usepackage{cancel}
\newcommand{\reels}{\mathbb{R}}
\newcommand{\entiers}{\mathbb{Z}}
\newcommand{\naturels}{\mathbb{N}}
\newcommand{\eval}{\biggr \rvert}
\newcommand{\esp}[2][]{\mathrm{E}_{#1} \left[ #2 \right]} % espérance
\newcommand{\variance}[1]{\mathrm{Var} \left( #1 \right)} % variance
\newcommand{\covar}[1]{\mathrm{Cov} \left( #1 \right)} % variance
\newcommand{\prob}[1]{\Pr \left( #1 \right)} % probabilité entre parenthèses
\newcommand{\laplace}{\mathcal{L}}
\newcommand{\matr}[1]{\mathbf{#1}} % Notation matricielle
\DeclareMathOperator{\Tr}{Tr}
\newcommand{\fgp}{\mathcal{P}}
\DeclareMathOperator{\Adj}{Adj}
\newcommand{\derivee}[1]{\frac{\partial}{\partial #1}}
\newcommand{\indic}[1]{\mathds{1}_{\{ #1 \}}}
\newcommand{\VaR}[2][k]{\mathrm{VaR}_{#1}{\left( #2 \right)}}
\newcommand{\TVaR}[2][k]{\mathrm{TVaR}_{#1}{\left( #2 \right)}}
\newcommand{\hatY}{\hat{Y}}
\newcommand{\hatbeta}{\hat{\beta}}
\newcommand{\sumn}{\sum_{i=0}^n}






% Matricial anotation for math symbols (\bm{•})
% à enlever éventuellement, j'ai ajouté la macro \matr{} à la place.
\usepackage{bm}

% Actuarial notation package
\usepackage{actuarialsymbol}
\usepackage{actuarialangle}

% To indicate equation number on a specific line in align environment
\newcommand\numberthis{\addtocounter{equation}{1}\tag{\theequation}}

% Other shortcut
\newcommand{\p}{\paragraph{}}
\newcommand{\n}{\newline}

% source : https://tex.stackexchange.com/questions/112576/math-mode-in-tabular-without-having-to-use-everywhere



% Special symbols package
 \usepackage[tikz]{bclogo}
\usepackage{fontawesome}

% Retire l'indentation automatique de Latex
\setlength{\parindent}{0pt}

% Utilisé pour la page couverture
\usepackage[absolute]{textpos} % Textblock environement
\usepackage{anyfontsize} % Avoir un gros titre
\usepackage{titling} % Avoir un gros titre
\usepackage{changepage} % ajustwidth environement

% Pour afficher du code
\usepackage{listings}

\definecolor{codegray}{gray}{0.9}
\newcommand{\code}[1]{\colorbox{codegray}{\texttt{#1}}}

\definecolor{insideBlackTerminal}{RGB}{33,33,33}
% Set Language
% \lstset{
%     language={bash},
%     basicstyle=\small\ttfamily\color{white}, % Global Code Style
%     captionpos=b, % Position of the Caption (t for top, b for bottom)
%     extendedchars=true, % Allows 256 instead of 128 ASCII characters
%     tabsize=2, % number of spaces indented when discovering a tab 
%     columns=fixed, % make all characters equal width
%     keepspaces=true, % does not ignore spaces to fit width, convert tabs to spaces
%     showstringspaces=false, % lets spaces in strings appear as real spaces
%     breaklines=true, % wrap lines if they don't fit
%     frame=single, % draw a frame at the top, right, left and bottom of the listing
%     numberstyle=\tiny\ttfamily, % style of the line numbers
%     % commentstyle=\color{red}, % style of comments
%     % keywordstyle=\color{red}, % style of keywords
%     % stringstyle=\color{red}, % style of strings
%     backgroundcolor = \color{insideBlackTerminal},
%     rulecolor=\color{red}
% }

\usepackage{lstlinebgrd}
\definecolor{grayComment}{HTML}{8D90B8}
\lstset{
language=R,                     % the language of the code
basicstyle=\ttfamily, % the size of the fonts that are used for the code
% numbers=left,                   % where to put the line-numbers
% numberstyle=\color{blue},  % the style that is used for the line-numbers
% stepnumber=1,                   % the step between two line-numbers. If it is 1, each line
% will be numbered
numbersep=5pt,                  % how far the line-numbers are from the code
backgroundcolor=\color{white},  % choose the background color. You must add \usepackage{color}
linebackgroundcolor=\color{white},
showspaces=false,               % show spaces adding particular underscores
showstringspaces=false,         % underline spaces within strings
showtabs=false,                 % show tabs within strings adding particular underscores
frame=single,                   % adds a frame around the code
rulecolor=\color{black},        % if not set, the frame-color may be changed on line-breaks within not-black text (e.g. commens (green here))
tabsize=2,                      % sets default tabsize to 2 spaces
captionpos=b,                   % sets the caption-position to bottom
breaklines=true,                % sets automatic line breaking
breakatwhitespace=false,        % sets if automatic breaks should only happen at whitespace
%   keywordstyle=\color{functionR},      % keyword style
commentstyle=\color[HTML]{9F0808},  %\color[HTML]{8D90B8},   % comment style
%   stringstyle=\color[HTML]{1D9507},      % string literal style
moredelim=**[is][\color{grayComment}]{@}{@}, % couleur manuel
literate=%
{à}{{\`a}}1
{é}{{\'e}}1
{è}{{\`e}}1
} 




















































% ---------------------------------------------
% END OF PREAMBLE
% ---------------------------------------------
