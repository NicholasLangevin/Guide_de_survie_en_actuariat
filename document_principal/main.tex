\documentclass[12pt, french]{report}
% !TEX encoding = UTF-8 Unicode
% LaTeX Preamble
% Author : Gabriel Crépeault-Cauchon
% Last update : 08/09/2019
% ---------------------------------------------
% BEGINNING OF PREAMBLE
% ---------------------------------------------
% Encoding packages
\usepackage[utf8]{inputenc}
\usepackage[T1]{fontenc}
\usepackage{babel}
\usepackage{lmodern}

% HYPERREF (URL's and Link options)
\usepackage{hyperref}
\hypersetup{colorlinks = true, urlcolor = red!60!black, linkcolor = red!60!black}

% POLICY (choose one of them)
%	\usepackage{concrete}
%	\usepackage{mathpazo}
%	\usepackage{frcursive} %% permet d'écrire en lettres attachées
 \usepackage{aeguill}
% 	\usepackage{mathptmx}
%	\usepackage{fourier} 

% Mathematics configuration
\usepackage{amsmath,amsthm,amssymb,latexsym,amsfonts}
\usepackage{empheq}
\usepackage{numprint}
\usepackage{dsfont} 


% Tcolorbox config
\usepackage{tcolorbox}
\tcbuselibrary{xparse}
\tcbuselibrary{breakable}

% Définition Boite pour exemple
\newcounter{ex}[section]
\DeclareTColorBox{exemple}{ o }% #1 parameter
{colframe=green!20!black,colback=green!5!white, % color of the box
breakable, pad at break*=0mm, % to split the box
before title = {\textbf{Exemple \stepcounter{ex} \arabic{chapter}.\arabic{section}.\arabic{ex} }},
IfValueTF = {#1}{title= {#1}}{title= \hphantom},
after title = {\large \hfill \faWrench}
}

%% Définition boite pour définition
\newcounter{def}[section]
\DeclareTColorBox{definition}{ o }% #1 parameter
{colframe=blue!60!green,colback=blue!5!white, % color of the box
breakable, pad at break*=0mm, % to split the box
before title = {\textbf{Définition \stepcounter{def} \arabic{chapter}.\arabic{section}.\arabic{def} }},
title = {#1},
after title = {\large \hfill \faBook}
}

\DeclareTColorBox{note}{ o }
    {colframe=black,
     colback=white,
     sharp corners,
     pad at break*=0mm,
     IfValueTF={#1}{title={#1}, fonttitle=\bfseries}{title=Note, fonttitle=\bfseries}}


% Graphics and picture import Packages
\usepackage{graphicx}
\usepackage{pict2e}

% insert PDF package
\usepackage{pdfpages}

% Color package
\usepackage{color, soulutf8, colortbl}

% Mathematics table
\usepackage{array}   % for \newcolumntype macro
\newcolumntype{L}{>{$}l<{$}} % math-mode version of "l" column type

% usefull shortcut for colored text
\newcommand{\orange}{\textcolor{orange}}
\newcommand{\red}{\textcolor{red}}
\newcommand{\cyan}{\textcolor{cyan}}
\newcommand{\blue}{\textcolor{blue}}
\newcommand{\green}{\textcolor{green}}
\newcommand{\darkgreen}{\textcolor{darkgreen}}
\newcommand{\purple}{\textcolor{magenta}}
\newcommand{\yellow}{\textcolor{yellow}}

% Colors define
\definecolor{darkgreen}{RGB}{37, 128, 40}
\definecolor{tocColor}{HTML}{8A2507}

% Custum enumerate & itemize Package
\usepackage{enumitem}
% French Setup for itemize function
\frenchbsetup{StandardItemLabels=true}

% Mathematics shortcut
\usepackage{cancel}
\newcommand{\reels}{\mathbb{R}}
\newcommand{\entiers}{\mathbb{Z}}
\newcommand{\naturels}{\mathbb{N}}
\newcommand{\eval}{\biggr \rvert}
\newcommand{\esp}[1]{\mathrm{E} \left[ #1 \right]} % espérance
\newcommand{\variance}[1]{\mathrm{Var} \left( #1 \right)} % variance
\newcommand{\covar}[1]{\mathrm{Cov} \left( #1 \right)} % variance
\newcommand{\prob}[1]{\Pr \left( #1 \right)} % probabilité entre parenthèses
\newcommand{\laplace}{\mathcal{L}}
\newcommand{\matr}[1]{\mathbf{#1}} % Notation matricielle
\DeclareMathOperator{\Tr}{Tr}
\newcommand{\fgp}{\mathcal{P}}
\DeclareMathOperator{\Adj}{Adj}
\newcommand{\derivee}[1]{\frac{\partial}{\partial #1}}
\newcommand{\indic}[1]{\mathds{1}_{\{ #1 \}}}
\newcommand{\VaR}[2][k]{\mathrm{VaR}_{#1}{\left( #2 \right)}}
\newcommand{\TVaR}[2][k]{\mathrm{TVaR}_{#1}{\left( #2 \right)}}


% Matricial anotation for math symbols (\bm{•})
% à enlever éventuellement, j'ai ajouté la macro \matr{} à la place.
\usepackage{bm}

% Actuarial notation package
\usepackage{actuarialsymbol}
\usepackage{actuarialangle}

% To indicate equation number on a specific line in align environment
\newcommand\numberthis{\addtocounter{equation}{1}\tag{\theequation}}

% Other shortcut
\newcommand{\p}{\paragraph{}}
\newcommand{\n}{\newline}

% source : https://tex.stackexchange.com/questions/112576/math-mode-in-tabular-without-having-to-use-everywhere



% Special symbols package
 \usepackage[tikz]{bclogo}
\usepackage{fontawesome}

% Retire l'indentation automatique de Latex
\setlength{\parindent}{0pt}

% Utilisé pour la page couverture
\usepackage[absolute]{textpos} % Textblock environement
\usepackage{anyfontsize} % Avoir un gros titre
\usepackage{titling} % Avoir un gros titre
\usepackage{changepage} % ajustwidth environement

% Pour afficher du code
\usepackage{listings}

\definecolor{codegray}{gray}{0.9}
\newcommand{\code}[1]{\colorbox{codegray}{\texttt{#1}}}

\definecolor{insideBlackTerminal}{RGB}{33,33,33}
% Set Language
% \lstset{
%     language={bash},
%     basicstyle=\small\ttfamily\color{white}, % Global Code Style
%     captionpos=b, % Position of the Caption (t for top, b for bottom)
%     extendedchars=true, % Allows 256 instead of 128 ASCII characters
%     tabsize=2, % number of spaces indented when discovering a tab 
%     columns=fixed, % make all characters equal width
%     keepspaces=true, % does not ignore spaces to fit width, convert tabs to spaces
%     showstringspaces=false, % lets spaces in strings appear as real spaces
%     breaklines=true, % wrap lines if they don't fit
%     frame=single, % draw a frame at the top, right, left and bottom of the listing
%     numberstyle=\tiny\ttfamily, % style of the line numbers
%     % commentstyle=\color{red}, % style of comments
%     % keywordstyle=\color{red}, % style of keywords
%     % stringstyle=\color{red}, % style of strings
%     backgroundcolor = \color{insideBlackTerminal},
%     rulecolor=\color{red}
% }

\usepackage{lstlinebgrd}
\definecolor{grayComment}{HTML}{8D90B8}
\lstset{
language=R,                     % the language of the code
basicstyle=\ttfamily, % the size of the fonts that are used for the code
% numbers=left,                   % where to put the line-numbers
% numberstyle=\color{blue},  % the style that is used for the line-numbers
% stepnumber=1,                   % the step between two line-numbers. If it is 1, each line
% will be numbered
numbersep=5pt,                  % how far the line-numbers are from the code
backgroundcolor=\color{white},  % choose the background color. You must add \usepackage{color}
linebackgroundcolor=\color{white},
showspaces=false,               % show spaces adding particular underscores
showstringspaces=false,         % underline spaces within strings
showtabs=false,                 % show tabs within strings adding particular underscores
frame=single,                   % adds a frame around the code
rulecolor=\color{black},        % if not set, the frame-color may be changed on line-breaks within not-black text (e.g. commens (green here))
tabsize=2,                      % sets default tabsize to 2 spaces
captionpos=b,                   % sets the caption-position to bottom
breaklines=true,                % sets automatic line breaking
breakatwhitespace=false,        % sets if automatic breaks should only happen at whitespace
%   keywordstyle=\color{functionR},      % keyword style
commentstyle=\color[HTML]{9F0808},  %\color[HTML]{8D90B8},   % comment style
%   stringstyle=\color[HTML]{1D9507},      % string literal style
moredelim=**[is][\color{grayComment}]{@}{@}, % couleur manuel
literate=%
{à}{{\`a}}1
{é}{{\'e}}1
{è}{{\`e}}1
} 




















































% ---------------------------------------------
% END OF PREAMBLE
% ---------------------------------------------

\author{Gabriel Crépeault-Cauchon et Nicholas Langevin}
\title{Guide de survie en actuariat}

\begin{document}
% Page titre et table des matières
\maketitle
\newpage
\tableofcontents
\newpage
% ------

\part{Fondements mathématiques utiles}


\chapter{Calculs}

\section{Règles de dérivation}
Dans le tableau, on utilise $k \in \reels$ ou $n$ pour parler d'une constante, $u$, $v$ ou $w$ pour parler d'une fonction.

\begin{tabular}{|L | L |}
% Voir préambule, L est un type de colonne défini (custom) qui place le texte entre $•$
\hline
\text{Fonction}	& \text{Dérivée} \\\hline \hline
f(x) = k	& f'(x) = 0 \\\hline
f(x) = k x	& f'(x) = k \\\hline
f(x) = x^{n}	& f'(x) = n x^{n-1} \\\hline
f(x) = k g(x)	& f'(x) = k g'(x) \\\hline
f(x) = g(x) \pm h(x)	& f'(x) = g'(x) \pm h'(x) \\\hline
f(x) = g(x) \cdot h(x)	& f'(x) = g'(x) \cdot h(x) + g(x) \cdot h'(x) \\\hline
f(x) = \frac{g(x)}{h(x)}	& f'(x) = \frac{g'(x) h(x) - g(x) h'(x)}{h(x)^2} \\\hline
f(x) = g(x)^{n}	& f'(x) = n \cdot g(x)^{n-1} \cdot g'(x) \\\hline
f(x) = k^{g(x)}	& f'(x) = k^{g(x)} \ln k \cdot g'(x) \\\hline
f(x) = e^{g(x)}	& f'(x) = e^{g(x)} \cdot g'(x) \\\hline
f(x) = \ln (g(x)) 	& f'(x) = \frac{g'(x)}{g(x)} \\\hline
\hline
\end{tabular}



% Chapitre sur l'algèbre linéaire
% Concepts préalables pour le cours STT-2200
\chapter{Algèbre linéaire}

\section{Définition d'un vecteur et une matrice}
\paragraph{Vecteur ligne} Un vecteur ligne $\bm{x}$est un vecteur de dimension $p \times 1$, tel que
\begin{align*}
\matr{x} = 
\begin{bmatrix}
x_1 \\
x_2 \\
... \\
x_p \\
\end{bmatrix}
\end{align*}

\paragraph{Matrice} Une matrice $\matr{A}=[a_{ij}]_{m \times n}$  de dimension $m$ lignes par $n$ colonnes , définie telle que
\begin{equation}
\label{eq:matrice-base}
\matr{A} = 
\begin{bmatrix}
a_{11}    & a_{12}    &  ...   &  a_{1n} \\
a_{21}    & a_{22}    &  ...   &  a_{2n} \\
...    & ...    &  \ddots   &  ... \\
a_{m1}    & a_{m2}    &  ...   &  a_{mn} \\
\end{bmatrix}
\end{equation}


\paragraph{Matrice carrée} Une matrice carrée $\matr{A}$ de dimensions $m \times m$ a autant de lignes que de colonnes.

\paragraph{non-négative} $\matr{A}$ est définie comme \textit{non-négative} si $\matr{x^{\top} A x }\geq 0$, $\forall \matr{x}\in \reels^n$.

\paragraph{positive} $\matr{A}$ est définie comme \textit{positive} si $\matr{x^{\top} A x} > 0$, $\forall \matr{x} \neq 0$.

\paragraph{semi-positive} $\matr{A}$ est définie comme non-négative, mais elle n'est \underline{pas} définie positive.

\paragraph{Orthogonale} $\matr{A}$ est \textit{orthogonale} si elle est non-singulière et $\matr{A}^{-1} = \matr{A}^{\top}$ (voir \autoref{ssec:matrice-inverse} pour définition de $\matr{A}^{-1}$)


\paragraph{Matrice symétrique} La mactrice $\matr{A}$ est symétrique si  $a_{ij} = a_{ji}  \ \forall i,j$, i.e
\begin{align*}
\matr{A} = 
\begin{bmatrix}
1     & 2    &  3   \\
2     & 1   & 4 \\
3   & 4   & 1 \\
\end{bmatrix}
\end{align*}

\paragraph{Matrice triangulaire (inférieure ou supérieure)} Une matrice inférieure $\matr{L}$ est constituée de 0 en dessous de la diagonale : 
\begin{align*}
\matr{L} = 
\begin{bmatrix}
1     & 2    &  3   \\
\red{0}     & 1   & 4 \\
\red{0}   & \red{0}   & 1 \\
\end{bmatrix}
\end{align*}

À l'inverse, on peut aussi avoir une matrice triangulaire supérieure $\matr{U}$ , où les éléments en haut de la diagonale sont tous égaux à 0 : 
\begin{align*}
\matr{U} = 
\begin{bmatrix}
1     & \red{0}    &  \red{0}   \\
8     & 3   & \red{0} \\
2   & 3   & 9 \\
\end{bmatrix}
\end{align*}

\paragraph{Matrice diagonale} Une matrice diagonale $\matr{D}$ a des éléments $d_{ii} > 0$ sur sa diagonale seulement. Cette matrice est à la fois triangulaire inférieure et supérieure. i.e.
\begin{align*}
D = 
\begin{bmatrix}
1     & \red{0}    &  \red{0}   \\
\red{0}     & 3   & \red{0} \\
\red{0}   & \red{0}   & 2 \\
\end{bmatrix}
\end{align*}
Un cas spécial de la matrice diagonale est la matrice identité $\matr{I}$, où $\matr{I}_{ii} = 1 ,\ \forall i$, i.e
\begin{equation}
\label{eq:matrice-identite}
\matr{I} = 
\begin{bmatrix}
1     & 0    &  0   \\
0     & 1   & 0 \\
0   & 0   & 1 \\
\end{bmatrix}
\end{equation}


\paragraph{Matrice diagonalisable} Une matrice $\matr{A}_{n\times n}$ est dite \textit{diagonalisable} s'il existe une matrice carrée $\matr{Q}_{n \times n}$ inversible (ou non-singulière) et une matrice $\matr{D}$ diagonale telle que
\begin{equation}
\label{eq:matrice-diagonalisable}
\matr{Q^{-1} A Q = D} \leftrightarrow \matr{A = Q D Q^{-1}}
\end{equation}
(Théorème sur les matrices symétriques) : Toute matrice carrée symétrique est diagonalisable apr uen matrice orthogonale $\matr{Q}$.

\section{Matrice transposée} Soit la matrice $\matr{A}$ définie en \eqref{eq:matrice-base}. On peut trouver la matrice transposée $\matr{A^{\top}}$, où $[a_{ij}] = [a_{ji]}$. \textbf{En d'autres mots, les lignes deviennent des colonnes.}

Voici quelques propriétés intéressantes avec les matrices transposées : 
\begin{itemize}
  \item $\matr{(A^\top)^\top = A}$
  \item $\matr{(A+B)^\top = A^\top + B^\top}$
  \item $\matr{(kA)^\top = kA^\top}$
  \item $\matr{(AB)^\top = B^\top A^\top}$
  \item $\matr{A^\top A}$ et $\matr{A A^\top}$ sont symétriques.
\end{itemize}

\section{Opérations matricielles} Voici une liste non-exhaustive des opérations matricielles possibles. Côté notation, $A$ et $B$ représente des matrices, $c$ re présente une constante
\begin{itemize}
\item $\matr{A + B}  = [a_{ij} + b_{ij}]$
\item $\matr{A - B}  = [a_{ij} - b_{ij}]$
\item $c\matr{A}   = [ca_{ij}]$
\item Produit matriciel :
\begin{equation}
\label{eq:produit-matriciel}
\matr{AB}   = \left[ \sum_{k=1}^p a_{ik} b_{kj}  \right]_{i \times j}
\end{equation}
,  avec $\matr{A}=[a_{ip}]$ et $B = [b_{pj}]$
\item $\matr{A(B +C) = AB + AC}$
\item $\matr{A^{-1} A = I =A A^{-1}}$, où $I$ est la matrice identité (voir \autoref{eq:matrice-identite}) et $A^{-1}$ est la matrice inverse de $A$ (voir \autoref{ssec:matrice-inverse} au besoin)
\item $(AB)^{-1} = B^{-1} A^{-1}$
\end{itemize}






\section{Trace, déterminant et matrice inverse}
\subsection{Trace d'une matrice}
\label{ssec:trace-matrice}
Soit la matrice carrée $\matr{A}$. On peut trouver la trace de cette matrice en sommant les éléments de sa diagonale, i.e.

\begin{equation}
\label{eq:trace-mat}
\Tr(\matr{A}) = \sum_{i=1}^n a_{ii}
\end{equation} 
\paragraph{Propriétés de la trace d'une matrice}
\begin{itemize}
\item $\Tr(\matr{A+B}) = \Tr(\matr{A}) + \Tr(\matr{B})$
\item $\Tr(\matr{AB}) = \Tr(\matr{BA})$ et $\Tr(\matr{ABC}) = \Tr(\matr{CAB}) = \Tr(\matr{BCA})$
\end{itemize}

\subsection{Déterminant d'une matrice}
\label{ssec:det-matrice}
Soit la matrice carrée $\matr{A}$. On peut trouver le déterminant de $\matr{A}$, noté $\det(\matr{A})$ ou $|\matr{A}|$, avec
\begin{equation}
\det(\matr{A}) =
\begin{vmatrix}
a & b \\
c & d \\
\end{vmatrix} = ad - bc
\end{equation}
De façon générale, lorsque les dimensions de la matrice carrée sont supérieures à 2, on a
\begin{equation}
\label{eq:det-matrice}
\det(A) = \sum_{j=1}^n a_{ij} C_{ij}
\end{equation}
avec $1 \le i \le n$ où $C_{ij}  = (-1)^{i+j} M_{ij}$ et $M_{ij}$ est le déterminant de la nouvelle matrice en enlevant la ligne $i$ et la colonne $j$.

Si la matrice $\matr{A}$ est inversible (ou non-singulière, voir la \autoref{ssec:matrice-inverse}), alors le déterminant aura les propriétés suivantes : 
\begin{itemize}
\item $\det(A^\top)  = \det(A)$
\item $\det(kA)   = k^n \det(A)$
\item $\det(A + B)  \neq \det(A) + \det(B)$
\item $\det(AB)   = \det(A) \det(B)$
\item $\det(A^{-1})  = \frac{1}{\det(AB)} = \det(A)^{-1}$
\end{itemize}

\subsection{Matrice inverse}
\label{ssec:matrice-inverse}
Soit la matrice carrée $\matr{A}$. On peut trouver la matrice inverse $\matr{A}^{-1}$ telle que

\begin{equation}
\label{eq:mat-inverse}
\matr{A}^{-1} = \frac{1}{\det(\matr{A})} \Adj(\matr{A})
\end{equation}
où $\Adj(\matr{A})   = [C_{ij}]_{m \times n}^T$ et  $C_{ij}   = (-1)^{i+j} M_{ij}$.


\section{Décomposition LDU de Choleski}
Soit $\matr{A}$ une matrice carrée symétrique définie positive. Alors, il existe une décomposition unique telle que
\begin{equation}
\label{eq:decomp-choleski}
\matr{A} = \matr{LDU}
\end{equation}
où $\matr{L, D, U}$ sont respectivement des matrices triangulaire inférieure, triangulaire supérieure et diagonale.

\hl{Cette décomposition peut être fortement utile en programmation lorsqu'on fait des opérations sur des matrices, afin de limiter le nombre d'opérations.}

\section{Vecteurs et valeurs propres}
\subsection{Définition}
Soit $\matr{A}$ une matrice carrée. On dit que $\lambda$ est une \textit{valeur propre} de $\matr{A}$ s'il existe un vecteur $\matr{x} \neq 0$ tel que
\begin{equation}
\label{eq:vecteur-propre}
\matr{Ax = \lambda x}
\end{equation}
On appelle le vecteur $\matr{x}$ un \textit{vecteur propre} correspondant à la valeur propre $\lambda$. De plus, l'ensemble des nombres réels $\lambda$ satisfaisant l'\autoref{eq:vecteur-propre} est appelé \textit{spectre} de la matrice $\matr{A}$.

\subsection{Propriétés intéressantes}
Les vecteurs propres et valeurs propres permettent d'avoir plusieurs propriétés appréciables, notamment : 
\begin{itemize}
\item Si $\matr{x}$ est un vecteur propre de $\matr{A}$ correspondant à la valeur propre $\lambda$, alors $c\matr{x}$ sera également un vecteur propre de $\matr{A}$ correspondant à $\lambda$.
\item Si $\matr{A}$ est symétrique et $\matr{x_1}$ et $\matr{x_2}$ sont des vecteurs propres correspondant à des valeurs propres différentes de $\matr{A}$, alors $\matr{x_1}$ et $\matr{x_2}$ sont des vecteurs ortogonaux, i.e. $\matr{x_{1}^{\top} x_2}  =0$.
\item Si $\matr{A}$ a les valeurs propres (pas nécessairement distinctes) $\lambda_1, \dots, \lambda_n$, alors $\det(\matr{A}) = \prod_{i=1}^{n} \lambda_i$ et $\Tr(\matr{A}) = \sum_{i=1}^{n} \lambda_i$.
\end{itemize}

\subsection{Décomposition spectrale} Soit $\matr{A}_{n\times n}$ une matrice symétrique avec les $n$ valeurs propres $\lambda_1, \dots, \lambda_n$. Il existe une matrice orthogonale $\matr{Q}$ telle que
\begin{equation}
\label{eq:decomposition-spectrale}
\matr{A = Q \Lambda Q^{\top}}
\end{equation}
avec $\matr{\Lambda} = diag(\lambda_1, \dots , \lambda_n)$. Cette décomposition est fort utile lorsqu'on veut faire des produits matriciels successifs de la même matrice (appliqué directement dans les chaînes de Markov, voir \autoref{sec:chaine-markov}) : 
\begin{align*}
\matr{AA} & \matr{= Q \Lambda \underbrace{Q^{\top} Q}_{=I} \Lambda Q^{\top} } \\
 & = \matr{Q \Lambda^2 Q^{\top}}
\end{align*}

\section{Dérivées de matrice ou vecteurs}
Voici quelques entités pratiques : 
\begin{align*}
\derivee{\matr{v}} \matr{w^{\top} v} = w
\end{align*}
\begin{align*}
\derivee{\matr{v}} = \matr{v^{\top} A v = (A + A^{\top})v}
\end{align*}















% Matière vue à l'Université
\part{Matière vue dans le baccalauréat en actuariat}

% Chapitre sur les concepts de probabilité (ACT-1002) et statistiques (ACT-2000)
\chapter{Probabilités et statistiques}

\section{Concepts de probabilité de base}

\subsection{Probabilité conditionnelle}
\paragraph{Définition de base}
\begin{equation}
\label{eq:prob-cond}
\prob{A|B} = \frac{\prob{A \cap B}}{\prob{B}}
\end{equation}

\paragraph{Loi des probabilités totales} Soit $E_i$ le \textit{outcome} $i$ parmi l'ensemble des $n$ \textit{outcome} possibles de l'évènement $E$, alors, on peut représenter la probabilité que l'évènement $A$ survienne comme
\begin{equation}
\label{eq:loi-prob-totales}
\prob{A} = \sum_{i=1}^{n} \prob{A | E_i} \prob{E_i}
\end{equation}
avec $\sum_{i=1}^{n} \prob{E_i} = 1$.

\paragraph{Relation importante} de l'\autoref{eq:prob-cond}, on peut représenter $\prob{A|B}$ comme
\begin{equation}
\label{eq:prob-cond-2}
\prob{A|B} = \frac{\prob{B|A} \prob{A}}{\prob{B}}
\end{equation}

\subsection{Théorème de Bayes} En combinant l'\autoref{eq:prob-cond-2} et la loi des probabilités totales (l'\autoref{eq:loi-prob-totales}), on obtient le théorème de Bayes : 
\begin{equation}
\prob{A|B} = \frac{\prob{B|A} \prob{A}}{\sum_{i=1}^{n} \prob{B | A_i} \prob{A_i}}
\end{equation}


\section{Définition d'une variable aléatoire}

\section{Distribution d'une variable aléatoire}
Fonction de densité, répartition, survie, hazard rate, etc.

\section{Moments et quantités importantes}
Espérance, variance, covariance, coefficient de variation, corrélation

\paragraph{Espérance} Soit une v.a. $X$ (continue ou discrète). Son espérance est définie telle que
\begin{equation}
\label{eq:esp-univarie}
\esp{X} = \mu = \sum_{x=0}^{\infty} x \prob{X = x} = \int_{0}^{\infty} x f_X(x) dx
\end{equation}
L'espérance d'une fonction de la v.a $X$ est
\begin{equation}
\label{eq:esp-fct-univarie}
\esp{g(X)} = \sum_{x=0}^{\infty} g(x) \prob{X = x} = \int_{0}^{\infty} g(x) f_X(x) dx
\end{equation}

\paragraph{Variance}
\begin{equation}
\label{eq:variance}
\variance{X} = \sigma^2 = \esp{(X - \esp{X})^2} = \esp{X^2} - \esp{X}^2
\end{equation}
quelques propriétés à savoir : 
\begin{align*}
\variance{aX} 		& = a^2 \variance{X} \\
\variance{X + b}	& = \variance{X}
\end{align*}

\paragraph{Covariance}
\begin{equation}
\label{eq:covariance}
\covar{X,Y} =  \sigma_{X,Y} = \esp{(X-\esp{X})(Y - \esp{Y})} = \esp{XY} - \esp{X} \esp{Y}
\end{equation}



\section{Distribution de probabilité qui reviennent souvent}
Un tableau récapitulatif des différentes distribution de probabilité est disponible à l'


\chapter{Mathématiques financières}


\chapter{Processus aléatoire}

\section{Chaîne de Markov}
\label{sec:chaine-markov}



\part{Matière pour les examens professionnels}



\appendix
\chapter{Principales distribution de probabilité utilisées}
introduction


\end{document}