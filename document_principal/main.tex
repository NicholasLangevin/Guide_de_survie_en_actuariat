\documentclass[12pt, french]{report}
% !TEX encoding = UTF-8 Unicode
% LaTeX Preamble
% Author : Gabriel Crépeault-Cauchon
% Last update : 08/09/2019
% ---------------------------------------------
% BEGINNING OF PREAMBLE
% ---------------------------------------------
% Encoding packages
\usepackage[utf8]{inputenc}
\usepackage[T1]{fontenc}
\usepackage{babel}
\usepackage{lmodern}

% HYPERREF (URL's and Link options)
\usepackage{hyperref}
\hypersetup{colorlinks = true, urlcolor = red!60!black, linkcolor = red!60!black}

% POLICY (choose one of them)
%	\usepackage{concrete}
%	\usepackage{mathpazo}
%	\usepackage{frcursive} %% permet d'écrire en lettres attachées
 \usepackage{aeguill}
% 	\usepackage{mathptmx}
%	\usepackage{fourier} 

% Mathematics configuration
\usepackage{amsmath,amsthm,amssymb,latexsym,amsfonts}
\usepackage{empheq}
\usepackage{numprint}
\usepackage{dsfont} 


% Tcolorbox config
\usepackage{tcolorbox}
\tcbuselibrary{xparse}
\tcbuselibrary{breakable}

% Définition Boite pour exemple
\newcounter{ex}[section]
\DeclareTColorBox{exemple}{ o }% #1 parameter
{colframe=green!20!black,colback=green!5!white, % color of the box
breakable, pad at break*=0mm, % to split the box
before title = {\textbf{Exemple \stepcounter{ex} \arabic{chapter}.\arabic{section}.\arabic{ex} }},
IfValueTF = {#1}{title= {#1}}{title= \hphantom},
after title = {\large \hfill \faWrench}
}

%% Définition boite pour définition
\newcounter{def}[section]
\DeclareTColorBox{definition}{ o }% #1 parameter
{colframe=blue!60!green,colback=blue!5!white, % color of the box
breakable, pad at break*=0mm, % to split the box
before title = {\textbf{Définition \stepcounter{def} \arabic{chapter}.\arabic{section}.\arabic{def} }},
title = {#1},
after title = {\large \hfill \faBook}
}

\DeclareTColorBox{note}{ o }
    {colframe=black,
     colback=white,
     sharp corners,
     pad at break*=0mm,
     IfValueTF={#1}{title={#1}, fonttitle=\bfseries}{title=Note, fonttitle=\bfseries}}


% Graphics and picture import Packages
\usepackage{graphicx}
\usepackage{pict2e}

% insert PDF package
\usepackage{pdfpages}

% Color package
\usepackage{color, soulutf8, colortbl}

% Mathematics table
\usepackage{array}   % for \newcolumntype macro
\newcolumntype{L}{>{$}l<{$}} % math-mode version of "l" column type

% usefull shortcut for colored text
\newcommand{\orange}{\textcolor{orange}}
\newcommand{\red}{\textcolor{red}}
\newcommand{\cyan}{\textcolor{cyan}}
\newcommand{\blue}{\textcolor{blue}}
\newcommand{\green}{\textcolor{green}}
\newcommand{\darkgreen}{\textcolor{darkgreen}}
\newcommand{\purple}{\textcolor{magenta}}
\newcommand{\yellow}{\textcolor{yellow}}

% Colors define
\definecolor{darkgreen}{RGB}{37, 128, 40}
\definecolor{tocColor}{HTML}{8A2507}

% Custum enumerate & itemize Package
\usepackage{enumitem}
% French Setup for itemize function
\frenchbsetup{StandardItemLabels=true}

% Mathematics shortcut
\usepackage{cancel}
\newcommand{\reels}{\mathbb{R}}
\newcommand{\entiers}{\mathbb{Z}}
\newcommand{\naturels}{\mathbb{N}}
\newcommand{\eval}{\biggr \rvert}
\newcommand{\esp}[1]{\mathrm{E} \left[ #1 \right]} % espérance
\newcommand{\variance}[1]{\mathrm{Var} \left( #1 \right)} % variance
\newcommand{\covar}[1]{\mathrm{Cov} \left( #1 \right)} % variance
\newcommand{\prob}[1]{\Pr \left( #1 \right)} % probabilité entre parenthèses
\newcommand{\laplace}{\mathcal{L}}
\newcommand{\matr}[1]{\mathbf{#1}} % Notation matricielle
\DeclareMathOperator{\Tr}{Tr}
\newcommand{\fgp}{\mathcal{P}}
\DeclareMathOperator{\Adj}{Adj}
\newcommand{\derivee}[1]{\frac{\partial}{\partial #1}}
\newcommand{\indic}[1]{\mathds{1}_{\{ #1 \}}}
\newcommand{\VaR}[2][k]{\mathrm{VaR}_{#1}{\left( #2 \right)}}
\newcommand{\TVaR}[2][k]{\mathrm{TVaR}_{#1}{\left( #2 \right)}}


% Matricial anotation for math symbols (\bm{•})
% à enlever éventuellement, j'ai ajouté la macro \matr{} à la place.
\usepackage{bm}

% Actuarial notation package
\usepackage{actuarialsymbol}
\usepackage{actuarialangle}

% To indicate equation number on a specific line in align environment
\newcommand\numberthis{\addtocounter{equation}{1}\tag{\theequation}}

% Other shortcut
\newcommand{\p}{\paragraph{}}
\newcommand{\n}{\newline}

% source : https://tex.stackexchange.com/questions/112576/math-mode-in-tabular-without-having-to-use-everywhere



% Special symbols package
 \usepackage[tikz]{bclogo}
\usepackage{fontawesome}

% Retire l'indentation automatique de Latex
\setlength{\parindent}{0pt}

% Utilisé pour la page couverture
\usepackage[absolute]{textpos} % Textblock environement
\usepackage{anyfontsize} % Avoir un gros titre
\usepackage{titling} % Avoir un gros titre
\usepackage{changepage} % ajustwidth environement

% Pour afficher du code
\usepackage{listings}

\definecolor{codegray}{gray}{0.9}
\newcommand{\code}[1]{\colorbox{codegray}{\texttt{#1}}}

\definecolor{insideBlackTerminal}{RGB}{33,33,33}
% Set Language
% \lstset{
%     language={bash},
%     basicstyle=\small\ttfamily\color{white}, % Global Code Style
%     captionpos=b, % Position of the Caption (t for top, b for bottom)
%     extendedchars=true, % Allows 256 instead of 128 ASCII characters
%     tabsize=2, % number of spaces indented when discovering a tab 
%     columns=fixed, % make all characters equal width
%     keepspaces=true, % does not ignore spaces to fit width, convert tabs to spaces
%     showstringspaces=false, % lets spaces in strings appear as real spaces
%     breaklines=true, % wrap lines if they don't fit
%     frame=single, % draw a frame at the top, right, left and bottom of the listing
%     numberstyle=\tiny\ttfamily, % style of the line numbers
%     % commentstyle=\color{red}, % style of comments
%     % keywordstyle=\color{red}, % style of keywords
%     % stringstyle=\color{red}, % style of strings
%     backgroundcolor = \color{insideBlackTerminal},
%     rulecolor=\color{red}
% }

\usepackage{lstlinebgrd}
\definecolor{grayComment}{HTML}{8D90B8}
\lstset{
language=R,                     % the language of the code
basicstyle=\ttfamily, % the size of the fonts that are used for the code
% numbers=left,                   % where to put the line-numbers
% numberstyle=\color{blue},  % the style that is used for the line-numbers
% stepnumber=1,                   % the step between two line-numbers. If it is 1, each line
% will be numbered
numbersep=5pt,                  % how far the line-numbers are from the code
backgroundcolor=\color{white},  % choose the background color. You must add \usepackage{color}
linebackgroundcolor=\color{white},
showspaces=false,               % show spaces adding particular underscores
showstringspaces=false,         % underline spaces within strings
showtabs=false,                 % show tabs within strings adding particular underscores
frame=single,                   % adds a frame around the code
rulecolor=\color{black},        % if not set, the frame-color may be changed on line-breaks within not-black text (e.g. commens (green here))
tabsize=2,                      % sets default tabsize to 2 spaces
captionpos=b,                   % sets the caption-position to bottom
breaklines=true,                % sets automatic line breaking
breakatwhitespace=false,        % sets if automatic breaks should only happen at whitespace
%   keywordstyle=\color{functionR},      % keyword style
commentstyle=\color[HTML]{9F0808},  %\color[HTML]{8D90B8},   % comment style
%   stringstyle=\color[HTML]{1D9507},      % string literal style
moredelim=**[is][\color{grayComment}]{@}{@}, % couleur manuel
literate=%
{à}{{\`a}}1
{é}{{\'e}}1
{è}{{\`e}}1
} 




















































% ---------------------------------------------
% END OF PREAMBLE
% ---------------------------------------------

\author{Gabriel Crépeault-Cauchon et Nicholas Langevin}
\title{Guide de survie en actuariat}

\begin{document}
% Page titre et table des matières
\maketitle
\newpage
\tableofcontents
\newpage
% ------

\part{Fondements mathématiques utiles}


\chapter{Calculs}

\section{Règles de dérivation}
Dans le tableau, on utilise $k \in \reels$ ou $n$ pour parler d'une constante, $u$, $v$ ou $w$ pour parler d'une fonction.

\begin{tabular}{|L | L |}
% Voir préambule, L est un type de colonne défini (custom) qui place le texte entre $•$
\hline
\text{Fonction}	& \text{Dérivée} \\\hline \hline
f(x) = k	& f'(x) = 0 \\\hline
f(x) = k x	& f'(x) = k \\\hline
f(x) = x^{n}	& f'(x) = n x^{n-1} \\\hline
f(x) = k g(x)	& f'(x) = k g'(x) \\\hline
f(x) = g(x) \pm h(x)	& f'(x) = g'(x) \pm h'(x) \\\hline
f(x) = g(x) \cdot h(x)	& f'(x) = g'(x) \cdot h(x) + g(x) \cdot h'(x) \\\hline
f(x) = \frac{g(x)}{h(x)}	& f'(x) = \frac{g'(x) h(x) - g(x) h'(x)}{h(x)^2} \\\hline
f(x) = g(x)^{n}	& f'(x) = n \cdot g(x)^{n-1} \cdot g'(x) \\\hline
f(x) = k^{g(x)}	& f'(x) = k^{g(x)} \ln k \cdot g'(x) \\\hline
f(x) = e^{g(x)}	& f'(x) = e^{g(x)} \cdot g'(x) \\\hline
f(x) = \ln (g(x)) 	& f'(x) = \frac{g'(x)}{g(x)} \\\hline
\hline
\end{tabular}



% Chapitre sur l'algèbre linéaire
% Concepts préalables pour le cours STT-2200
\chapter{Algèbre linéaire}

\section{Définition d'un vecteur et une matrice}
\paragraph{Vecteur ligne} Un vecteur ligne $\bm{x}$est un vecteur de dimension $p \times 1$, tel que
\begin{align*}
\matr{x} = 
\begin{bmatrix}
x_1 \\
x_2 \\
... \\
x_p \\
\end{bmatrix}
\end{align*}

\paragraph{Matrice} Une matrice $\matr{A}=[a_{ij}]_{m \times n}$  de dimension $m$ lignes par $n$ colonnes , définie telle que
\begin{equation}
\label{eq:matrice-base}
\matr{A} = 
\begin{bmatrix}
a_{11}    & a_{12}    &  ...   &  a_{1n} \\
a_{21}    & a_{22}    &  ...   &  a_{2n} \\
...    & ...    &  \ddots   &  ... \\
a_{m1}    & a_{m2}    &  ...   &  a_{mn} \\
\end{bmatrix}
\end{equation}


\paragraph{Matrice carrée} Une matrice carrée $\matr{A}$ de dimensions $m \times m$ a autant de lignes que de colonnes.

\paragraph{non-négative} $\matr{A}$ est définie comme \textit{non-négative} si $\matr{x^{\top} A x }\geq 0$, $\forall \matr{x}\in \reels^n$.

\paragraph{positive} $\matr{A}$ est définie comme \textit{positive} si $\matr{x^{\top} A x} > 0$, $\forall \matr{x} \neq 0$.

\paragraph{semi-positive} $\matr{A}$ est définie comme non-négative, mais elle n'est \underline{pas} définie positive.

\paragraph{Orthogonale} $\matr{A}$ est \textit{orthogonale} si elle est non-singulière et $\matr{A}^{-1} = \matr{A}^{\top}$ (voir \autoref{ssec:matrice-inverse} pour définition de $\matr{A}^{-1}$)


\paragraph{Matrice symétrique} La mactrice $\matr{A}$ est symétrique si  $a_{ij} = a_{ji}  \ \forall i,j$, i.e
\begin{align*}
\matr{A} = 
\begin{bmatrix}
1     & 2    &  3   \\
2     & 1   & 4 \\
3   & 4   & 1 \\
\end{bmatrix}
\end{align*}

\paragraph{Matrice triangulaire (inférieure ou supérieure)} Une matrice inférieure $\matr{L}$ est constituée de 0 en dessous de la diagonale : 
\begin{align*}
\matr{L} = 
\begin{bmatrix}
1     & 2    &  3   \\
\red{0}     & 1   & 4 \\
\red{0}   & \red{0}   & 1 \\
\end{bmatrix}
\end{align*}

À l'inverse, on peut aussi avoir une matrice triangulaire supérieure $\matr{U}$ , où les éléments en haut de la diagonale sont tous égaux à 0 : 
\begin{align*}
\matr{U} = 
\begin{bmatrix}
1     & \red{0}    &  \red{0}   \\
8     & 3   & \red{0} \\
2   & 3   & 9 \\
\end{bmatrix}
\end{align*}

\paragraph{Matrice diagonale} Une matrice diagonale $\matr{D}$ a des éléments $d_{ii} > 0$ sur sa diagonale seulement. Cette matrice est à la fois triangulaire inférieure et supérieure. i.e.
\begin{align*}
D = 
\begin{bmatrix}
1     & \red{0}    &  \red{0}   \\
\red{0}     & 3   & \red{0} \\
\red{0}   & \red{0}   & 2 \\
\end{bmatrix}
\end{align*}
Un cas spécial de la matrice diagonale est la matrice identité $\matr{I}$, où $\matr{I}_{ii} = 1 ,\ \forall i$, i.e
\begin{equation}
\label{eq:matrice-identite}
\matr{I} = 
\begin{bmatrix}
1     & 0    &  0   \\
0     & 1   & 0 \\
0   & 0   & 1 \\
\end{bmatrix}
\end{equation}


\paragraph{Matrice diagonalisable} Une matrice $\matr{A}_{n\times n}$ est dite \textit{diagonalisable} s'il existe une matrice carrée $\matr{Q}_{n \times n}$ inversible (ou non-singulière) et une matrice $\matr{D}$ diagonale telle que
\begin{equation}
\label{eq:matrice-diagonalisable}
\matr{Q^{-1} A Q = D} \leftrightarrow \matr{A = Q D Q^{-1}}
\end{equation}
(Théorème sur les matrices symétriques) : Toute matrice carrée symétrique est diagonalisable apr uen matrice orthogonale $\matr{Q}$.

\section{Matrice transposée} Soit la matrice $\matr{A}$ définie en \eqref{eq:matrice-base}. On peut trouver la matrice transposée $\matr{A^{\top}}$, où $[a_{ij}] = [a_{ji]}$. \textbf{En d'autres mots, les lignes deviennent des colonnes.}

Voici quelques propriétés intéressantes avec les matrices transposées : 
\begin{itemize}
  \item $\matr{(A^\top)^\top = A}$
  \item $\matr{(A+B)^\top = A^\top + B^\top}$
  \item $\matr{(kA)^\top = kA^\top}$
  \item $\matr{(AB)^\top = B^\top A^\top}$
  \item $\matr{A^\top A}$ et $\matr{A A^\top}$ sont symétriques.
\end{itemize}

\section{Opérations matricielles} Voici une liste non-exhaustive des opérations matricielles possibles. Côté notation, $A$ et $B$ représente des matrices, $c$ re présente une constante
\begin{itemize}
\item $\matr{A + B}  = [a_{ij} + b_{ij}]$
\item $\matr{A - B}  = [a_{ij} - b_{ij}]$
\item $c\matr{A}   = [ca_{ij}]$
\item Produit matriciel :
\begin{equation}
\label{eq:produit-matriciel}
\matr{AB}   = \left[ \sum_{k=1}^p a_{ik} b_{kj}  \right]_{i \times j}
\end{equation}
,  avec $\matr{A}=[a_{ip}]$ et $B = [b_{pj}]$
\item $\matr{A(B +C) = AB + AC}$
\item $\matr{A^{-1} A = I =A A^{-1}}$, où $I$ est la matrice identité (voir \autoref{eq:matrice-identite}) et $A^{-1}$ est la matrice inverse de $A$ (voir \autoref{ssec:matrice-inverse} au besoin)
\item $(AB)^{-1} = B^{-1} A^{-1}$
\end{itemize}






\section{Trace, déterminant et matrice inverse}
\subsection{Trace d'une matrice}
\label{ssec:trace-matrice}
Soit la matrice carrée $\matr{A}$. On peut trouver la trace de cette matrice en sommant les éléments de sa diagonale, i.e.

\begin{equation}
\label{eq:trace-mat}
\Tr(\matr{A}) = \sum_{i=1}^n a_{ii}
\end{equation} 
\paragraph{Propriétés de la trace d'une matrice}
\begin{itemize}
\item $\Tr(\matr{A+B}) = \Tr(\matr{A}) + \Tr(\matr{B})$
\item $\Tr(\matr{AB}) = \Tr(\matr{BA})$ et $\Tr(\matr{ABC}) = \Tr(\matr{CAB}) = \Tr(\matr{BCA})$
\end{itemize}

\subsection{Déterminant d'une matrice}
\label{ssec:det-matrice}
Soit la matrice carrée $\matr{A}$. On peut trouver le déterminant de $\matr{A}$, noté $\det(\matr{A})$ ou $|\matr{A}|$, avec
\begin{equation}
\det(\matr{A}) =
\begin{vmatrix}
a & b \\
c & d \\
\end{vmatrix} = ad - bc
\end{equation}
De façon générale, lorsque les dimensions de la matrice carrée sont supérieures à 2, on a
\begin{equation}
\label{eq:det-matrice}
\det(A) = \sum_{j=1}^n a_{ij} C_{ij}
\end{equation}
avec $1 \le i \le n$ où $C_{ij}  = (-1)^{i+j} M_{ij}$ et $M_{ij}$ est le déterminant de la nouvelle matrice en enlevant la ligne $i$ et la colonne $j$.

Si la matrice $\matr{A}$ est inversible (ou non-singulière, voir la \autoref{ssec:matrice-inverse}), alors le déterminant aura les propriétés suivantes : 
\begin{itemize}
\item $\det(A^\top)  = \det(A)$
\item $\det(kA)   = k^n \det(A)$
\item $\det(A + B)  \neq \det(A) + \det(B)$
\item $\det(AB)   = \det(A) \det(B)$
\item $\det(A^{-1})  = \frac{1}{\det(AB)} = \det(A)^{-1}$
\end{itemize}

\subsection{Matrice inverse}
\label{ssec:matrice-inverse}
Soit la matrice carrée $\matr{A}$. On peut trouver la matrice inverse $\matr{A}^{-1}$ telle que

\begin{equation}
\label{eq:mat-inverse}
\matr{A}^{-1} = \frac{1}{\det(\matr{A})} \Adj(\matr{A})
\end{equation}
où $\Adj(\matr{A})   = [C_{ij}]_{m \times n}^T$ et  $C_{ij}   = (-1)^{i+j} M_{ij}$.


\section{Décomposition LDU de Choleski}
Soit $\matr{A}$ une matrice carrée symétrique définie positive. Alors, il existe une décomposition unique telle que
\begin{equation}
\label{eq:decomp-choleski}
\matr{A} = \matr{LDU}
\end{equation}
où $\matr{L, D, U}$ sont respectivement des matrices triangulaire inférieure, triangulaire supérieure et diagonale.

\hl{Cette décomposition peut être fortement utile en programmation lorsqu'on fait des opérations sur des matrices, afin de limiter le nombre d'opérations.}

\section{Vecteurs et valeurs propres}
\subsection{Définition}
Soit $\matr{A}$ une matrice carrée. On dit que $\lambda$ est une \textit{valeur propre} de $\matr{A}$ s'il existe un vecteur $\matr{x} \neq 0$ tel que
\begin{equation}
\label{eq:vecteur-propre}
\matr{Ax = \lambda x}
\end{equation}
On appelle le vecteur $\matr{x}$ un \textit{vecteur propre} correspondant à la valeur propre $\lambda$. De plus, l'ensemble des nombres réels $\lambda$ satisfaisant l'\autoref{eq:vecteur-propre} est appelé \textit{spectre} de la matrice $\matr{A}$.

\subsection{Propriétés intéressantes}
Les vecteurs propres et valeurs propres permettent d'avoir plusieurs propriétés appréciables, notamment : 
\begin{itemize}
\item Si $\matr{x}$ est un vecteur propre de $\matr{A}$ correspondant à la valeur propre $\lambda$, alors $c\matr{x}$ sera également un vecteur propre de $\matr{A}$ correspondant à $\lambda$.
\item Si $\matr{A}$ est symétrique et $\matr{x_1}$ et $\matr{x_2}$ sont des vecteurs propres correspondant à des valeurs propres différentes de $\matr{A}$, alors $\matr{x_1}$ et $\matr{x_2}$ sont des vecteurs ortogonaux, i.e. $\matr{x_{1}^{\top} x_2}  =0$.
\item Si $\matr{A}$ a les valeurs propres (pas nécessairement distinctes) $\lambda_1, \dots, \lambda_n$, alors $\det(\matr{A}) = \prod_{i=1}^{n} \lambda_i$ et $\Tr(\matr{A}) = \sum_{i=1}^{n} \lambda_i$.
\end{itemize}

\subsection{Décomposition spectrale} Soit $\matr{A}_{n\times n}$ une matrice symétrique avec les $n$ valeurs propres $\lambda_1, \dots, \lambda_n$. Il existe une matrice orthogonale $\matr{Q}$ telle que
\begin{equation}
\label{eq:decomposition-spectrale}
\matr{A = Q \Lambda Q^{\top}}
\end{equation}
avec $\matr{\Lambda} = diag(\lambda_1, \dots , \lambda_n)$. Cette décomposition est fort utile lorsqu'on veut faire des produits matriciels successifs de la même matrice (appliqué directement dans les chaînes de Markov, voir \autoref{sec:chaine-markov}) : 
\begin{align*}
\matr{AA} & \matr{= Q \Lambda \underbrace{Q^{\top} Q}_{=I} \Lambda Q^{\top} } \\
 & = \matr{Q \Lambda^2 Q^{\top}}
\end{align*}

\section{Dérivées de matrice ou vecteurs}
Voici quelques entités pratiques : 
\begin{align*}
\derivee{\matr{v}} \matr{w^{\top} v} = w
\end{align*}
\begin{align*}
\derivee{\matr{v}} = \matr{v^{\top} A v = (A + A^{\top})v}
\end{align*}















% Matière vue à l'Université
\part{Matière vue dans le baccalauréat en actuariat}

% Chapitre sur les concepts de probabilité (ACT-1002) et statistiques (ACT-2000)
\chapter{Probabilités et statistiques}

\section{Concepts de probabilité de base}

\subsection{Probabilité conditionnelle}
\paragraph{Définition de base}
\begin{equation}
\label{eq:prob-cond}
\prob{A|B} = \frac{\prob{A \cap B}}{\prob{B}}
\end{equation}

\paragraph{Loi des probabilités totales} Soit $E_i$ le \textit{outcome} $i$ parmi l'ensemble des $n$ \textit{outcome} possibles de l'évènement $E$, alors, on peut représenter la probabilité que l'évènement $A$ survienne comme
\begin{equation}
\label{eq:loi-prob-totales}
\prob{A} = \sum_{i=1}^{n} \prob{A | E_i} \prob{E_i}
\end{equation}
avec $\sum_{i=1}^{n} \prob{E_i} = 1$.

\paragraph{Relation importante} de l'\autoref{eq:prob-cond}, on peut représenter $\prob{A|B}$ comme
\begin{equation}
\label{eq:prob-cond-2}
\prob{A|B} = \frac{\prob{B|A} \prob{A}}{\prob{B}}
\end{equation}

\subsection{Théorème de Bayes} En combinant l'\autoref{eq:prob-cond-2} et la loi des probabilités totales (l'\autoref{eq:loi-prob-totales}), on obtient le théorème de Bayes : 
\begin{equation}
\prob{A|B} = \frac{\prob{B|A} \prob{A}}{\sum_{i=1}^{n} \prob{B | A_i} \prob{A_i}}
\end{equation}


\section{Définition d'une variable aléatoire}

\section{Distribution d'une variable aléatoire}
Fonction de densité, répartition, survie, hazard rate, etc.

\section{Moments et quantités importantes}
Espérance, variance, covariance, coefficient de variation, corrélation

\paragraph{Espérance} Soit une v.a. $X$ (continue ou discrète). Son espérance est définie telle que
\begin{equation}
\label{eq:esp-univarie}
\esp{X} = \mu = \sum_{x=0}^{\infty} x \prob{X = x} = \int_{0}^{\infty} x f_X(x) dx
\end{equation}
L'espérance d'une fonction de la v.a $X$ est
\begin{equation}
\label{eq:esp-fct-univarie}
\esp{g(X)} = \sum_{x=0}^{\infty} g(x) \prob{X = x} = \int_{0}^{\infty} g(x) f_X(x) dx
\end{equation}

\paragraph{Variance}
\begin{equation}
\label{eq:variance}
\variance{X} = \sigma^2 = \esp{(X - \esp{X})^2} = \esp{X^2} - \esp{X}^2
\end{equation}
quelques propriétés à savoir : 
\begin{align*}
\variance{aX} 		& = a^2 \variance{X} \\
\variance{X + b}	& = \variance{X}
\end{align*}

\paragraph{Covariance}
\begin{equation}
\label{eq:covariance}
\covar{X,Y} =  \sigma_{X,Y} = \esp{(X-\esp{X})(Y - \esp{Y})} = \esp{XY} - \esp{X} \esp{Y}
\end{equation}



\section{Distribution de probabilité qui reviennent souvent}
Un tableau récapitulatif des différentes distribution de probabilité est disponible à l'


\chapter{Mathématiques financières}


\chapter{Processus aléatoire}

\section{Chaîne de Markov}
\label{sec:chaine-markov}



\part{Matière pour les examens professionnels}



\appendix
\chapter{Principales distribution de probabilité utilisées}
introduction


% Annexe des preuves

% \chapter{Preuves}
\section{Stop-Loss ($\pi_X(d)$)}
\label{preuve:stoploss}
Dans un contexte continu,
\begin{align*}
\pi_X(d) = \int_d^\infty \overline{F}(d) du
\end{align*}

\begin{proof}
\begin{align*}
\pi_X(d)		& = E[\max(X-d,0)] \\
	& = \int_0^\infty \max(x - d, 0) F_X(x) dx \\
	& = \int_0^\infty (x-d) 1_{\{X > d \}} f_X(x) dx \\
	& = \int_d^\infty (x-d) f_X(x) dx \\
	& = \int_d^\infty x f_X(x) dx - \int_d^\infty d f_X(x) dx \\
\end{align*}
On doit alors faire une intégration par partie, en posant

\begin{displaymath}
\begin{aligned}
u	& = x		& du	 & = dx \\
dv	& = dF_X(x)	& v	 & = -S(x) \\	
\end{aligned}
\end{displaymath}
Note : si on fait tendre $S(x)$ vers l'infini, ça va tendre plus rapidement vers 0 que $x$ seul.
\begin{align*}
\pi_X(d)	& = -xS(x) \eval_d^\infty - \int_d^\infty - S(x) dx - d \big(F(\infty) - F(d) \big) \\
	& = 0 + \cancel{dS(d)} + \int_d^\infty S(x) dx - \cancel{dS(d)} \\
	& = \int_d^\infty S(x) dx \\
\end{align*}
\end{proof}
Il existe aussi le contexte discret : 
\begin{align*}
\pi_X(d) = \sum_{k=d}^\infty S(k)
\end{align*}
\begin{proof}
\begin{align*}
\pi_X(k) 	& = E[\max(N-k),0)] \\
	& = \sum_{j=k}^\infty (j-k) P(N = j) \\
	& = (k-k) P(N = k) + ((k+1)-k) P(N = k+1) + P((k+2)-k) P(N = k+2) + ... \\
	& = P(N = k+1) + 2 P(N = k+2) + 3 P(N = k+3) + ... \\
	& = \underbrace{(P(N = k+1) + P(N = k+2) + P(N = k+3) + ...)}_{S(k)} \\
	& + \underbrace{(P(N = k+2) + P(N = k+3) + P(N = k+4) + ...)}_{S(k+1)} \\
	& + \underbrace{(P(N = k+3) + (PN = k+4) + P(N = k+5) + ...)}_{S(k+2)} \\
	& + ... \\
	& = \sum_{i = k}^\infty S(i)
\end{align*}
\end{proof}

\section{TVaR}
\subsection{Les 3 formes explicites de la $TVaR$	}
\label{sec:preuve}

Pour la $TVaR$, il y a 3 preuves à bien connaître : 
\begin{equation*}
TVaR_\kappa(X) = \frac{1}{1 - \kappa} \pi_X(VaR_\kappa(X)) + VaR_\kappa(X)
\end{equation*}


\begin{proof}
\label{preuve:tvar_stoploss}
\begin{align*}
TvaR_\kappa(X)  & = \frac{1}{1 - \kappa} \int_\kappa^1 VaR_u(X) du \\
    & = \frac{1}{1 - \kappa} \int_\kappa^1 (VaR_u(X) - VaR_\kappa(X) + VaR_\kappa(X)) du \\
    & = \frac{1}{1 - \kappa} \int_\kappa^1 (\underbrace{VaR_u(X)}_{\substack{\text{fonction} \\ \text{quantile}}} - VaR_\kappa(X)) du + \underbrace{\int_\kappa^1 VaR_\kappa(X) du}_{\text{intégration d'une constante}} \\
    & = \frac{1}{1 - \kappa} \int_\kappa^1 (F_X^{-1}(u) - VaR_\kappa(X)) \underbrace{f_U(u)}_{U \backsim Unif(0,1)} du + \frac{1}{\cancel{1 - \kappa}} VaR_\kappa(X) (\cancel{1 - \kappa}) \\
    & = \frac{1}{1 - \kappa} E[\max(\underbrace{F_X^{-1}(U)}_{F_X^{-1} \backsim X} - VaR_\kappa(X);0)] + VaR_\kappa(X) \\
    & = \frac{1}{1 - \kappa} E[\max(X - VaR_\kappa(X) ; 0)] + VaR_\kappa(X) \\
    & = \frac{1}{1 - \kappa} \pi_X(VaR_\kappa(X)) + VaR_\kappa(X) \\
\end{align*}
\end{proof}

à partir de la preuve ci-dessus, on peut démontrer celle-ci : 
$$
TVaR_\kappa(X) = \frac{E[X \times 1_{\{X > VaR_\kappa(X) \}}] + VaR_\kappa(X)(F_X(VaR_\kappa(X)) - \kappa)}{1-\kappa} 
$$

\begin{proof}
\begin{align*}
TVaR_\kappa(X)  & = \frac{1}{1 - \kappa} \pi_X(VaR_\kappa(X)) + VaR_\kappa(X) \\
    & = \frac{1}{1 - \kappa} E[\max(X - VaR_\kappa(X); 0)] + VaR_\kappa(X) \\
    & = \frac{1}{1 - \kappa} E[(X - VaR_\kappa(X)) \times 1_{\{X > VaR_\kappa(X) \}}] + VaR_\kappa(X) \\
    & = \frac{1}{1 - \kappa} E[X \times 1_{\{X > VaR_\kappa(X) \}}] - \frac{1}{1 - \kappa} E[VaR_\kappa(X) \times \underbrace{1_{\{X > VaR_\kappa(X) \}}}_{= S_X(VaR_\kappa(X))}] + VaR_\kappa(X) \\
    & = \frac{1}{1 - \kappa} E[X \times 1_{\{X > VaR_\kappa(X) \}}] - \frac{1}{1 - \kappa} VaR_\kappa(X)(1 - F_X(VaR_\kappa(X))) + \frac{1 - \kappa}{1 - \kappa}VaR_\kappa(X) \\
    & = \frac{E[X \times 1_{\{X > VaR_\kappa(X) \}}] + VaR_\kappa(X)(-1 + F_X(VaR_\kappa(X)) + 1 - \kappa)}{1 - \kappa}  \\
    & = \frac{E[X \times 1_{\{X > VaR_\kappa(X) \}}] + VaR_\kappa(X)(F_X(VaR_\kappa(X)) - \kappa)}{1 - \kappa}  \\
\end{align*}
\end{proof}

Une dernière preuve fortement utilisée pour la $TVaR$, qui découle directement de la dernière :  : 
\begin{align*}
TVaR_\kappa(X) = \frac{E[X \times 1_{\{X > VaR_\kappa(X) \}}]}{1 - \kappa}  \\
\end{align*}


\begin{proof}
Étant donné que cette formule ne fonctionne seulement que pour une v.a. continue, elle est très facile à prouver : 
\begin{align*}
\text{si $X$ est continue}, \quad \forall x, F_X(VaR_\kappa(X))) = \kappa \\
\end{align*}
Alors, on peut enlever la partie de droite de l'équation.
\end{proof}

\section{Sous-additivité de la $TVaR$}
\label{preuve:subadditivity_tvar}
Il y a \blue{\href{https://people.math.ethz.ch/~embrecht/ftp/Seven_Proofs.pdf}{plusieurs façons}} de prouver la sous-additivité de la $TVaR$.

\subsection{À l'aide de la fonction convexe $\varphi(x)$}


On sait que la fonction $\varphi(x)$ est convexe : 
\begin{align*}
\varphi(x) = x + \frac{1}{1 - \kappa} \pi_X(x)
\end{align*}

Et on sait aussi que 
\begin{align*}
TVaR_\kappa(X) = \inf \big \{ \varphi(x) \big \}
\end{align*}

Il faut prouver que $TVaR_\kappa(X + Y) \le TVaR_\kappa(X) + TVaR_\kappa(Y)$
\begin{proof}
Puisque $\varphi(x)$ est une fonction convexe, on peut dire que
\begin{align*}
TVaR_\kappa(X) & \le \varphi(x) \\
	& \le x + \frac{1}{1 - \kappa} \pi_X(x) \\ 
\end{align*} 
On pose le changement de variable \fbox{$X^* = \alpha X + (1-\alpha ) Y$}
\p
On peut donc remplacer $x$ dans $\varphi(x)$ par
\begin{align*}
x_0 & = VaR_\kappa(X^*) \\
	& = VaR_\kappa(\alpha X + (1-\alpha ) Y) \\
	& = \alpha VaR_\kappa(X) + (1-\alpha) VaR_\kappa(Y) \\
\end{align*}
Alors,

\begin{align*}
TVaR_\kappa(\alpha X + (1-\alpha ) Y) & \le \alpha VaR_\kappa(X) + (1-\alpha) VaR_\kappa(Y) \\
	& + \frac{1}{1 - \kappa} E [ \max( \blue{\alpha} X + \red{(1-\alpha)} Y - \blue{\alpha} VaR_\kappa(X) - \red{(1-\alpha)} VaR_\kappa(Y) ; 0)] \\ 
	& = \alpha VaR_\kappa(X) + (1-\alpha) VaR_\kappa(Y) \\
	& + \frac{1}{1 - \kappa} E[ \max( \blue{\alpha} (X - VaR_\kappa(X)) + \red{(1-\alpha )} (Y - VaR_\kappa(Y)) ; 0)] \\
	& \le \alpha VaR_\kappa(X) + (1-\alpha) VaR_\kappa(Y) \\
	& + \alpha \left( \frac{1}{1 - \kappa} E[\max( X - VaR_\kappa(X);0)] \right) \\
	& + (1 - \alpha) \left( \frac{1}{1 - \kappa} E[\max( X - VaR_\kappa(X);0)] \right) \\
	& \text{Si on met en commun, on retrouve les expressions de la $TVaR$} \\
	& = \alpha \left( \frac{1}{1 - \kappa} \pi_X(VaR_\kappa(X)) + VaR_\kappa(X) \right) \\
	& + (1 - \alpha) \left( \frac{1}{1 - \kappa} \pi_Y(VaR_\kappa(Y)) + VaR_\kappa(Y) \right) \\
TVaR_\kappa(\alpha X + (1- \alpha ) Y) & \le \alpha TVaR_\kappa(X) + (1- \alpha) TVaR_\kappa(Y) \\
\end{align*}
La relation se vérifie très bien avec le cas où $\alpha = 0,5$ : 
\begin{align*}
TVaR_\kappa(0,5X + (1-0,5)Y) & \le 0,5 TVaR_\kappa(X) + (1-0,5) TVaR_\kappa(Y) \\
\red{0,5} TVaR_\kappa(X + Y) & \le \red{0,5} TVaR_\kappa(X) + \red{0,5} TVaR_\kappa(Y) \\
	& \text{on multiplie par 2 pour enlever les \red{0,5}} \\
\mathbf{TVaR_\kappa(X + Y)} & \mathbf{\le TVaR_\kappa(X) + TVaR_\kappa(Y)} \\
\end{align*}

\end{proof}



\subsection{Avec les fonctions indicatrices}
Si on a les v.a. continues $X$ et $Y$ (les espérances existent) avec les fonctions de répartition respectives $F_X$ et $F_Y$, alors
\begin{align*}
TVaR_\kappa(X) & = \frac{E[ X \times 1_{\{X > VaR_\kappa(X) \}} ]}{1 - \kappa}  \\
(1 - \kappa) TVaR_\kappa(X) & = E[ X \times 1_{\{X > VaR_\kappa(X) \}} ]
\end{align*}
est valide pour toute v.a. continue $X$.
\p
On veut alors démontrer que
\begin{equation}
\label{eq:subaddit_tvar}
TVaR_\kappa(X) + TVaR_\kappa(Y) - TVaR_\kappa(X + Y) \ge 0
\end{equation}
\begin{proof}
.
\begin{enumerate}[label=(\arabic*)]
\item On peut écrire le membre de gauche de l'inégalité \eqref{eq:subaddit_tvar} comme
\begin{align*}
& \underbrace{(1 - \kappa)TVaR_\kappa(X)}_{E[ X \times 1_{\{X > VaR_\kappa(X) \}} ]} + \underbrace{(1 - \kappa)TVaR_\kappa(Y)}_{E[ Y \times 1_{\{Y > VaR_\kappa(Y) \}} ]} - \underbrace{(1 - \kappa)TVaR_\kappa(X + Y)}_{E[ (X+Y) \times 1_{\{X+Y > VaR_\kappa(X+Y) \}} ]} \\
& = E[X \times 1_{\{X > VaR_\kappa(X) \}}] + E[Y \times 1_{\{Y > VaR_\kappa(Y) \}}] - \underbrace{E[ (X+Y) \times 1_{\{X + Y > VaR_\kappa(X + Y) \}} ]}_{\text{On \textit{split} cette espérance}} \\
& = \underbrace{E[X \times 1_{\{X > VaR_\kappa(X) \}}] - E[X \times 1_{\{X+Y > VaR_\kappa(X+Y) \}}]}_{\text{On peut rassembler les indicatrices}} \\
& + \underbrace{E[Y \times 1_{\{Y > VaR_\kappa(Y) \}}] - E[Y \times 1_{\{X+Y > VaR_\kappa(X+Y) \}}]}_{\text{ici aussi}} \\
\end{align*}
\begin{equation}
\label{eq:espTronq_VaR}
= E[X \times (1_{\{X > VaR_\kappa(X) \}} - 1_{\{X+Y > VaR_\kappa(X+Y) \}})] + E[Y \times (1_{\{Y > VaR_\kappa(Y) \}} - 1_{\{X+Y > VaR_\kappa(X+Y) \}})]
\end{equation}
\item Rendu ici, on veut prouver que chacun de ces espérance $\ge 0$, pour que la somme des 2 soit $\ge 0$ aussi. Étant donné que les 2 parties du membre de gauche sont identiques, on va le prouver seulement pour un côté.
\p
\begin{enumerate}[label=(2.\arabic*)]
	\item Pour nous aider, on va créer un terme \textit{auxiliaire}, i.e un terme qui est égal à zéro, mais qui va nous aider à faire la preuve, soit le terme suivant : 
\begin{align*}
& E[VaR_\kappa(X) \times (1_{\{X > VaR_\kappa(X) \}} - 1_{\{X+Y > VaR_\kappa(X+Y) \}})] \\
& = VaR_\kappa(X) E[(1_{\{X > VaR_\kappa(X) \}} - 1_{\{X+Y > VaR_\kappa(X+Y) \}})] \\
& = VaR_\kappa(X) \big( (1-\kappa) - (1-\kappa)  \big) \\
& = 0 \\
\end{align*} 
\item Alors, l'équation \eqref{eq:espTronq_VaR} devient
\begin{align*}
E[(X - VaR_\kappa(X)) \times (1_{\{X > VaR_\kappa(X) \}} - 1_{\{X+Y > VaR_\kappa(X+Y) \}})]
\end{align*}
\item On va prouver que la quantité à l'intérieur de l'espérance ci-haut sera toujours $\ge 0$, de sorte que l'espérance sera toujours positive aussi : 
\begin{align*}
(X - VaR_\kappa(X))(1_{\{X > VaR_\kappa(X) \}} - 1_{\{X+Y > VaR_\kappa(X+Y) \}}) \ge 0 & \text{ si } & X < VaR_\kappa(X) \\
(X - VaR_\kappa(X))(1_{\{X > VaR_\kappa(X) \}} - 1_{\{X+Y > VaR_\kappa(X+Y) \}}) = 0 & \text{ si } & X = VaR_\kappa(X) \\
(X - VaR_\kappa(X))(1_{\{X > VaR_\kappa(X) \}} - 1_{\{X+Y > VaR_\kappa(X+Y) \}}) \ge 0 & \text{ si } & X > VaR_\kappa(X) \\
\end{align*}
\item Alors, on déduit que 
\begin{align*}
& E[(X - VaR_\kappa(X)) \times (1_{\{X > VaR_\kappa(X) \}} - 1_{\{X+Y > VaR_\kappa(X+Y) \}})] \\
& = E[X \times (1_{\{X > VaR_\kappa(X) \}} - 1_{\{X+Y > VaR_\kappa(X+Y) \}})] \ge 0
\end{align*}
\end{enumerate} % partie 2 de la preuve
\item Par conséquent,
\begin{align*}
(1 - \kappa)TVaR_\kappa(X) + (1 - \kappa)TVaR_\kappa(Y)- (1 - \kappa)TVaR_\kappa(X + Y) \ge 0 \\
\mathbf{TVaR_\kappa(X) + TVaR_\kappa(Y)- TVaR_\kappa(X + Y) \ge 0} \\
\end{align*}
\end{enumerate} % fin de l'énumération de la preuve
\end{proof}



\subsection{À l'aide des statistiques d'ordre}
Pour pouvoir prouver la sous-additivité de la $TVaR$, on peut aussi utiliser la relation des statistiques d'ordre\footnote{Relation qui d'ailleurs est utilisée dans le contexte de simulation Monte Carlo pour estimer la TVaR d'une variable aléatoire.} : 
\begin{align*}
TVaR_\kappa(X)	& = \lim_{n \to \infty} \frac{\sum_{j= [ n \kappa ] + 1}^{n} X_{j:n}}{[n (1-\kappa)]}
\end{align*}
\hl{à compléter plus tard}





\section{Loi des grands nombres}
\label{preuve:grand_nombre}
Cette preuve était demandée à l'examen Intra traditionnel H2017 du cours ACT-2001.

\begin{tcolorbox}[title=Théorème, colframe=darkgray, colback=white]
Soit les v.a. \textit{iid} $X_1, ..., X_n$ avec $E[X^m] < \infty \quad, m = 1,2, ...$ et $Var(X) < \infty$. Alors,
\begin{equation}
\lim_{n \to \infty} F_{W_n}(x) \longrightarrow F_Z(x)
\end{equation}
où $Z$ est une v.a. tel que $P(Z = E[X]) = 1$.
\end{tcolorbox}

\begin{proof}

\begin{enumerate}[label=(\arabic*)]
\item Première étape, on va démontrer que $\lim_{n \to \infty} \laplace_{w_n}(t) \rightarrow \laplace_Z(t)$
\begin{enumerate}[label=(1.\arabic*)]
\item On sait que $\laplace_{w_n}(t) = \laplace_X \left( \frac{t}{n} \right)^n \quad n = 1,2,...$.
\item Soit une v.a. $Y$ positive. On fixe $t$ tout petit
\item Alors
\begin{align*}
\laplace_Y(t) & = E[e^{-tY}] \\
	& \approx E[1 - tY] \quad \text{par dév. de Taylor}
	& = E[1] - t E[Y] \\
\end{align*}
\item 
\begin{align*}
\laplace_{w_n}(t) & = \laplace_X \left( \frac{t}{n} \right)^n  \\
	& \simeq \left(1 - \frac{t}{n} E[X] \right)^n \\
\end{align*}
\item On prends la limite de part et d'autre de l'égalité en (1.3)
\begin{align*}
\lim_{n \to \infty} \laplace_{w_n}(t) & = \lim_{n \to \infty} \left( \laplace_X \left( \frac{t}{n} \right) \right)^n \\
	& \simeq \lim_{n \to \infty} \left(1 - \frac{t}{n} E[X] \right)^n \\
	& = e^{-t E[X]} \\
	& = \laplace_Z(t) \\
\end{align*}
Ce qui correspond à la Transformée de la v.a. $Z$ où $P(Z = E[X]) = 1$
\end{enumerate}
\item On applique le résultat de (1.4)
\begin{align*}
\lim_{n \to \infty} F_{w_n} (x) = F_Z(x), \quad \forall x
\end{align*}
\end{enumerate}
\end{proof}



\section{Somme de v.a. indépendantes d'une loi Poisson Composée}
\label{preuve:poissoncompose}
\begin{proof}
Soit les v.a. indépendantes $X_1, ..., X_n$ où

$X_i \sim PoisComp(\lambda_i ; F_{B_i}) \quad, i = 1, ..., n$


Ainsi,
\begin{align*}
\laplace_{X_1}(t)	& = \fgp_{M_i} \left( \laplace_{B_i}(t) \right) \\
	& = e^{\lambda \left( \laplace_{B_i}(t) - 1 \right)} \quad, i = 1,2, ... , n \\
\end{align*}
On peut trouver la transformée de $S$,
\begin{equation}
\label{eq:laplaceS}
\laplace_S(t) = \prod_{i=1}^n \laplace_{X_i}(t) = \prod_{i=1}^{n} e^{\lambda \left( \laplace_{B_i}(t) - 1 \right)}
\end{equation} 
Le passage de l'équation \eqref{eq:laplaceS} aux étapes suivantes résulte d'une propriété de la loi de Poisson, i.e.
\begin{align*}
\laplace_S(t)	& = e^{\sum_{i=1}^{n} \lambda_i \left( \laplace_{B_i}(t) - 1 \right)} \\
	& = e^{\sum_{i=1}^{n} \lambda_i \laplace_{B_i}(t) - \lambda_i} \\
	& = e^{\sum_{i=1}^{n}\lambda_i \laplace_{B_i}(t) - \sum_{i=1}^{n} \lambda_i} \\
		& = e^{\sum_{i=1}^{n}\lambda_i \laplace_{B_i}(t) - \lambda_S} \\
\end{align*}
Si on met en évidence le $\lambda_S$ ...
\begin{align}
\label{eq:lambdaevidence}
\laplace_S(t)	& = e^{\lambda_S \left( \sum_{i=1}^{n} \frac{\lambda_i}{\lambda_S} \laplace_{B_i}(t) - 1 \right)}
\end{align}
Si on pose $c_i = \frac{\lambda_i}{\lambda_S}$, on observe que $0 < c_i < 1$ et que $\sum_{i=1}^{n} c_i = 1$.
\p
On se définit une nouvelle v.a., $D$, où
\begin{align}
\label{eq:variablealeaD}
\laplace_D(t) = \sum_{i=1}^{n} c_i \laplace_{B_i}(t)
\end{align}

Ce qui implique que $D$ obéit à une loi mélange : 
\begin{align*}
F_D(x) = \sum_{i=1}^{n} c_1 F_{B_i}(x) \quad, x \ge 0
\end{align*}

en combinant \eqref{eq:lambdaevidence} et \eqref{eq:variablealeaD}, on obtient
\begin{align}
\label{eq:laplaceD}
\laplace_S(t)	& = e^{ \lambda_S \left( \laplace_D(t) - 1 \right)}
\end{align}
On introduit une nouvelle v.a., $N_S \sim Pois(\lambda_S)$ et $P_N(s) = e^{\lambda_S(s-1)}$. Alors, \eqref{eq:laplaceD} devient
\begin{align*}
\laplace_S(t) = \fgp_{N_S}(\laplace_D(t))
\end{align*}
On peut donc représenter $S$ comme
\begin{align*}
S = 
\begin{cases}
\sum_{k=1}^{N_s} D_k & N_s > 0 \\
0	& N_s = 0 \\
\end{cases}
\end{align*}
où $D_k, \quad k = 1,2,...$ forme une suite de v.a \textit{iid}, et $D$ et $N_s$ sont indépendants.
\end{proof}


\section{Théorème d'Euler}
\label{preuve:euler}
\paragraph{Définition}
Soit une fonction $\phi : \reels^n \rightarrow \reels$ homogène d'ordre $n$. Alors, pour toute fonction $\phi$ dérivable partout, on a
\begin{align*}
n \phi(x_1, ..., x_n)  = \sum_{i=1}^{n} x_i \frac{\partial \phi(x_1, .., x_n)}{\partial x_i}
\end{align*}
\begin{proof}.
\begin{enumerate}[label=(\arabic*)]
\item Puisque $\phi$ est homogène d'ordre $n$, on a
\begin{equation}
\label{eq:homogeneite}
\phi(x_1, ..., x_n) = \lambda^n \phi(x_1, ..., x_n)
\end{equation}

\item On dérive le terme de gauche de l'équation dans l'équation \eqref{eq:homogeneite} par rapport à $\lambda$ et on pose $\lambda=1$.
\begin{align*}
\frac{\partial}{\partial \lambda} \lambda^n \phi(x_1, ..., x_n) \eval_{\lambda=1} & = n \lambda^{n-1} \phi(x_1, ..., x_n) \eval_{\lambda = 1} \\
	& = n \phi(x_1, ..., x_n) \\
\end{align*}

\item On dérive le terme de droite de l'équation dans l'équation \eqref{eq:homogeneite} par rapport à $\lambda$ et on pose $\lambda=1$.
\begin{align*}
\frac{\partial}{\partial \lambda} \lambda^n \phi(x_1, ..., x_n) \eval_{\lambda=1} & = \sum_{i=1}^{n} \frac{\partial \phi(x_1, ..., x_n)}{\partial (\lambda x_i)} \frac{\partial (\lambda x_i) }{\partial \lambda} \eval_{\lambda=1} \\ 
	& = \sum_{i=1}^{n} \frac{\partial ( \lambda x_i, ..., \lambda x_n)}{\partial \lambda x_i} x_i \eval_{\lambda=1} \\	
	& = \sum_{i=1}^{n} x_i \frac{\partial (x_1, ..., x_n)}{\partial x_i} \\
\end{align*}
\item On pose $(4) = (3)$, et on obtient le résultat souhaité.
\end{enumerate}
\end{proof}


\section{Dérivée de l'écart-type (générale)}
\label{preuve:derivee_sd}
Lorsqu'on prouve la contribution $C(X_i)$ pour $\rho(X) = \sqrt{Var(\sum_{i=1}^{n} X_i)}$, on doit dériver l'écart-type... voici le développement complet, avec un exemple où $n=3$. Ce qui est important de suivre, c'est qu'on cherche ici la contribution de la v.a. $X_i$ : alors, lorsqu'on dérive par rapport à $\lambda_i$, ça peut être n'importe quoi le $i$ : $1, 2, ..., n$. 

\paragraph{Rappel d'ACT-1002} Pour les propriétés de la covariance, voir la sous-section \ref{subsec:propriete_covariance}.
\begin{align*}
\frac{\partial}{\partial \lambda_i} \sqrt{Var \left( \sum_{i=1}^{n} \lambda_i X_i \right)} & = \frac{1}{2} \left( \frac{1}{\sqrt{Var \left( \sum_{i=1}^{n} \lambda_i X_i \right)}}   \right) \times  \\
	& \red{\frac{\partial}{\partial \lambda_i} \left( \sum_{i=1}^{n} \lambda_i^2 Var(X_i) + \sum_{i=1}^{n} \sum_{k=1, k \neq i}^{n} \lambda_i \lambda_k Cov(X_i, X_k) \right)} \\
\end{align*}
\begin{tcolorbox}[title=Explication de la forme générale de la variance, colframe=red, colback=white]
Avec un exemple $n=3$, il est très facile de comprendre d'où vient la formule générale de la variance (qui est universelle si les $X_i$ sont indépendants ou non).
\tcblower
\begin{align*}
Var(X_1 + X_2 + X_3)	& = Cov(X_1 + X_2 + X_3, X_1 + X_2 + X_3) \\
	& = \blue{Cov(X_1, X_1)} + \darkgreen{Cov(X_1, X_2)} + \purple{Cov(X_1, X_3)} \\
	& + \orange{Cov(X_2, X_1)} + \blue{Cov(X_2, X_2)} + \purple{Cov(X_2, X_3)} \\
	& + \orange{Cov(X_3, X_1)} + \darkgreen{Cov(X_3, X_2)} + \blue{Cov(X_3, X_3)} \\
\end{align*}
On remarque \blue{en bleu} les variances séparées pour chacun de nos $X_i$ de notre exemple, qu'on va pouvoir rassembler ensemble dans une même somme. On remarque aussi que les covariances sont similaires. On remarque en \orange{orange} les covariances reliées à $X_1$, en \darkgreen{vert} les covariances reliées à $X_2$ et finalement en \purple{violet} les covariances qui sont reliées à $X_3$.
\p
En étant attentif, on remarque qu'on peut sommer ensemble chaque \textit{paquet} de covariance sur tout le support ($n=3$), sauf la combinaison $Cov(X_i, X_i)$, car celle-ci a été prise pour rassembler les variances ensemble ($Var(X_i) = Cov(X_i, X_i)$).
\p
Alors, on obtient (pour le cas $n=3$) : 
\begin{align*}
Var \left( \sum_{i=1}^{3} X_i \right)	& = \sum_{i=1}^{3} Var(X_i) + \sum_{i=1}^{3} \sum_{k=1, k \neq i}^{3} Cov(X_i, X_k)
\end{align*}
\end{tcolorbox}
Si on développe le \red{la dérivée en rouge} seule du reste (en prenant l'exemple du cas $n-3$ et qu'on dérive par rapport à $\lambda_1$), on obtient : 
\begin{align*}
\frac{\partial}{\partial \lambda_1} \rho(X_1 + X_2 + X_3) & = \frac{\partial}{\partial \lambda_1} \Big[ \lambda_1^2 Var(X_1) + \lambda_2^2 Var(X_2) + \lambda_3^2 Var(X_3) \\
	& + \lambda_1 \lambda_2 Cov(X_1, X_2) + \lambda_1 \lambda_3 Cov(X_1, X_3) \\
	& + \lambda_2 \lambda_1 Cov(X_2, X_1) + \lambda_2 \lambda_3 Cov(X_2, X_3) \\
	& + \lambda_3 \lambda_1 Cov(X_3, X_1) + \lambda_3 \lambda_2 Cov(X_3, X_2) \Big] \\
	& = 2 \lambda_1 Var(X_1) \\
	& + \lambda_2 Cov(X_1, X_2) + \lambda_3 Cov(X_1, X_3) \\	
	& + \lambda_2 Cov(X_2, X_1) + \lambda_3 Cov(X_3, X_1) \\
	& = 2 \lambda_1 Var(X_1) + \sum_{k=1, k \neq 1}^{3} \lambda_k Cov(X_1, X_k) + \sum_{k=1, k \neq 1}^{3} \lambda_k Cov(X_k, X_1) \\
	& = 2 \lambda_1 Var(X_1) + 2 \sum_{k=1, k \neq 1}^{3} \lambda_k Cov(X_1, X_k) \\
\end{align*}
Il ne reste plus qu'à remettre toute l'équation ensemble : 
\begin{align*}
\frac{\partial}{\partial \lambda_i} & = \frac{1}{\cancel{2}} \frac{\cancel{2} \lambda_i Var(X_i) + \cancel{2} \sum_{k=1, k \neq i}^{3} \lambda_k Cov(X_i, X_k)}{\sqrt{Var \left( \sum_{i=1}^{n} \lambda_i X_i \right)}} \\
\end{align*}
Si on pose $\lambda_1 = ... = \lambda_i = ... = \lambda_n = 1$ et qu'on utilise les définitions des covariances pour rentrer les sommes dans la covariance, tel que

\begin{align*}
Var(X_i) + \sum_{k=1, k \neq i}^{n} Cov(X_i, X_k) & = \sum_{k=1}^{n} Cov(X_i, X_k) \\
	& = Cov \left( X_i, \sum_{k=1}^{n} X_k \right) \\
\end{align*}
Alors,
\begin{align*}
C(X_i)	& = \frac{Cov \left( X_i, \sum_{k=1}^{n} X_k   \right)}{\sqrt{Var \left(\sum_{k=1}^{n} X_k \right)}} = \frac{Cov( X_i, S)}{\sqrt{Var (S)}}
\end{align*}


\section{Distribution limite de $W_n$}
\label{preuve:dist_limiteWn}
\begin{tcolorbox}
Soit la v.a. $Z$ où $\Pr(Z = E[X]) = 1$. On veut démontrer (à l'aide des transformées de Laplace) que
\begin{align*}
F_{W_n}(x) \longleftarrow F_{Z}(x) \quad x > 0
\end{align*}
où $\Pr(Z = \gamma_j) = \Pr(\Theta = \theta_j)$ et $\gamma_j = E[X | \Theta = \theta_j]$.
\end{tcolorbox}

\begin{proof}
Pour faire la preuve, il faut savoir les 2 résultats suivants : 
\begin{equation}
\label{eq:dev_exponentielle}
e^{-x} \approx 1 - x
\end{equation}
\begin{equation}
\label{eq:dev_taylor}
\lim_{n \to \infty} \left( 1 + \frac{x}{n} \right)^n = e^x
\end{equation}
Si on développe la transformée de Laplace : 
\begin{align*}
\laplace_{W_n}(t)	& = E[e^{-t W_n}] \\
	& = E_\Theta [ E[ e^{-t W_n} | \Theta = \theta]] \\
	& = \int_{0}^\infty E[e^{-t W_n} | \Theta = \theta] f_\Theta(\theta) d\theta \\
	& = \int_{0}^{\infty} E[e^{\frac{t}{n} (X_1 + ... + X_n)} | \Theta = \theta] f_\Theta(\theta) d \theta \\
	& = \int_{0}^{\infty} \prod_{i=1}^{n} E[e^{-\frac{t}{n} X_i} | \Theta = \theta] f_\Theta (\theta) d \theta \quad \text{(Car les risques sont cond. indép.)}\\
	& = \int_{0}^{\infty} E[ \blue{e^{- \frac{t}{n} X}} | \Theta = \theta]^n f_\Theta (\theta) d \theta \quad \text{(car les v.a. sont \textit{id})} \\
	& \approx \int_{0}^{\infty} E \left[ \blue{\left(1 - \frac{t}{n} X \right)} | \Theta = \theta \right]^n f_\Theta (\theta) d \theta  \quad \text{(par l'équation \eqref{eq:dev_exponentielle})} \\
	& = \int_{0}^{\infty} \left( E[1 | \Theta] - \frac{t}{n} E[X|\Theta] \right)^n f_\Theta (\theta) d \theta \\
	& = \int_{0}^{\infty} \left( 1 - \frac{t}{n} E[X|\Theta] \right)^n f_\Theta (\theta) d \theta \\
	& \text{Si on pose la limite $n \to \infty$,} \\
\lim_{n \to \infty} \laplace_{W_n}(t)	& = \int_{0}^{\infty} \blue{\lim_{n \to \infty} \left(1 - \frac{t}{n} E[X|\Theta] \right)^n} f_\Theta (\theta) d \theta \\
	& = \int_{0}^{\infty} \blue{e^{-t E[X|\Theta]}} f_\Theta (\theta) d \theta \quad \text{(par l'équation \eqref{eq:dev_taylor})} \\
	& = \int_{0}^{\infty} e^{-t \gamma} f_\Theta (\theta) d \theta \quad \text{, où } \gamma = E[X|\Theta] \\
	& = \laplace_Z(t) \\ 
\end{align*}
\end{proof}



\end{document}