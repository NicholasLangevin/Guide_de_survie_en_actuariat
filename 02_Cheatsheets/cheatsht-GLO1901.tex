\documentclass[10pt, french]{article}
%% -----------------------------
%% Préambule
%% -----------------------------
% !TEX encoding = UTF-8 Unicode
% LaTeX Preamble for all cheatsheets
% Author : Gabriel Crépeault-Cauchon

% HOW-TO : copy-paste this file in the same directory as your .tex file, and add in your preamble the next command right after you have specified your documentclass : 
% \input{preamble-cheatsht.tex}
% ---------------------------------------------
% ---------------------------------------------

% Extra note : this preamble creates document that are meant to be used inside the multicols environment. See the documentation on internet for further information.

%% -----------------------------
%% Encoding packages
%% -----------------------------
\usepackage[utf8]{inputenc}
\usepackage[T1]{fontenc}
\usepackage{babel}
\usepackage{lmodern}

%% -----------------------------
%% Variable definition
%% -----------------------------
\def\auteur{Gabriel Crépeault-Cauchon / Nicholas Langevin}
\def\BackgroundColor{white}

%% -----------------------------
%% Margin and layout
%% -----------------------------
% Determine the margin for cheatsheet
\usepackage[landscape, hmargin=1cm, vmargin=1.7cm]{geometry}
\usepackage{multicol}

% Remove automatic indentation after section/subsection title.
\setlength{\parindent}{0cm}

% Save space in cheatsheet by removing space between align environment and normal text.
\usepackage{etoolbox}
\newcommand{\zerodisplayskips}{%
  \setlength{\abovedisplayskip}{0pt}%
  \setlength{\belowdisplayskip}{0pt}%
  \setlength{\abovedisplayshortskip}{0pt}%
  \setlength{\belowdisplayshortskip}{0pt}}
\appto{\normalsize}{\zerodisplayskips}
\appto{\small}{\zerodisplayskips}
\appto{\footnotesize}{\zerodisplayskips}

%% -----------------------------
%% URL and links
%% -----------------------------
\usepackage{hyperref}
\hypersetup{colorlinks = true, urlcolor = gray!70!white, linkcolor = black}

%% -----------------------------
%% Document policy (uncomment only one)
%% -----------------------------
%	\usepackage{concrete}
	\usepackage{mathpazo}
%	\usepackage{frcursive} %% permet d'écrire en lettres attachées
%	\usepackage{aeguill}
%	\usepackage{mathptmx}
%	\usepackage{fourier} 

%% -----------------------------
%% Math configuration
%% -----------------------------
\usepackage[fleqn]{amsmath}
\usepackage{amsthm,amssymb,latexsym,amsfonts}
\usepackage{empheq}
\usepackage{numprint}
\usepackage{dsfont} % Pour avoir le symbole du domaine Z

% Mathematics shortcuts

\newcommand{\reels}{\mathbb{R}}
\newcommand{\entiers}{\mathbb{Z}}
\newcommand{\naturels}{\mathbb{N}}
\newcommand{\eval}{\biggr \rvert}
\usepackage{cancel}
\newcommand{\derivee}[1]{\frac{\partial}{\partial #1}}
\newcommand{\prob}[1]{\Pr \left( #1 \right)}
\newcommand{\esp}[1]{\mathrm{E} \left[ #1 \right]} % espérance
\newcommand{\variance}[1]{\mathrm{Var} \left( #1   \right)}
\newcommand{\covar}[1]{\mathrm{Cov} \left( #1   \right)}
\newcommand{\laplace}{\mathcal{L}}
\newcommand{\deriv}[2][]{\frac{\partial^{#1}}{\partial #2^{#1}}}
\newcommand{\e}[1]{\mathrm{e}^{#1}}
\newcommand{\te}[1]{\text{exp}\left\{#1\right\}}
\DeclareMathSymbol{\shortminus}{\mathbin}{AMSa}{"39}



% To indicate equation number on a specific line in align environment
\newcommand\numberthis{\addtocounter{equation}{1}\tag{\theequation}}

%
% Actuarial notation packages
%
\usepackage{actuarialsymbol}
\usepackage{actuarialangle}

%
% Matrix notation for math symbols (\bm{•})
%
\usepackage{bm}
% Matrix notation variable (bold style)
\newcommand{\matr}[1]{\mathbf{#1}}



%% -----------------------------
%% tcolorbox configuration
%% -----------------------------
\usepackage[most]{tcolorbox}
\tcbuselibrary{xparse}
\tcbuselibrary{breakable}

%%
%% Coloured box "definition" for definitions
%%
\DeclareTColorBox{definition}{ o }				% #1 parameter
{
	colframe=blue!60!green,colback=blue!5!white, % color of the box
	breakable, 
	pad at break* = 0mm, 						% to split the box
	title = {#1},
	after title = {\large \hfill \faBook},
}
%%
%% Coloured box "definition2" for definitions
%%
\DeclareTColorBox{definitionNOHFILL}{ o }				% #1 parameter
{
	colframe=blue!60!green,colback=blue!5!white, % color of the box
	pad at break* = 0mm, 						% to split the box
	title = {#1},
	before title = {\faBook \quad },
	breakable
}


%%
%% Coloured box "algo" for algorithms
%%
\newtcolorbox{algo}[ 1 ]
{
	colback = blue!5!white,
	colframe = blue!75!black,
	title=#1,
	fonttitle = \bfseries,
	breakable
}
%%
%% Coloured box "conceptgen" for points adding to a concept's deifintion
%%
\newtcolorbox{conceptgen}[ 1 ]
{
	breakable,
	colback = beaublue,
	colframe = airforceblue,
	title=#1,
	fonttitle = \bfseries
}
%%
%% Coloured box "probch3" pour formules relatives au 3ème chapitre de prob
%%
\newtcolorbox{probch3}[ 1 ]
{
	colback = ruddypink,
	colframe = burgundy,
	fonttitle = \bfseries,	
	breakable,
	title=#1
}
%%
%% Coloured box "formula" for formulas
%%
\newtcolorbox{formula}[ 1 ]
{
	colback = green!5!white,
	colframe = green!70!black,
	breakable,
	fonttitle = \bfseries,
	title=#1
}
%%
%% Coloured box "formula" for formulas
%%
\DeclareTColorBox{algo2}{ o }
{
	enhanced,
	title = #1,
	colback=blue!5!white,	
	colbacktitle=blue!75!black,
	fonttitle = \bfseries,
	breakable,
	boxed title style={size=small,colframe=arsenic} ,
	attach boxed title to top center = {yshift=-3mm,yshifttext=-1mm},
}
%%
%% Coloured box "examplebox" for formulas
%%
\newtcolorbox{examplebox}[ 1 ]
{
	colback = lightmauve,
	colframe = antiquefuchsia,
	breakable,
	fonttitle = \bfseries,title=#1
}
%%
%% Coloured box "rappel" pour rappel de formules
%%
\newtcolorbox{rappel}[ 1 ]
{
	colback = ashgrey,
	colframe = arsenic,
	breakable,
	fonttitle = \bfseries,title=#1
}
%%
%% Coloured box "rappel" pour rappel de formules
%%
\DeclareTColorBox{rappel_enhanced}{ o }
{
	enhanced,
	title = #1,
	colback=ashgrey, % color of the box
%	colframe=blue(pigment),
%	colframe=arsenic,	
	colbacktitle=arsenic,
	fonttitle = \bfseries,
	breakable,
	boxed title style={size=small,colframe=arsenic} ,
	attach boxed title to top center = {yshift=-3mm,yshifttext=-1mm},
}
%%
%% Coloured box "notation" for notation and terminology
%%
\DeclareTColorBox{distributions}{ o }			% #1 parameter
{
	enhanced,
	title = #1,
	colback=gray(x11gray), % color of the box
%	colframe=blue(pigment),
	colframe=arsenic,	
	colbacktitle=aurometalsaurus,
	fonttitle = \bfseries,
	boxed title style={size=small,colframe=arsenic} ,
	attach boxed title to top center = {yshift=-3mm,yshifttext=-1mm},
	breakable
%	left=0pt,
%  	right=0pt,
%    box align=center,
%    ams align*
%  	top=-10pt
}

%% -----------------------------
%% Graphics and pictures
%% -----------------------------
\usepackage{graphicx}
\usepackage{pict2e}
\usepackage{tikz}

%% -----------------------------
%% insert pdf pages into document
%% -----------------------------
\usepackage{pdfpages}

%% -----------------------------
%% Color configuration
%% -----------------------------
\usepackage{color, soulutf8, colortbl}


%
%	Colour definitions
%
\definecolor{blue(munsell)}{rgb}{0.0, 0.5, 0.69}
\definecolor{blue(matcha)}{rgb}{0.596, 0.819, 1.00}
\definecolor{blue(munsell)-light}{rgb}{0.5, 0.8, 0.9}
\definecolor{bleudefrance}{rgb}{0.19, 0.55, 0.91}
\definecolor{blizzardblue}{rgb}{0.67, 0.9, 0.93}
\definecolor{bondiblue}{rgb}{0.0, 0.58, 0.71}
\definecolor{blue(pigment)}{rgb}{0.2, 0.2, 0.6}
\definecolor{bluebell}{rgb}{0.64, 0.64, 0.82}
\definecolor{airforceblue}{rgb}{0.36, 0.54, 0.66}
\definecolor{beaublue}{rgb}{0.74, 0.83, 0.9}
\definecolor{cobalt}{rgb}{0.0, 0.28, 0.67}	% nice light blue-ish
\definecolor{blue_rectangle}{RGB}{83, 84, 244}		% ACT-2004
\definecolor{indigo(web)}{rgb}{0.29, 0.0, 0.51}	% purple-ish
\definecolor{antiquefuchsia}{rgb}{0.57, 0.36, 0.51}	%	pastel dark purple ish
\definecolor{darkpastelpurple}{rgb}{0.59, 0.44, 0.84}
\definecolor{gray(x11gray)}{rgb}{0.75, 0.75, 0.75}
\definecolor{aurometalsaurus}{rgb}{0.43, 0.5, 0.5}
\definecolor{ruddypink}{rgb}{0.88, 0.56, 0.59}
\definecolor{pastelred}{rgb}{1.0, 0.41, 0.38}		
\definecolor{lightmauve}{rgb}{0.86, 0.82, 1.0}
\definecolor{azure(colorwheel)}{rgb}{0.0, 0.5, 1.0}
\definecolor{darkgreen}{rgb}{0.0, 0.2, 0.13}			
\definecolor{burntorange}{rgb}{0.8, 0.33, 0.0}		
\definecolor{burntsienna}{rgb}{0.91, 0.45, 0.32}		
\definecolor{ao(english)}{rgb}{0.0, 0.5, 0.0}		% ACT-2003
\definecolor{amber(sae/ece)}{rgb}{1.0, 0.49, 0.0} 	% ACT-2004
\definecolor{green_rectangle}{RGB}{131, 176, 84}		% ACT-2004
\definecolor{red_rectangle}{RGB}{241,112,113}		% ACT-2004
\definecolor{amethyst}{rgb}{0.6, 0.4, 0.8}
\definecolor{amethyst-light}{rgb}{0.6, 0.4, 0.8}
\definecolor{ashgrey}{rgb}{0.7, 0.75, 0.71}			% dark grey-black-ish
\definecolor{arsenic}{rgb}{0.23, 0.27, 0.29}			% light green-beige-ish gray
\definecolor{amaranth}{rgb}{0.9, 0.17, 0.31}
\definecolor{brickred}{rgb}{0.8, 0.25, 0.33}
\definecolor{pastelred}{rgb}{1.0, 0.41, 0.38}

%
% Useful shortcuts for coloured text
%
\newcommand{\orange}{\textcolor{orange}}
\newcommand{\red}{\textcolor{red}}
\newcommand{\cyan}{\textcolor{cyan}}
\newcommand{\blue}{\textcolor{blue}}
\newcommand{\green}{\textcolor{green}}
\newcommand{\purple}{\textcolor{magenta}}
\newcommand{\yellow}{\textcolor{yellow}}

%% -----------------------------
%% Enumerate environment configuration
%% -----------------------------
%
% Custum enumerate & itemize Package
%
\usepackage{enumitem}
%
% French Setup for itemize function
%
\frenchbsetup{StandardItemLabels=true}
%
% Change default label for itemize
%
\renewcommand{\labelitemi}{\faAngleRight}


%% -----------------------------
%% Tabular column type configuration
%% -----------------------------
\newcolumntype{C}{>{$}c<{$}} % math-mode version of "l" column type
\newcolumntype{L}{>{$}l<{$}} % math-mode version of "l" column type
\newcolumntype{R}{>{$}r<{$}} % math-mode version of "l" column type
\newcolumntype{f}{>{\columncolor{green!20!white}}p{1cm}}
\newcolumntype{g}{>{\columncolor{green!40!white}}m{1.2cm}}
\newcolumntype{a}{>{\columncolor{red!20!white}$}p{2cm}<{$}}	% ACT-2005
% configuration to force a line break within a single cell
\usepackage{makecell}


%% -----------------------------
%% Fontawesome for special symbols
%% -----------------------------
\usepackage{fontawesome}

%% -----------------------------
%% Section Font customization
%% -----------------------------
\usepackage{sectsty}
\sectionfont{\color{\SectionColor}}
\subsectionfont{\color{\SubSectionColor}}

%% -----------------------------
%% Footer/Header Customization
%% -----------------------------
\usepackage{lastpage}
\usepackage{fancyhdr}
\pagestyle{fancy}

%
% Header
%
\fancyhead{} 	% Reset
\fancyhead[L]{Aide-mémoire pour~ \cours ~(\textbf{\sigle})}
\fancyhead[R]{\auteur}

%
% Footer
%
\fancyfoot{}		% Reset
\fancyfoot[R]{\thepage ~de~ \pageref{LastPage}}
\fancyfoot[L]{\href{https://github.com/ressources-act/Guide_de_survie_en_actuariat}{\faGithub \ ressources-act/Guide de survie en actuariat}}
%
% Page background color
%
\pagecolor{\BackgroundColor}




%% END OF PREAMBLE
% ---------------------------------------------
% ---------------------------------------------
%% -----------------------------
%% Variable definition
%% -----------------------------
\def\cours{Introduction à la programmation avec Python}
\def\sigle{GLO-1901}
%% -----------------------------
%% Colour setup for sections
%% -----------------------------
\def\SectionColor{burntorange}
\def\SubSectionColor{burntsienna}
\def\SubSubSectionColor{burntsienna}
%% -----------------------------
%% Colour setup for prestations
%%	Ajoute couleurs sur les trêmas des signes de prestations
%% -----------------------------
\usepackage{stackengine}
\newcommand\cumlaut[2][black]{\stackon[.33ex]{#2}{\textcolor{#1}{\kern-.04ex.\kern-.2ex.}}}
%
% Save more space than default
%
\setlength{\abovedisplayskip}{-15pt}
\setlist{leftmargin=*}
%
%	Extra math symbols
%
\usepackage{mathrsfs}
\usetikzlibrary{matrix}
\usepackage{listings}
%
% thin space, limits underneath in displays
%

%% -----------------------------
%% 	Colour setup for sections
%% -----------------------------
\def\SectionColor{cobalt}
\def\SubSectionColor{azure(colorwheel)}
\def\SubSubSection{azure(colorwheel)}
%% -----------------------------
\setcounter{secnumdepth}{0}

%% -----------------------------
%% Color definitions
%% -----------------------------
\definecolor{indigo(web)}{rgb}{0.29, 0.0, 0.51}
\definecolor{cobalt}{rgb}{0.0, 0.28, 0.67}
\definecolor{azure(colorwheel)}{rgb}{0.0, 0.5, 1.0}
%% -----------------------------
%% Variable definition
%% -----------------------------
%%
%% Matrix notation variable (bold style)
%%
\newcommand\cololine[2]{\colorlet{temp}{.}\color{#1}\bar{\color{temp}#2}\color{temp}}
\newcommand\colbar[2]{\colorlet{temp}{.}\color{#1}\bar{\color{temp}#2}\color{temp}}

\begin{document}

\begin{center}
	\textsc{\Large Contributeurs}\\[0.5cm] 
\end{center}
%\begin{contrib}{ACT-1XXX\: Cours de première année}
\begin{description}
	\item[aut., cre.] Alec James van Rassel
\end{description}
\end{contrib}

\newpage
\raggedcolumns
\begin{multicols*}{2}
\section{Introduction}

\begin{definitionNOHFILLprop}[Étapes de développement]
\begin{enumerate}[label = \circled{\arabic*}{trueblue}]
	\item	L'analyse
		\begin{itemize}
		\item	Quel est le \textit{problème} ?
		\item	Que veut l'\textit{utilisateur} ?
		\item	Quel est son \textit{budget} ?
		\item	Quelles sont les \textbf{conséquences} d'une \textit{erreur} ?
		\end{itemize}
	\item	La conception
		\begin{itemize}
		\item	Comment résoudre le problème ?
		\item	Quelles sont les \textit{structures de données} appropriées ?
		\item	Quels sont les \textit{algorithmes} nécessaires ? 
		\item	Quelles sont les \textit{interfaces} requises ?
		\end{itemize}
	\item	La programmation
		\begin{itemize}
		\item	Implantation de la solution développée aux étapes précédente, en utilisant un ou plusieurs langages de programmation.
		\end{itemize}
	\item	Les tests d'intégration
		\begin{itemize}
		\item	L'intégration des différents modules en un tout cohérent ;
		\item	Les procédures de tests qui permettent d'établir la validité et la fiabilité du logiciel.
		\end{itemize}
\end{enumerate}
\end{definitionNOHFILLprop}


\begin{definitionNOHFILLpropos}[L'approche hiérarchique pour traiter des données]
\begin{itemize}
	\item	les \textbf{programmes} sont composés de \textbf{modules} ;        
	\item	les \textbf{modules} contiennent des \textbf{énoncés} ;            
	\item	les \textbf{énoncés} contiennent des \textbf{expressions} ;        
	\item	les \textbf{expressions} créent et manipulent les \textbf{données}.
\end{itemize}
\end{definitionNOHFILLpropos}



\subsection{Syntaxe de base}
\begin{description}
	\item[Affectation]	énoncé (e.g. \texttt{pi = 3.1415}) ayant habituellement 3 éléments :
		\begin{enumerate}[label = \rectangled{\arabic*}{lightgray}]
		\item	un nom de variable (e.g. \texttt{pi}) appelé \textbf{identifieur} ;
		\item	l'opérateur (e.g. \texttt{=}) ;
		\item	une \textbf{valeur} affectée à la variable (e.g. 3.1415) appelée \textbf{expression}.
		\end{enumerate}
	\item[Commentaires]	Des commentaires sont des lignes du code, qui commencent par \#, qui ne sont pas exécutées.
\end{description}
	
	
\subsubsection{Opérateurs arithmétiques}
\begin{itemize}[leftmargin = 5mm]
	\item[+]	addition ;
	\item[-]	soustraction ;
	\item[*]	multiplication ;
	\item[/]	division régulière ;
	\item[//]	division entière ;
	\item[\%]	reste de la division entière ;
	\item[**]	exponentiation.
\end{itemize}

Fonction \texttt{int} retourne la partie entière d'un nombre.
3 types de nombres : entiers, flottants et complexes. Lorsque l'on effectue une opération arithmétique entre certains nombres, Python conserve le type le plus général.
	de même pour \texttt{float}
	
Module de maths a plus d'opérateur (e.g. sin, cos, sqrt) et est importé avec \texttt{import math}. Puis, on utilise ses fonctions avec, p. ex., \texttt{math.sqrt()}.


\subsection{Fonctions de base}
\begin{definitionNOHFILLsub}[\texttt{print}]
Permet d'afficher à la console la valeur d'une ou de plusieurs expressions.
\begin{itemize}
	\item	permet aussi avec toute sorte d'options de spécifier la façon dont cette ou ces valeurs seront affichées.
\end{itemize}
\end{definitionNOHFILLsub}

\begin{definitionNOHFILLsub}[\texttt{input}]
Permet de lire ce que vous entrez au clavier et retourne le résultat sous la forme d'une chaîne de caractères.

\end{definitionNOHFILLsub}



\newpage
\section{Fonctions}
\subsection{Introduction}
\subsubsection{Qualités d'une fonction}
\begin{definitionGENERAL}{Cohérence}[\circled{1}{trueblue}]
Une fonction est \textbf{\textit{cohérente}} si elle accomplit une seule tâche. On doit pouvoir résumer en peu de mots ce qu'accomplit la fonction.
\end{definitionGENERAL}

\begin{definitionGENERAL}{Indépendance}[\circled{2}{trueblue}]
Une fonction est \textbf{\textit{indépendante}} si sa sortie dépend uniquement de ses entrées (arguments) et d'aucune autre variable. Il ne faut pas définir des fonctions qui dépendent de \textit{variables globales}.
\end{definitionGENERAL}

\begin{definitionGENERAL}{Concision}[\circled{3}{trueblue}]
La \textit{\textit{concision}} consiste à limiter la longueur des fonctions. Plus la fonction est courte, plus elle sera facile à comprendre pour un humain.
\end{definitionGENERAL}


\subsubsection{Définition}
De façon générale, on définit une fonction en Python avec \texttt{def} :
\begin{lstlisting}
def name(arg1, arg2, ..., argn):
	#	bloc d'énoncés indentés
	return expression	#	optionnel
\end{lstlisting}



\subsubsection{Booléens}
Les opérateurs suivants permettent de comparer les valeurs respectives de deux objets : 
\begin{description}
	\item[$<$]	inférieur ;
	\item[$>$]	supérieur ;
	\item[$<=$]	inférieur ou égal ;
	\item[$>=$]	supérieur ou égal ;
	\item[$==$]	égal ;
	\item[$!=$]	pas égal.
\end{description}

\bigskip

Les opérateurs suivants permettent de combiner une ou plusieurs expressions booléennes : 
\begin{description}
	\item[and]	conjonction ($\cap$) ;
	\item[or]	disjonction ($\cup$) ;
	\item[not]	négation ($A^{\complement}$).
\end{description}

\subsection{Énoncés conditionnels}
\subsubsection{Forme générale du \texttt{if}}
\begin{lstlisting}
if expression_1:
	#	bloc d'énoncés 1
elif expression_1:
	#	bloc d'énoncés 2
#	...
elif expression_n:
	#	bloc d'énoncés n
else:
	#	bloc d'énoncés n + 1
\end{lstlisting}

On peut aussi utiliser l'opérateur \texttt{if else} pour définir une variable. Par exemple, pour l'expression \texttt{a = y if x else z} si \texttt{x} est vraie, alors \texttt{a = y} sinon \texttt{a = z}.

\end{multicols*}

\end{document}
