\documentclass[10pt, french]{article}

%% -----------------------------
%% Préambule
%% -----------------------------
% !TEX encoding = UTF-8 Unicode
% LaTeX Preamble for all cheatsheets
% Author : Gabriel Crépeault-Cauchon

% HOW-TO : copy-paste this file in the same directory as your .tex file, and add in your preamble the next command right after you have specified your documentclass : 
% \input{preamble-cheatsht.tex}
% ---------------------------------------------
% ---------------------------------------------

% Extra note : this preamble creates document that are meant to be used inside the multicols environment. See the documentation on internet for further information.

%% -----------------------------
%% Encoding packages
%% -----------------------------
\usepackage[utf8]{inputenc}
\usepackage[T1]{fontenc}
\usepackage{babel}
\usepackage{lmodern}

%% -----------------------------
%% Variable definition
%% -----------------------------
\def\auteur{Gabriel Crépeault-Cauchon / Nicholas Langevin}
\def\BackgroundColor{white}

%% -----------------------------
%% Margin and layout
%% -----------------------------
% Determine the margin for cheatsheet
\usepackage[landscape, hmargin=1cm, vmargin=1.7cm]{geometry}
\usepackage{multicol}

% Remove automatic indentation after section/subsection title.
\setlength{\parindent}{0cm}

% Save space in cheatsheet by removing space between align environment and normal text.
\usepackage{etoolbox}
\newcommand{\zerodisplayskips}{%
  \setlength{\abovedisplayskip}{0pt}%
  \setlength{\belowdisplayskip}{0pt}%
  \setlength{\abovedisplayshortskip}{0pt}%
  \setlength{\belowdisplayshortskip}{0pt}}
\appto{\normalsize}{\zerodisplayskips}
\appto{\small}{\zerodisplayskips}
\appto{\footnotesize}{\zerodisplayskips}

%% -----------------------------
%% URL and links
%% -----------------------------
\usepackage{hyperref}
\hypersetup{colorlinks = true, urlcolor = gray!70!white, linkcolor = black}

%% -----------------------------
%% Document policy (uncomment only one)
%% -----------------------------
%	\usepackage{concrete}
	\usepackage{mathpazo}
%	\usepackage{frcursive} %% permet d'écrire en lettres attachées
%	\usepackage{aeguill}
%	\usepackage{mathptmx}
%	\usepackage{fourier} 

%% -----------------------------
%% Math configuration
%% -----------------------------
\usepackage[fleqn]{amsmath}
\usepackage{amsthm,amssymb,latexsym,amsfonts}
\usepackage{empheq}
\usepackage{numprint}
\usepackage{dsfont} % Pour avoir le symbole du domaine Z

% Mathematics shortcuts

\newcommand{\reels}{\mathbb{R}}
\newcommand{\entiers}{\mathbb{Z}}
\newcommand{\naturels}{\mathbb{N}}
\newcommand{\eval}{\biggr \rvert}
\usepackage{cancel}
\newcommand{\derivee}[1]{\frac{\partial}{\partial #1}}
\newcommand{\prob}[1]{\Pr \left( #1 \right)}
\newcommand{\esp}[1]{\mathrm{E} \left[ #1 \right]} % espérance
\newcommand{\variance}[1]{\mathrm{Var} \left( #1   \right)}
\newcommand{\covar}[1]{\mathrm{Cov} \left( #1   \right)}
\newcommand{\laplace}{\mathcal{L}}
\newcommand{\deriv}[2][]{\frac{\partial^{#1}}{\partial #2^{#1}}}
\newcommand{\e}[1]{\mathrm{e}^{#1}}
\newcommand{\te}[1]{\text{exp}\left\{#1\right\}}
\DeclareMathSymbol{\shortminus}{\mathbin}{AMSa}{"39}



% To indicate equation number on a specific line in align environment
\newcommand\numberthis{\addtocounter{equation}{1}\tag{\theequation}}

%
% Actuarial notation packages
%
\usepackage{actuarialsymbol}
\usepackage{actuarialangle}

%
% Matrix notation for math symbols (\bm{•})
%
\usepackage{bm}
% Matrix notation variable (bold style)
\newcommand{\matr}[1]{\mathbf{#1}}



%% -----------------------------
%% tcolorbox configuration
%% -----------------------------
\usepackage[most]{tcolorbox}
\tcbuselibrary{xparse}
\tcbuselibrary{breakable}

%%
%% Coloured box "definition" for definitions
%%
\DeclareTColorBox{definition}{ o }				% #1 parameter
{
	colframe=blue!60!green,colback=blue!5!white, % color of the box
	breakable, 
	pad at break* = 0mm, 						% to split the box
	title = {#1},
	after title = {\large \hfill \faBook},
}
%%
%% Coloured box "definition2" for definitions
%%
\DeclareTColorBox{definitionNOHFILL}{ o }				% #1 parameter
{
	colframe=blue!60!green,colback=blue!5!white, % color of the box
	pad at break* = 0mm, 						% to split the box
	title = {#1},
	before title = {\faBook \quad },
	breakable
}


%%
%% Coloured box "algo" for algorithms
%%
\newtcolorbox{algo}[ 1 ]
{
	colback = blue!5!white,
	colframe = blue!75!black,
	title=#1,
	fonttitle = \bfseries,
	breakable
}
%%
%% Coloured box "conceptgen" for points adding to a concept's deifintion
%%
\newtcolorbox{conceptgen}[ 1 ]
{
	breakable,
	colback = beaublue,
	colframe = airforceblue,
	title=#1,
	fonttitle = \bfseries
}
%%
%% Coloured box "probch3" pour formules relatives au 3ème chapitre de prob
%%
\newtcolorbox{probch3}[ 1 ]
{
	colback = ruddypink,
	colframe = burgundy,
	fonttitle = \bfseries,	
	breakable,
	title=#1
}
%%
%% Coloured box "formula" for formulas
%%
\newtcolorbox{formula}[ 1 ]
{
	colback = green!5!white,
	colframe = green!70!black,
	breakable,
	fonttitle = \bfseries,
	title=#1
}
%%
%% Coloured box "formula" for formulas
%%
\DeclareTColorBox{algo2}{ o }
{
	enhanced,
	title = #1,
	colback=blue!5!white,	
	colbacktitle=blue!75!black,
	fonttitle = \bfseries,
	breakable,
	boxed title style={size=small,colframe=arsenic} ,
	attach boxed title to top center = {yshift=-3mm,yshifttext=-1mm},
}
%%
%% Coloured box "examplebox" for formulas
%%
\newtcolorbox{examplebox}[ 1 ]
{
	colback = lightmauve,
	colframe = antiquefuchsia,
	breakable,
	fonttitle = \bfseries,title=#1
}
%%
%% Coloured box "rappel" pour rappel de formules
%%
\newtcolorbox{rappel}[ 1 ]
{
	colback = ashgrey,
	colframe = arsenic,
	breakable,
	fonttitle = \bfseries,title=#1
}
%%
%% Coloured box "rappel" pour rappel de formules
%%
\DeclareTColorBox{rappel_enhanced}{ o }
{
	enhanced,
	title = #1,
	colback=ashgrey, % color of the box
%	colframe=blue(pigment),
%	colframe=arsenic,	
	colbacktitle=arsenic,
	fonttitle = \bfseries,
	breakable,
	boxed title style={size=small,colframe=arsenic} ,
	attach boxed title to top center = {yshift=-3mm,yshifttext=-1mm},
}
%%
%% Coloured box "notation" for notation and terminology
%%
\DeclareTColorBox{distributions}{ o }			% #1 parameter
{
	enhanced,
	title = #1,
	colback=gray(x11gray), % color of the box
%	colframe=blue(pigment),
	colframe=arsenic,	
	colbacktitle=aurometalsaurus,
	fonttitle = \bfseries,
	boxed title style={size=small,colframe=arsenic} ,
	attach boxed title to top center = {yshift=-3mm,yshifttext=-1mm},
	breakable
%	left=0pt,
%  	right=0pt,
%    box align=center,
%    ams align*
%  	top=-10pt
}

%% -----------------------------
%% Graphics and pictures
%% -----------------------------
\usepackage{graphicx}
\usepackage{pict2e}
\usepackage{tikz}

%% -----------------------------
%% insert pdf pages into document
%% -----------------------------
\usepackage{pdfpages}

%% -----------------------------
%% Color configuration
%% -----------------------------
\usepackage{color, soulutf8, colortbl}


%
%	Colour definitions
%
\definecolor{blue(munsell)}{rgb}{0.0, 0.5, 0.69}
\definecolor{blue(matcha)}{rgb}{0.596, 0.819, 1.00}
\definecolor{blue(munsell)-light}{rgb}{0.5, 0.8, 0.9}
\definecolor{bleudefrance}{rgb}{0.19, 0.55, 0.91}
\definecolor{blizzardblue}{rgb}{0.67, 0.9, 0.93}
\definecolor{bondiblue}{rgb}{0.0, 0.58, 0.71}
\definecolor{blue(pigment)}{rgb}{0.2, 0.2, 0.6}
\definecolor{bluebell}{rgb}{0.64, 0.64, 0.82}
\definecolor{airforceblue}{rgb}{0.36, 0.54, 0.66}
\definecolor{beaublue}{rgb}{0.74, 0.83, 0.9}
\definecolor{cobalt}{rgb}{0.0, 0.28, 0.67}	% nice light blue-ish
\definecolor{blue_rectangle}{RGB}{83, 84, 244}		% ACT-2004
\definecolor{indigo(web)}{rgb}{0.29, 0.0, 0.51}	% purple-ish
\definecolor{antiquefuchsia}{rgb}{0.57, 0.36, 0.51}	%	pastel dark purple ish
\definecolor{darkpastelpurple}{rgb}{0.59, 0.44, 0.84}
\definecolor{gray(x11gray)}{rgb}{0.75, 0.75, 0.75}
\definecolor{aurometalsaurus}{rgb}{0.43, 0.5, 0.5}
\definecolor{ruddypink}{rgb}{0.88, 0.56, 0.59}
\definecolor{pastelred}{rgb}{1.0, 0.41, 0.38}		
\definecolor{lightmauve}{rgb}{0.86, 0.82, 1.0}
\definecolor{azure(colorwheel)}{rgb}{0.0, 0.5, 1.0}
\definecolor{darkgreen}{rgb}{0.0, 0.2, 0.13}			
\definecolor{burntorange}{rgb}{0.8, 0.33, 0.0}		
\definecolor{burntsienna}{rgb}{0.91, 0.45, 0.32}		
\definecolor{ao(english)}{rgb}{0.0, 0.5, 0.0}		% ACT-2003
\definecolor{amber(sae/ece)}{rgb}{1.0, 0.49, 0.0} 	% ACT-2004
\definecolor{green_rectangle}{RGB}{131, 176, 84}		% ACT-2004
\definecolor{red_rectangle}{RGB}{241,112,113}		% ACT-2004
\definecolor{amethyst}{rgb}{0.6, 0.4, 0.8}
\definecolor{amethyst-light}{rgb}{0.6, 0.4, 0.8}
\definecolor{ashgrey}{rgb}{0.7, 0.75, 0.71}			% dark grey-black-ish
\definecolor{arsenic}{rgb}{0.23, 0.27, 0.29}			% light green-beige-ish gray
\definecolor{amaranth}{rgb}{0.9, 0.17, 0.31}
\definecolor{brickred}{rgb}{0.8, 0.25, 0.33}
\definecolor{pastelred}{rgb}{1.0, 0.41, 0.38}

%
% Useful shortcuts for coloured text
%
\newcommand{\orange}{\textcolor{orange}}
\newcommand{\red}{\textcolor{red}}
\newcommand{\cyan}{\textcolor{cyan}}
\newcommand{\blue}{\textcolor{blue}}
\newcommand{\green}{\textcolor{green}}
\newcommand{\purple}{\textcolor{magenta}}
\newcommand{\yellow}{\textcolor{yellow}}

%% -----------------------------
%% Enumerate environment configuration
%% -----------------------------
%
% Custum enumerate & itemize Package
%
\usepackage{enumitem}
%
% French Setup for itemize function
%
\frenchbsetup{StandardItemLabels=true}
%
% Change default label for itemize
%
\renewcommand{\labelitemi}{\faAngleRight}


%% -----------------------------
%% Tabular column type configuration
%% -----------------------------
\newcolumntype{C}{>{$}c<{$}} % math-mode version of "l" column type
\newcolumntype{L}{>{$}l<{$}} % math-mode version of "l" column type
\newcolumntype{R}{>{$}r<{$}} % math-mode version of "l" column type
\newcolumntype{f}{>{\columncolor{green!20!white}}p{1cm}}
\newcolumntype{g}{>{\columncolor{green!40!white}}m{1.2cm}}
\newcolumntype{a}{>{\columncolor{red!20!white}$}p{2cm}<{$}}	% ACT-2005
% configuration to force a line break within a single cell
\usepackage{makecell}


%% -----------------------------
%% Fontawesome for special symbols
%% -----------------------------
\usepackage{fontawesome}

%% -----------------------------
%% Section Font customization
%% -----------------------------
\usepackage{sectsty}
\sectionfont{\color{\SectionColor}}
\subsectionfont{\color{\SubSectionColor}}

%% -----------------------------
%% Footer/Header Customization
%% -----------------------------
\usepackage{lastpage}
\usepackage{fancyhdr}
\pagestyle{fancy}

%
% Header
%
\fancyhead{} 	% Reset
\fancyhead[L]{Aide-mémoire pour~ \cours ~(\textbf{\sigle})}
\fancyhead[R]{\auteur}

%
% Footer
%
\fancyfoot{}		% Reset
\fancyfoot[R]{\thepage ~de~ \pageref{LastPage}}
\fancyfoot[L]{\href{https://github.com/ressources-act/Guide_de_survie_en_actuariat}{\faGithub \ ressources-act/Guide de survie en actuariat}}
%
% Page background color
%
\pagecolor{\BackgroundColor}




%% END OF PREAMBLE
% ---------------------------------------------
% ---------------------------------------------
%% -----------------------------
%% Variable definition
%% -----------------------------
\def\cours{Introduction à l'actuariat II}
\def\sigle{ACT-2001}
\def\SectionColor{burntorange}
\def\SubSectionColor{burntsienna}
\def\SubSubSectionColor{burntsienna}

%% Reduce margin space
\setlength{\abovedisplayskip}{-15pt}

\newcommand{\bettershortstack}[2][c]{%
  \begin{tabular}[b]{@{}#1@{}}#2\end{tabular}%
}
\usepackage{stackengine}
\newcommand\cumlaut[2][black]{\stackon[.33ex]{#2}{\textcolor{#1}{\kern-.04ex.\kern-.2ex.}}}
%% -----------------------------
%% Début du document
%% -----------------------------
\begin{document}

\begin{center}
	\textsc{\Large Contributeurs}\\[0.5cm] 
\end{center}
%\input{contributeurs/contrib-ACT2001}

\newpage

\raggedcolumns
\begin{multicols*}{2} 

\begin{distributions}[Notation]
\begin{description}
	\item[$S$]	Les coûts d'un portefeuille.
	\item[$\rho(S)$]	Une mesure de risque.
\end{description}
\end{distributions}

\section{Mesures de risque}

\begin{description}
	\item[Capital économique]	Allocation de surplus de la compagnie;
		\begin{align*}
		CE(S)	
		&=	\rho(S)	-	\esp{S}
		\end{align*}
	\item[Marge de risque]	associée à une prime $P(X)$;
		\begin{align*}
		MR(X)
		&=	\rho(X)	-	\esp{X}
		\end{align*}
\end{description}

$\rho$ introduit une marge de risque:
\begin{description}
	\item[positive]	lorsque \icbox{$\rho(X)	\geq		\esp{X}$} pour une v.a. $X$ avec \icbox[red][palechestnut]{$\esp{X} < \infty$};
	\item[justifiée]	lorsque \icbox{$\rho(X)	=	\rho(a)	=	a$} pour une v.a. $X$ avec \icbox[red][palechestnut]{$\Pr(X	=	a)	=	1, \alpha > 0$};
	\item[non-excessive]	lorsque \icbox{$\rho(X)	\leq		a_{\max}$} pour une v.a. $X$ \icbox[red][palechestnut]{s'il existe $a_{\max}	<	\infty$} \icbox[red][palechestnut]{tel que $\Pr(X	\leq		a_{\max})	=	1$};
\end{description}

\subsection{Propriétés désirables d'une mesure de risque}
\begin{definitionNOHFILLsub}[Homogénéité]
Soit une v.a. $X$ et un scalaire \icbox[red][palechestnut]{$c	>	0$}, la mesure de risque $\rho$ est dite homogène si \icbox{$\rho(cX)	=	c\rho(X)$}.
\end{definitionNOHFILLsub}

\begin{definitionNOHFILLsub}[Invariance à la translation]
Soit une v.a. $X$ et un scalaire \icbox[red][palechestnut]{$c	\in	\mathbb{R}$}, la mesure de risque $\rho$ satisfait la propriété d'invariance à la translation si \icbox{$\rho(X + c)	=	\rho(X) + c$}.	\\

Ajouter un montant positif à un risque ajoute un montant équivalent à la mesure de risque.
\end{definitionNOHFILLsub}

\begin{definitionNOHFILLsub}[Monotonicité]
Soit les v.a. $X_{1}$ et $X_{2}$ \icbox[red][palechestnut]{tel que $\Pr(X	\leq	X_{2})	=	1$}, la mesure de risque $\rho$ satisfait la propriété de monotonicité si \icbox{$\rho(X_{1})	\leq	\rho(X_{2})$} ou si \icbox[red][palechestnut]{$\forall	u	\in	(0, 1)$}, \icbox{$F_{X_{1}}^{-1}(u)	\leq	F_{X_{2}}^{-1}(u)$}.

\end{definitionNOHFILLsub}

\begin{definitionNOHFILLsub}[Sous-additivité]
Soit les v.a. $X_{1}$ et $X_{2}$, la mesure de risque $\rho$ satisfait la propriété de sous-additivité si \icbox{$\rho(X_{1}	+	X_{2})	\leq	\rho(X_{1})	+	\rho(X_{2})$}.

\end{definitionNOHFILLsub}

\begin{definitionNOHFILLsub}[Convexité]
Soit les v.a. $X_{1}$ et $X_{2}$, la mesure de risque $\rho$ satisfait la propriété de convexité si \icbox{$\rho(\alpha X_{1}	+	(1	-	\alpha)X_{2})	\leq	\alpha\rho(X_{1})	+	(1	-	\alpha)\rho(X_{2})$}.

\end{definitionNOHFILLsub}

\subsection{TVaR et VaR}

\begin{itemize}
	\item	La \textbf{Value-at-Risk} correspond au $100\alpha^{\text{e}}$ pourcentile;
	\item	Si $X$ représente les gains, on s'intéresse à l'extrémité inférieure de la distribution des gains et \icbox{$TVaR_{\alpha}(X)	=	\esp{X | X \leq \alpha}	=	\frac{1}{\alpha} \int_{-\infty}^{VaR_{\alpha}} x f_{X}(x)dx$};
	\item	Si $X$ représente les pertes, on s'intéresse à l'extrémité supérieure de la distribution des gains et \icbox{$TVaR_{\alpha}(X)	=	\esp{X | X > \alpha}	=	\frac{1}{1 - \alpha} \int_{VaR_{\alpha}}^{\infty} x f_{X}(x)dx$};
\end{itemize}

\pagebreak

\section{Modèles de risques non-vie}
\begin{distributions}[Notation]
\begin{description}
	\item[$M$]	Variable aléatoire du nombre de sinistres pour un risque;
	\item[$B_{k}$]	Variable aléatoire du montant du $k^{\text{e}}$ sinistre.
\end{description}
\end{distributions}

\begin{definitionNOHFILL}[Modèle fréquence-sinistre]
On défini la v.a. $X$ comme étant les coûts (pertes) pour un risque tel que \icbox[red][palechestnut]{$\forall	M	>	0$}:
\begin{align*}
	X
	&=	\sum_{k	=	1}^{M} B_{k}
\end{align*}

\tcbline

\begin{align*}
	\esp{X}
	&=	\text{E}_{\text{M}}\left[\text{E}_{\text{B}}[X | M]\right]	\\
	&=	\text{E}[M]	\times	\text{E}[B]	\\
	\text{Var}(X)
	&=	\underbrace{\text{Var}_{\text{M}}(\text{E}_{\text{B}}[X | M])}_{\text{variabilité du \textit{nombre} de sinistres}}	+	\underbrace{\text{E}_{\text{M}}\left[\text{Var}_{\text{B}}(X | M)\right]}_{\text{variabilité du \textit{coût} par sinistre}}	\\
	&=	\text{E}[M]\text{Var}(B)	+	\text{E}^{2}[B]\text{Var}(M)	\\
\end{align*}

\tcbline

\begin{align*}
	F_{X}(x)
	&=	\Pr(M	=	0)	+	\sum_{k	=	1}^{\infty} \Pr(M	=	k)F_{B_{1}	+	\dots	+	B_{k}}(x)
\end{align*}

Par exemple, pour $B_{k}	\sim	\Gamma(\alpha,	\beta)$:
\begin{align*}
	F_{X}(x)
	&=	\Pr(M	=	0)	+	\sum_{k	=	1}^{\infty} \Pr(M	=	k)H(x;	\alpha k, \beta)
\end{align*}

\tcbline

\begin{align*}
	\mathcal{L}_{X}(t)
	&=	P_{M}\left(\mathcal{L}_{B}(t)\right), \quad	t > 0	\\
	\esp{X	\times	\bm{1}_{\{X	>	b\}}}
	&=	\sum_{k	=	1}^{\infty} \Pr(M	=	k)E\left[(B_{1}	+	\dots	+	B_{k})	\times	\bm{1}_{\{B_{1}	+	\dots	+	B_{k} > b\}}\right]
\end{align*}

Par exemple, pour $B_{k}	\sim	\Gamma(\alpha,	\beta)$:
\begin{align*}
	\esp{X	\times	\bm{1}_{\{X	>	b\}}}
	&=	\sum_{k	=	1}^{\infty} \Pr(M	=	k) \frac{k\alpha}{\beta} \overline{H}(b;	\alpha k + 1, \beta)
\end{align*}
\end{definitionNOHFILL}

\end{multicols*}
\end{document}

