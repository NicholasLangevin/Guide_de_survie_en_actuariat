% ACT-4105 / Automne 2019
% Chapitre 10 : ASP12
\section{ASP \# 12}
\subsection{Définitions}
\begin{description}
\item[Crédibilité] Mesure de la valeur prédictive accordée aux données
\item[Homogénéité] Degré auxquels les résultats espérés dans une classe de risque ont des valeurs comparables
\item[Pratique] Approche réaliste sachant l'objet, la nature et la portée du travail à réaliser
\item[Risque] Individus (ou compagnies) couverts par un système financier
\item[Caractéristique d'un risque] Caractéristiques mesurables et observables qui sont utilisés pour assigner à un risque une classe de risque
\item[Classe de risque] Sous-ensemble de risques groupés ensemble sous un système de classification des risques
\item[Système de classification des risques] système utilisé pour assigner chaque risque à un groupe basé sur les coûts espérés de la couverture fournie.
\end{description}

\subsection{Considérations pour la sélection des caractéristiques de risques}
\begin{itemize}
\item Les caractéristiques doivent être corrélées avec les coûts espérés. L'interdépendance entre les caractéristiques de risque doit aussi être considérée.
\item Il doit y avoir un lien entre les caractéristiques et coûts espérés, mais pas nécessairement un lien de cause à effet
\item Doit être basée sur des faits observables et vérificables, qui ne peuvent pas être manipulés
\item La caractéristique doit être \textbf{pratique} (au niveau des coûts,  temps et efforts requis pour l'obtenir)
\item Doit respecter les lois applicables
\item Suivre les pratiques de l'industrie et de la compagnie
\end{itemize}

\subsection{Considérations dans l'établissement des classes de risques}
\begin{itemize}
\item Quelle sera l'utilisation ? (établissement réserve ou tarification?)
\item Considérations actuarielles (anti-sélection, crédibilité, pratique , ...)
\item Autres considérations : Légales, pratiques de l'industrie, pratiques de la compagnies, ...
\item Vérifier que les résultats sont raisonnables et consistants
\end{itemize}

\subsection{Validation du système de classification des risques}
\begin{itemize}
\item Estimer/quantifier l'effet de l'anti-sélection
\item Utiliser les classes de risque différentes pour le test
\item Tester l'effet des différents changements (système de classif., pratiques de l'industrie, pratiques de la compagnie, etc..)
\end{itemize}