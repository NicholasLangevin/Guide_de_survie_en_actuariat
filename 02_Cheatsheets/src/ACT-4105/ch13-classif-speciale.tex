% ACT-4105
% Classification spéciale
\section{Classification spéciale}
\subsection{Analyse de territoire}
%TODO
\hl{TODO : il manque des infos}

\begin{itemize}
\item Déterminer les unités géographiques
\item Calculer l'estimateur géographique
\item Lissage (\textit{smoothing})
\begin{itemize}
	\item Basé sur la distance
	\item Basé sur les unités adjacents
\end{itemize}
\item Regroupement de territoires
\item Calculer les différentiels par territoire
\end{itemize}

\subsection{Tarification des limites augmentées} On utilise les \textit{increased limit factors} (ILF) pour augmenter le taux de base si l'assuré sélectionne une limite $H$ supérieure à la limite de base $B$. Le ILF indiqué est
\begin{equation}
ILF_I(H) = \frac{(\overline{L+E_L})_H}{(\overline{L+E_L})_B}
\end{equation}

Si la fréquence et la sévérité sont indépendants, on a

\begin{equation}
ILF_I(H) =\frac{\text{Fréquence}_H \times \text{Sévérité}_H}{\text{Fréquence}_B \times \text{Sévérité}_B}
\end{equation}

Si la fréquence est la même (peu importe la limite), alors on peut simplifier à 

\begin{align*}
ILF_I(H) =\frac{\text{Sévérité}_H}{\text{Sévérité}_B} = \frac{LAS(H)}{LAS(B)}
\end{align*}
où $LAS(H)$ est le \textit{limited average severity at $H$ limit}.

\subsubsection{Approche standard de calcul des ILF (sinistres censurés)}
Voir exemple 13.2 au besoin. De façon générale, pour $H_1 \leq H_2$, on a
\begin{align*}
LAS(H_2) = LAS(H_1) + LAS(H_2 - H_1 | X \geq H_1) \prob{X \geq H_1}
\end{align*}
On doit utiliser les sinistres projetés (avec facteur de tendance) à l'ultime

\subsection{Tarificaiton de franchise}

\subsubsection{\textit{Loss Elimination Ratio (LER) approach}}
Dans le cas discret,
\begin{align*}
LER(d) = 1 - \frac{\sum_{x_i} \max (x_i - d ; 0)}{\sum_{x_i} x_i}
\end{align*}
Dans le cas continu,
\hl{TODO}

Si on veut mesurer le LER suite à un changement de franchise (de $B$ à $D$) : 
\begin{align*}
LER(B \rightarrow D) = \frac{(L + E_L)_B - (L+E_L)_D}{(L+E_L)_B}
\end{align*}

\subsection{Grosseurs des risques en \textit{workers compensation}}

\subsubsection{Composante de frais (\textit{fees component}} En \textit{Workers Compensation}, on utilise la méthode \textit{All Variable Approach} pour déterminer la provision pour frais. Or, une petite compagnie (i.e. petite prime) seront sous-tarifé par rapport aux frais et l'inverse pour les grandes compagnies. \textbf{on doit apporter quelques ajustements}
\begin{itemize}
\item Calculer une provision pour frais qui s'appliquera seulement sur le premier 5000\$.
\item Charger un frais constant à tous les risques
\item Accorder un rabais aux polices ayant une prime supérieure à un certain montant (\textbf{escompte graduée}).
\end{itemize}

\paragraph{Calcul de l'escompte graduée} \hl{Voir au besoin exemple 13.5}


\subsubsection{Composante de perte (\textit{loss components})} Les petits risques ont tendance à avoir une expérience (en \% de la prime) moins favorables que les gros risques. Il faut donc ajouter un frais constant pour perte, afin que la prime soit adéquate
\hl{Voir exemple 13.6}

\subsection{Assurance à la valeur (\textit{Insurance-to-value, ITV})} Lorsqu'un assuré est en sous-assurance, la prime n'est pas suffisante pour couvrir les paiements espérés. Il y a 2 façon de régler les problèmes de sous-assurance : coassurance ou chager une prime qui tient réellement compte des sinistres espérés.

\subsubsection{Coassurance}
% Début split en 2 colonnes
\begin{multicols*}{2}
L'indemnité payée par l'assureur est
\begin{align*}
I = L \times \frac{F}{cV}
\end{align*}
Donc, la pénalité est définie par
\begin{align*}
e = 
\begin{cases}
L - I	& , L \leq F \\
F - I 	& , F < L < cV \\
0		& , L \geq cV \\
\end{cases}
\end{align*}
\vfill
\columnbreak
où
\begin{description}
\item[I] Indemnité
\item[L] Montant du sinistre après franchise
\item[F] Montant d'assurance sur la police
\item[V] Valeur de la propriété
\item[c] \% de coassurance exigé par le contrat
\item[e] pénalité
\end{description}
\end{multicols*}
% fin split en 2 colonnes

\subsubsection{Variation des taux selon le niveau d'assurance à la valeur}
\hl{TODO: par sûr de comprendre}