% ACT-4105
% Classification multivariée
\section{Classification multivariée}

\subsection{Défaillance des méthodes de classification traditionnelles}
\begin{itemize}
\item La méthode de la PP ne considère pas la corrélation entre les variables dans les unités d'exposition
\item La méthode du taux de sinistre / PP ajustée ne tiennent compte que \underline{partiellement} de la distribution du portefeuille.
\end{itemize}

\subsection{Bénéfice des méthodes de classification multivariées}
\begin{enumerate}
\item Considèrent toutes les variables simultanément et ajustement automatiquement pour la corrélation entre les variables.
\item Tentent de capturer les effets systématiques (signal) et non les effets non-systématiques (bruit)
\item Produisent des modèles diagnostics
\item Permettent d'inclure une considération pour les interactions entre 2 variables
\end{enumerate}

\subsection{Modèles linéaires généralisés (GLM)}

\subsubsection{Tests de diagnostic}
\begin{itemize}
\item Calcul de l'écart-type
\item Consistance dans les résultats d'une année à l'autre
\item Holdout : comparer le résultats espéré prédit vs résultat sur Holdout
\item Test statistique (Chi-Squared) :
\begin{itemize}
	\item $H_0$ : le modèle actuel (sans la variable) est adéquat
	\item Selon la valeur de la \textit{p-value} : 
	\begin{align*}
\text{Décision} = 
\begin{cases}
\text{p-value} < 0.05 & , \text{Rejète } H_0 \\
0.05 \leq \text{p-value} \leq 0.30 & \text{On ne peut rien conclure} \\
\text{p-value} > 0.30 & , \text{On ne rejète pas } H_0 \\
\end{cases}
\end{align*}
\end{itemize}
\end{itemize}