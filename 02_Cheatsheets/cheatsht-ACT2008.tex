%% Aide-mémoire
\documentclass[10pt, french]{article}
%% -----------------------------
%% 	Préambule
%% -----------------------------
% !TEX encoding = UTF-8 Unicode
% LaTeX Preamble for all cheatsheets
% Author : Gabriel Crépeault-Cauchon

% HOW-TO : copy-paste this file in the same directory as your .tex file, and add in your preamble the next command right after you have specified your documentclass : 
% \input{preamble-cheatsht.tex}
% ---------------------------------------------
% ---------------------------------------------

% Extra note : this preamble creates document that are meant to be used inside the multicols environment. See the documentation on internet for further information.

%% -----------------------------
%% Encoding packages
%% -----------------------------
\usepackage[utf8]{inputenc}
\usepackage[T1]{fontenc}
\usepackage{babel}
\usepackage{lmodern}

%% -----------------------------
%% Variable definition
%% -----------------------------
\def\auteur{Gabriel Crépeault-Cauchon / Nicholas Langevin}
\def\BackgroundColor{white}

%% -----------------------------
%% Margin and layout
%% -----------------------------
% Determine the margin for cheatsheet
\usepackage[landscape, hmargin=1cm, vmargin=1.7cm]{geometry}
\usepackage{multicol}

% Remove automatic indentation after section/subsection title.
\setlength{\parindent}{0cm}

% Save space in cheatsheet by removing space between align environment and normal text.
\usepackage{etoolbox}
\newcommand{\zerodisplayskips}{%
  \setlength{\abovedisplayskip}{0pt}%
  \setlength{\belowdisplayskip}{0pt}%
  \setlength{\abovedisplayshortskip}{0pt}%
  \setlength{\belowdisplayshortskip}{0pt}}
\appto{\normalsize}{\zerodisplayskips}
\appto{\small}{\zerodisplayskips}
\appto{\footnotesize}{\zerodisplayskips}

%% -----------------------------
%% URL and links
%% -----------------------------
\usepackage{hyperref}
\hypersetup{colorlinks = true, urlcolor = gray!70!white, linkcolor = black}

%% -----------------------------
%% Document policy (uncomment only one)
%% -----------------------------
%	\usepackage{concrete}
	\usepackage{mathpazo}
%	\usepackage{frcursive} %% permet d'écrire en lettres attachées
%	\usepackage{aeguill}
%	\usepackage{mathptmx}
%	\usepackage{fourier} 

%% -----------------------------
%% Math configuration
%% -----------------------------
\usepackage[fleqn]{amsmath}
\usepackage{amsthm,amssymb,latexsym,amsfonts}
\usepackage{empheq}
\usepackage{numprint}
\usepackage{dsfont} % Pour avoir le symbole du domaine Z

% Mathematics shortcuts

\newcommand{\reels}{\mathbb{R}}
\newcommand{\entiers}{\mathbb{Z}}
\newcommand{\naturels}{\mathbb{N}}
\newcommand{\eval}{\biggr \rvert}
\usepackage{cancel}
\newcommand{\derivee}[1]{\frac{\partial}{\partial #1}}
\newcommand{\prob}[1]{\Pr \left( #1 \right)}
\newcommand{\esp}[1]{\mathrm{E} \left[ #1 \right]} % espérance
\newcommand{\variance}[1]{\mathrm{Var} \left( #1   \right)}
\newcommand{\covar}[1]{\mathrm{Cov} \left( #1   \right)}
\newcommand{\laplace}{\mathcal{L}}
\newcommand{\deriv}[2][]{\frac{\partial^{#1}}{\partial #2^{#1}}}
\newcommand{\e}[1]{\mathrm{e}^{#1}}
\newcommand{\te}[1]{\text{exp}\left\{#1\right\}}
\DeclareMathSymbol{\shortminus}{\mathbin}{AMSa}{"39}



% To indicate equation number on a specific line in align environment
\newcommand\numberthis{\addtocounter{equation}{1}\tag{\theequation}}

%
% Actuarial notation packages
%
\usepackage{actuarialsymbol}
\usepackage{actuarialangle}

%
% Matrix notation for math symbols (\bm{•})
%
\usepackage{bm}
% Matrix notation variable (bold style)
\newcommand{\matr}[1]{\mathbf{#1}}



%% -----------------------------
%% tcolorbox configuration
%% -----------------------------
\usepackage[most]{tcolorbox}
\tcbuselibrary{xparse}
\tcbuselibrary{breakable}

%%
%% Coloured box "definition" for definitions
%%
\DeclareTColorBox{definition}{ o }				% #1 parameter
{
	colframe=blue!60!green,colback=blue!5!white, % color of the box
	breakable, 
	pad at break* = 0mm, 						% to split the box
	title = {#1},
	after title = {\large \hfill \faBook},
}
%%
%% Coloured box "definition2" for definitions
%%
\DeclareTColorBox{definitionNOHFILL}{ o }				% #1 parameter
{
	colframe=blue!60!green,colback=blue!5!white, % color of the box
	pad at break* = 0mm, 						% to split the box
	title = {#1},
	before title = {\faBook \quad },
	breakable
}


%%
%% Coloured box "algo" for algorithms
%%
\newtcolorbox{algo}[ 1 ]
{
	colback = blue!5!white,
	colframe = blue!75!black,
	title=#1,
	fonttitle = \bfseries,
	breakable
}
%%
%% Coloured box "conceptgen" for points adding to a concept's deifintion
%%
\newtcolorbox{conceptgen}[ 1 ]
{
	breakable,
	colback = beaublue,
	colframe = airforceblue,
	title=#1,
	fonttitle = \bfseries
}
%%
%% Coloured box "probch3" pour formules relatives au 3ème chapitre de prob
%%
\newtcolorbox{probch3}[ 1 ]
{
	colback = ruddypink,
	colframe = burgundy,
	fonttitle = \bfseries,	
	breakable,
	title=#1
}
%%
%% Coloured box "formula" for formulas
%%
\newtcolorbox{formula}[ 1 ]
{
	colback = green!5!white,
	colframe = green!70!black,
	breakable,
	fonttitle = \bfseries,
	title=#1
}
%%
%% Coloured box "formula" for formulas
%%
\DeclareTColorBox{algo2}{ o }
{
	enhanced,
	title = #1,
	colback=blue!5!white,	
	colbacktitle=blue!75!black,
	fonttitle = \bfseries,
	breakable,
	boxed title style={size=small,colframe=arsenic} ,
	attach boxed title to top center = {yshift=-3mm,yshifttext=-1mm},
}
%%
%% Coloured box "examplebox" for formulas
%%
\newtcolorbox{examplebox}[ 1 ]
{
	colback = lightmauve,
	colframe = antiquefuchsia,
	breakable,
	fonttitle = \bfseries,title=#1
}
%%
%% Coloured box "rappel" pour rappel de formules
%%
\newtcolorbox{rappel}[ 1 ]
{
	colback = ashgrey,
	colframe = arsenic,
	breakable,
	fonttitle = \bfseries,title=#1
}
%%
%% Coloured box "rappel" pour rappel de formules
%%
\DeclareTColorBox{rappel_enhanced}{ o }
{
	enhanced,
	title = #1,
	colback=ashgrey, % color of the box
%	colframe=blue(pigment),
%	colframe=arsenic,	
	colbacktitle=arsenic,
	fonttitle = \bfseries,
	breakable,
	boxed title style={size=small,colframe=arsenic} ,
	attach boxed title to top center = {yshift=-3mm,yshifttext=-1mm},
}
%%
%% Coloured box "notation" for notation and terminology
%%
\DeclareTColorBox{distributions}{ o }			% #1 parameter
{
	enhanced,
	title = #1,
	colback=gray(x11gray), % color of the box
%	colframe=blue(pigment),
	colframe=arsenic,	
	colbacktitle=aurometalsaurus,
	fonttitle = \bfseries,
	boxed title style={size=small,colframe=arsenic} ,
	attach boxed title to top center = {yshift=-3mm,yshifttext=-1mm},
	breakable
%	left=0pt,
%  	right=0pt,
%    box align=center,
%    ams align*
%  	top=-10pt
}

%% -----------------------------
%% Graphics and pictures
%% -----------------------------
\usepackage{graphicx}
\usepackage{pict2e}
\usepackage{tikz}

%% -----------------------------
%% insert pdf pages into document
%% -----------------------------
\usepackage{pdfpages}

%% -----------------------------
%% Color configuration
%% -----------------------------
\usepackage{color, soulutf8, colortbl}


%
%	Colour definitions
%
\definecolor{blue(munsell)}{rgb}{0.0, 0.5, 0.69}
\definecolor{blue(matcha)}{rgb}{0.596, 0.819, 1.00}
\definecolor{blue(munsell)-light}{rgb}{0.5, 0.8, 0.9}
\definecolor{bleudefrance}{rgb}{0.19, 0.55, 0.91}
\definecolor{blizzardblue}{rgb}{0.67, 0.9, 0.93}
\definecolor{bondiblue}{rgb}{0.0, 0.58, 0.71}
\definecolor{blue(pigment)}{rgb}{0.2, 0.2, 0.6}
\definecolor{bluebell}{rgb}{0.64, 0.64, 0.82}
\definecolor{airforceblue}{rgb}{0.36, 0.54, 0.66}
\definecolor{beaublue}{rgb}{0.74, 0.83, 0.9}
\definecolor{cobalt}{rgb}{0.0, 0.28, 0.67}	% nice light blue-ish
\definecolor{blue_rectangle}{RGB}{83, 84, 244}		% ACT-2004
\definecolor{indigo(web)}{rgb}{0.29, 0.0, 0.51}	% purple-ish
\definecolor{antiquefuchsia}{rgb}{0.57, 0.36, 0.51}	%	pastel dark purple ish
\definecolor{darkpastelpurple}{rgb}{0.59, 0.44, 0.84}
\definecolor{gray(x11gray)}{rgb}{0.75, 0.75, 0.75}
\definecolor{aurometalsaurus}{rgb}{0.43, 0.5, 0.5}
\definecolor{ruddypink}{rgb}{0.88, 0.56, 0.59}
\definecolor{pastelred}{rgb}{1.0, 0.41, 0.38}		
\definecolor{lightmauve}{rgb}{0.86, 0.82, 1.0}
\definecolor{azure(colorwheel)}{rgb}{0.0, 0.5, 1.0}
\definecolor{darkgreen}{rgb}{0.0, 0.2, 0.13}			
\definecolor{burntorange}{rgb}{0.8, 0.33, 0.0}		
\definecolor{burntsienna}{rgb}{0.91, 0.45, 0.32}		
\definecolor{ao(english)}{rgb}{0.0, 0.5, 0.0}		% ACT-2003
\definecolor{amber(sae/ece)}{rgb}{1.0, 0.49, 0.0} 	% ACT-2004
\definecolor{green_rectangle}{RGB}{131, 176, 84}		% ACT-2004
\definecolor{red_rectangle}{RGB}{241,112,113}		% ACT-2004
\definecolor{amethyst}{rgb}{0.6, 0.4, 0.8}
\definecolor{amethyst-light}{rgb}{0.6, 0.4, 0.8}
\definecolor{ashgrey}{rgb}{0.7, 0.75, 0.71}			% dark grey-black-ish
\definecolor{arsenic}{rgb}{0.23, 0.27, 0.29}			% light green-beige-ish gray
\definecolor{amaranth}{rgb}{0.9, 0.17, 0.31}
\definecolor{brickred}{rgb}{0.8, 0.25, 0.33}
\definecolor{pastelred}{rgb}{1.0, 0.41, 0.38}

%
% Useful shortcuts for coloured text
%
\newcommand{\orange}{\textcolor{orange}}
\newcommand{\red}{\textcolor{red}}
\newcommand{\cyan}{\textcolor{cyan}}
\newcommand{\blue}{\textcolor{blue}}
\newcommand{\green}{\textcolor{green}}
\newcommand{\purple}{\textcolor{magenta}}
\newcommand{\yellow}{\textcolor{yellow}}

%% -----------------------------
%% Enumerate environment configuration
%% -----------------------------
%
% Custum enumerate & itemize Package
%
\usepackage{enumitem}
%
% French Setup for itemize function
%
\frenchbsetup{StandardItemLabels=true}
%
% Change default label for itemize
%
\renewcommand{\labelitemi}{\faAngleRight}


%% -----------------------------
%% Tabular column type configuration
%% -----------------------------
\newcolumntype{C}{>{$}c<{$}} % math-mode version of "l" column type
\newcolumntype{L}{>{$}l<{$}} % math-mode version of "l" column type
\newcolumntype{R}{>{$}r<{$}} % math-mode version of "l" column type
\newcolumntype{f}{>{\columncolor{green!20!white}}p{1cm}}
\newcolumntype{g}{>{\columncolor{green!40!white}}m{1.2cm}}
\newcolumntype{a}{>{\columncolor{red!20!white}$}p{2cm}<{$}}	% ACT-2005
% configuration to force a line break within a single cell
\usepackage{makecell}


%% -----------------------------
%% Fontawesome for special symbols
%% -----------------------------
\usepackage{fontawesome}

%% -----------------------------
%% Section Font customization
%% -----------------------------
\usepackage{sectsty}
\sectionfont{\color{\SectionColor}}
\subsectionfont{\color{\SubSectionColor}}

%% -----------------------------
%% Footer/Header Customization
%% -----------------------------
\usepackage{lastpage}
\usepackage{fancyhdr}
\pagestyle{fancy}

%
% Header
%
\fancyhead{} 	% Reset
\fancyhead[L]{Aide-mémoire pour~ \cours ~(\textbf{\sigle})}
\fancyhead[R]{\auteur}

%
% Footer
%
\fancyfoot{}		% Reset
\fancyfoot[R]{\thepage ~de~ \pageref{LastPage}}
\fancyfoot[L]{\href{https://github.com/ressources-act/Guide_de_survie_en_actuariat}{\faGithub \ ressources-act/Guide de survie en actuariat}}
%
% Page background color
%
\pagecolor{\BackgroundColor}




%% END OF PREAMBLE
% ---------------------------------------------
% ---------------------------------------------
%% -----------------------------
%% Variable definition
%% -----------------------------
\def\cours{Mathématiques actuarielles IARD II}
\def\sigle{ACT-2008}
\def\SubSectionColor{blue!50!white}
\def\SectionColor{blue}
%
%	Save more space than default
%
\setlength{\abovedisplayskip}{-15pt}

%% -----------------------------
%% Begin document
%% -----------------------------
\begin{document}

\begin{center}
	\textsc{\Large Contributeurs}\\[0.5cm] 
\end{center}
\begin{contrib}{ACT-2008\: Mathématiques actuarielles IARD II}
\begin{description}
	\item[aut.] Nicholas Langevin
	\item[aut.] Gabriel Crépeault-Cauchon 
	\item[aut.] Alexandre Turcotte 
	\item[src.] Vincent Goulet
\end{description}
\end{contrib}


\raggedcolumns
\newpage
\begin{multicols*}{3} % Permet d'écrire dans plusieurs colonnes l'entièreté du document
% -------------------------------------------------------------
\section{Introduction et perspective historique}
Section plus qualitative, à compléter plus tard (sous forme de checklist)



\columnbreak
\section{Crédibilité de stabilité}
\subsection*{Définition de la crédibilité totale}
\begin{definition}[Crédibilité complète]
Une crédibilité complète d'ordre $(k,p)$ est attribuée à l'expérience $S$ d'un contrat si les paramètres de la distribution de $S$ sont tels que la relation
\[\prob{(1-k)\esp{S} \leq S \leq (1+k) \esp{S}} \geq p \]
est vérifiée. Par le théorème Central Limite, on peut démontrer que ça revient à respecter l'inégalité suivante : 
\begin{equation}
\esp{S} \geq \left( \frac{\Phi^{-1}\left(\frac{p+1}{2} \right)}{k} \right) \sqrt{\variance{S}}
\end{equation}
\end{definition}

\subsection*{Nombre de sinistres dans une période}
Soit $S = X_1 + ... + X_N$, avec $N \sim Pois(\lambda)$ et $X$ qui a une fonction de répartition $F_X$. On cherche le nombre moyen de sinistres $\lambda$ qui donne une plein crédibilité à l'espérience $S$. On peut démontrer que
\begin{equation}
\lambda \geq \left( \frac{\Phi^{-1}\left(\frac{p+1}{2} \right)}{k} \right)^2 \cdot \left(1 + \frac{\variance{X}}{\esp{X}^2} \right)
\end{equation}
\paragraph{Note} si $X$ est une v.a. dégénérée (i.e. $\prob{X = m} = 1$ pour un $m$ fixé), alors $\variance{X} = 0$ et $\lambda \geq 1082,41$.

\subsection*{Nombre d'années d'expérience $n$}
Soit la v.a. $W = \frac{S_1 + ... + S_n}{n}$. On a donc $\esp{W} = \esp{S}$ et $\variance{W} = \frac{\variance{S}}{n}$. On cherche le nombre d'années d'expérience $n$ nécessaire pour attribuer une pleine crédibilité au contrat. On peut démontrer que
\begin{equation}
n \geq \left( \frac{\Phi^{-1}\left(\frac{p+1}{2} \right)}{k} \right)^2 \cdot \frac{\variance{S}}{\esp{S}^2}
\end{equation}

\subsection*{Nombre d'employés / unité d'exposition}
Soit $S \sim Bin(n, \theta)$ qui représente le nombre de sinistres pour un groupe de $n$ employés. On cherche le nombre minimal $n$ d'employés nécessaires dans un groupe pour attribuer une pleine crédibilité au contrat. On peut démontrer que
\begin{equation}
n \geq \left( \frac{\Phi^{-1}\left(\frac{p+1}{2} \right)}{k} \right)^2 \cdot \frac{1 - \theta}{\theta}
\end{equation}

\subsection*{Définition crédibilité partielle}
\begin{definition}[Crédibilité partielle]
La crédibilité partielle permet de pondérer l'expérience $S$ d'un contrat et la prime collective $m$ par un facteur de crédibilité $z$, avec $0 < z < 1$, afin d'obtenir une prime linéaire de la forme
\[\pi = z S + (1-z) m\]
\end{definition}
\begin{itemize}
\item Plusieurs formules ont été proposés, on retient celle de Whitney : 
\begin{equation}
z = \frac{n}{n+K}
\end{equation}
\item Dans l'approche de crédibilité de stabilité, on met de côté le concept de précision pour éviter d'avoir des primes qui fluctuent beaucoup d'une année à l'autre.
\item \textbf{Complément de crédibilité} : en pratique, le complément de crédibilité ($1-z$) n'est pas donné entièrement à la prime collective $m$. Il peut y avoir une proportion reliée à autre chose.
\end{itemize}


\columnbreak
\section{Tarification Bayésienne}
\subsection*{Modèle d'hétérogénéité}
\begin{description}
\item[$\Theta_i$] niveau de risque du contrat $i$
\item[$U(\Theta)$] fonction de répartition de $\Theta$ (fonction de \emph{structure})
\item[$u(\theta)$] fonction de densité/masse de probabilité de $\Theta$
\end{description}
\paragraph{Hypothèses}
\begin{enumerate}
\item Les observations du contrat $i$ sont \emph{conditionnellement indépendantes}\footnote{Concept de contagion apparente} et $\emph{iid}$ avec fonction de répartition $F_{X|\Theta}$
\item Les variables $\Theta_1, ..., \Theta_I$ sont \emph{iid} avec fonction de répartition $U(\Theta)$
\item Les $I$ contrats du portefeuille sont indépendants
\end{enumerate}

\subsection*{Définition des 3 types de primes}
\begin{definition}[Prime de risque]
Si on connait le niveau de risque du contrat $i$, alors la meilleure prévision est la \textbf{prime de risque} : 
\begin{equation}
\mu(\theta_i) = \esp{S_{it} | \Theta_i = \theta_i} = \int_{0}^{\infty} x f(x | \theta_i) dx
\end{equation}
La prime de risque $\mu(\theta_i)$ serait l'idéal, sauf qu'on ne connait pas le niveau de risque du contrat.
\end{definition}

\begin{definition}[Prime collective]
Il s'agit d'une moyenne pondérée de toutes les primes de risque possible pour un contrat donné : 
\begin{equation}
m = \esp{\mu(\Theta_i)} = \int_{-\infty}^{\infty} \mu(\theta) u(\theta) d\theta
\end{equation}
Cette prime est globalement adéquate, mais pas équitable (ou optimale).
\end{definition}

\begin{definition}[Prime Bayésienne]
La meilleure approximation de la prime de risque $\mu(\theta_i)$ est une fonction $g*(x_1, ..., x_n)$ qui minimise l'erreur quadratique. On peut prouver que cette fonction est la prime Bayésienne telle que
\begin{align*}
B_{i, n+1} & = \esp{\mu(\Theta_i) | S_{i1} = x_{i1}, ..., S_{in} =x_{in}} \\
& = \int_{-\inf}^{\infty} \mu(\theta) u(\theta | x_{i1}, ..., x_{in}) d \theta \numberthis
\end{align*}
\end{definition}
\begin{itemize}
\item Comme $m$, la prime Bayésienne est aussi une prime pondérée des primes de risque.
\item  La différence ici est qu'on utilise la \emph{distribution a postériori} de $\Theta_i$, i.e. la distribution révisée après avoir observé l'espérience $S_{i1}, ..., S_{in}$ : 
\begin{align*}
u(\theta_i | x_{i1}, ..., x_{in}) & = \frac{f(x_{i1}, ..., x_{in} | \theta_i) u(\theta_i)}{\int_{-\infty}^{\infty} f(x_{i1}, ..., x_{in} | \theta_i) u(\theta_i) d \theta_i} \\
& = \frac{\prod_{t=1}^{n}f(x_{it} | \theta_i) u(\theta_i)}{\int_{-\infty}^{\infty} \prod_{t=1}^{n}f(x_{it} | \theta_i) u(\theta_i) d \theta_i} \\
&  \propto u(\theta_i) \prod_{t=1}^{n} f(x_{it} | \theta_i)
\end{align*}
\end{itemize}


\subsection*{Calcul de la prime Bayésienne avec la distribution prédictive}
En plus de calculer $B_{i, n+1}$ avec les primes de risques, on peut aussi la calculer avec la distribution prédictive $S_{i, n+1} | S_1, ..., S_n$, avec la fonction de densité
\begin{align*}
f(x_{n+1} | x_1, ..., x_n) = \int_{-\infty}^{\infty} f(x | \theta) u(\theta | x_1, ..., x_n) d \theta
\end{align*}

\subsection*{Crédibilité bayésienne linéaire}
Certaines combinaison de distributions permettent d'obtenir une prime Bayésienne qui peut être exprimée sous la forme
\[\pi = z \bar{S} + (1-z) m \]
avec $z \in [0,1]$, qu'on appelle la prime de crédibilité.
\paragraph{Avantages}
\begin{itemize}
\item linéaire, donc facile à justifier/expliquer
\item lorsque $n \to \infty$, $z \to 1$, ce qui est aussi facile à justifier
\end{itemize}

Il existe 5 combinaisons de distribution qui résultent en une prime Bayésienne linéaire : 
\begin{itemize}
\item $S | \Theta \sim Pois(\Theta)$ et $\Theta \sim \Gamma(\alpha, \lambda)$
\item $S | \Theta \sim Exp(\Theta)$ et $\Theta \sim \Gamma(\alpha, \lambda)$
\item $S | \Theta \sim N(\Theta, \sigma_2^2)$ et $\Theta \sim N(\mu, \sigma_1^2)$
\item $S | \Theta \sim Bern(\Theta)$ et $\Theta \sim Beta(a,b)$
\item $S | \Theta \sim Géo(\Theta)$ et $\Theta \sim Beta(a,b)$
\end{itemize}

\subsection*{Modèle de Jewell}
\begin{itemize}
\item Si $u(\theta | x_1, ..., x_n)$ appartiennent à la même famille que $u(\theta)$, on dit de $u(\theta)$ et $f(x | \theta)$ qu'elles sont des \emph{conjugées naturelles}
\item Les loi Poisson, exponentielle, normale, Bernouilli et géométrique appartiennent à la famille exponentielle univariée, i.e. leur fonction de masse/densité peut être écrite sous la forme
\[f(x | \theta) =  \frac{p(x) e^{-\theta x}}{q(\theta)}   \]

\item Lorsqu'une fonction de vraisemblance $f(x|\theta)$ de la famille exponentielle univariée est combinée avec sa conjugée naturelle, alors la prime Bayésienne est toujours une prime de crédibilité exacte.
\end{itemize}


\columnbreak
\section{Modèle de crédibilité de \\ Bühlmann}
\subsection*{Notation et relation de covariance}
\begin{itemize}
    \item La prime collective. \[ m = \esp{\mu(\Theta_i)} \]
    \item La variance intra (\emph{within}) ou la variabilité moyenne du portefeuille. \[ s^2 = \esp{\sigma^2(\Theta_i)} \]
    \item La variance inter (\emph{between}) ou la variabilité entre les moyennes des contrat, ce qui représente l'homogénéité du portefeuille. \[ a = \variance{\mu(\Theta_i)} \]
\end{itemize}
\paragraph*{Covariance}
Soit $X,Y$ et $\Theta$ des variables aléatoires dont la densité conjointe existe. \[ \covar{X,Y} = \covar{\esp{X|\Theta_i}, \esp{Y|\Theta_i}} + \esp{\covar{X,Y|\Theta_i}} \]

\paragraph*{Application}
\begin{align*}
    \covar{S_t,S_u} &= a + \delta_{tu}s^2\quad t,u = 1,...,n  \\
    \covar{\mu(\Theta_i),S_t} &= a
\end{align*} \\
où $\delta_{iu}$ est le delta de Kronecker 
\[ 
\delta_{iu} = 
\left\{
\begin{array}{lc}
    1, & t = u \\
    0. & t \neq u
\end{array}
\right. 
\]

\subsection*{Modèle de prévision}
\begin{itemize}
    \item[(B1)] Les contrats $(\Theta_i, \mathbf{S}_i), i = 1,...,I$ sont indépendants, les variables aléatoire $\Theta_1,...,\Theta_I$ sont identiquement distribuées et les variances aléatoire $S_{it}$ ont une variance finie.
    \item[(B2)] Les variables aléatoires $S_{it}$, sont telles que 
    \begin{align*}
        \esp{S_{it}|\Theta_i} &= \mu(\Theta_i)\quad i = 1,...,I \\
        \covar{S_{it},S_{iu}|\Theta_i} &= \delta_{tu}\sigma^2(\Theta_i)\quad t,u = 1,...,n
    \end{align*}    
\end{itemize}
\begin{definition}[Prime de crédibilité]
    Pour un portefeuille sous les hypothèses (\emph{B1}) et (\emph{B2}), la meilleur approximation non homogène de la prime de risque $\mu(\Theta_i)$ est
    \[ \pi_{i,n+1}^B = z\bar{S}_i + (1 - z)m \]
    où
    \begin{align*}
        \bar{S}_i &= \frac{1}{n} \sum_{t=1}^n S_{it} \\
        z &= \frac{n}{n+K},\quad K = \frac{s^2}{a}
    \end{align*}
\end{definition}

\subsection*{Approche paramétrique}
L'approche paramétrique permet de retrouver la prime de crédibilité bayésienne. Puisque les distributions de ($S_t|\Theta = \theta$) et de ($\Theta$) sont connues, il est possible d'évaluer directement $m, s^2$ et $a$.

\subsection*{Approche non paramétrique}
Avec l'approche non paramétrique, nous délaissons l'approche bayésienne pure pour l'approche \textbf{bayésienne empirique}.
\begin{itemize}
    \item Estimation de la prime collective. \[ \hat{m} = \bar{S} =  \frac{1}{I} \sum_{i=1}^I \bar{S}_i \]
    \item Estimation de la variance intra. \[ \hat{s}^2 = \frac{1}{I} \sum_{i=1}^I \hat{\sigma}_{i,(n-1)}^2 \]
    \item Estimation de la variance inter. \[ \hat{a} = \frac{1}{I-1} \sum_{i=1}^I (\bar{S}_i - \hat{m})^2 - \frac{1}{n} \hat{s}^2 \]
\end{itemize}
Il est important de savoir que les estimateurs $\hat{m}, \hat{s}^2$ et $\hat{a}$ sont tous des estimateurs sans biais, mais $\hat{K}$ et donc $\hat{z}$ ne sont pas nécéssairement sans biais.

\subsection{Interprétation des résultats}
\begin{enumerate}
    \item Plus le nombre d'années est grand ($n\to\infty$), plus l'expérience d'un contrat représente exactement son niveau de risque.
    \item Plus $s^2$ est petit ($s^2\to 0$), plus l'expérience est globalement stable dans le temps. Les moyennes individuelle $\bar{S}_i$ représente alors bien les niveaux de risque des contrats, ce qui réduit l'utilité de la prime collective.
    \item Plus $a$ est grand ($a\to\infty$), plus le portefeuille est hétérogène. Les moyennes individuelle $\bar{S}_i$ sont de meilleur approximation des primes de risque que la prime collective.
\end{enumerate}



% -------------------------------------------------------------
% Fin de la feuille aide-mémoire
\end{multicols*}
\end{document}
