\documentclass[10pt, french]{article}

%% -----------------------------
%% Préambule
%% -----------------------------
% !TEX encoding = UTF-8 Unicode
% LaTeX Preamble for all cheatsheets
% Author : Gabriel Crépeault-Cauchon

% HOW-TO : copy-paste this file in the same directory as your .tex file, and add in your preamble the next command right after you have specified your documentclass : 
% \input{preamble-cheatsht.tex}
% ---------------------------------------------
% ---------------------------------------------

% Extra note : this preamble creates document that are meant to be used inside the multicols environment. See the documentation on internet for further information.

%% -----------------------------
%% Encoding packages
%% -----------------------------
\usepackage[utf8]{inputenc}
\usepackage[T1]{fontenc}
\usepackage{babel}
\usepackage{lmodern}

%% -----------------------------
%% Variable definition
%% -----------------------------
\def\auteur{Gabriel Crépeault-Cauchon / Nicholas Langevin}
\def\BackgroundColor{white}

%% -----------------------------
%% Margin and layout
%% -----------------------------
% Determine the margin for cheatsheet
\usepackage[landscape, hmargin=1cm, vmargin=1.7cm]{geometry}
\usepackage{multicol}

% Remove automatic indentation after section/subsection title.
\setlength{\parindent}{0cm}

% Save space in cheatsheet by removing space between align environment and normal text.
\usepackage{etoolbox}
\newcommand{\zerodisplayskips}{%
  \setlength{\abovedisplayskip}{0pt}%
  \setlength{\belowdisplayskip}{0pt}%
  \setlength{\abovedisplayshortskip}{0pt}%
  \setlength{\belowdisplayshortskip}{0pt}}
\appto{\normalsize}{\zerodisplayskips}
\appto{\small}{\zerodisplayskips}
\appto{\footnotesize}{\zerodisplayskips}

%% -----------------------------
%% URL and links
%% -----------------------------
\usepackage{hyperref}
\hypersetup{colorlinks = true, urlcolor = gray!70!white, linkcolor = black}

%% -----------------------------
%% Document policy (uncomment only one)
%% -----------------------------
%	\usepackage{concrete}
	\usepackage{mathpazo}
%	\usepackage{frcursive} %% permet d'écrire en lettres attachées
%	\usepackage{aeguill}
%	\usepackage{mathptmx}
%	\usepackage{fourier} 

%% -----------------------------
%% Math configuration
%% -----------------------------
\usepackage[fleqn]{amsmath}
\usepackage{amsthm,amssymb,latexsym,amsfonts}
\usepackage{empheq}
\usepackage{numprint}
\usepackage{dsfont} % Pour avoir le symbole du domaine Z

% Mathematics shortcuts

\newcommand{\reels}{\mathbb{R}}
\newcommand{\entiers}{\mathbb{Z}}
\newcommand{\naturels}{\mathbb{N}}
\newcommand{\eval}{\biggr \rvert}
\usepackage{cancel}
\newcommand{\derivee}[1]{\frac{\partial}{\partial #1}}
\newcommand{\prob}[1]{\Pr \left( #1 \right)}
\newcommand{\esp}[1]{\mathrm{E} \left[ #1 \right]} % espérance
\newcommand{\variance}[1]{\mathrm{Var} \left( #1   \right)}
\newcommand{\covar}[1]{\mathrm{Cov} \left( #1   \right)}
\newcommand{\laplace}{\mathcal{L}}
\newcommand{\deriv}[2][]{\frac{\partial^{#1}}{\partial #2^{#1}}}
\newcommand{\e}[1]{\mathrm{e}^{#1}}
\newcommand{\te}[1]{\text{exp}\left\{#1\right\}}
\DeclareMathSymbol{\shortminus}{\mathbin}{AMSa}{"39}



% To indicate equation number on a specific line in align environment
\newcommand\numberthis{\addtocounter{equation}{1}\tag{\theequation}}

%
% Actuarial notation packages
%
\usepackage{actuarialsymbol}
\usepackage{actuarialangle}

%
% Matrix notation for math symbols (\bm{•})
%
\usepackage{bm}
% Matrix notation variable (bold style)
\newcommand{\matr}[1]{\mathbf{#1}}



%% -----------------------------
%% tcolorbox configuration
%% -----------------------------
\usepackage[most]{tcolorbox}
\tcbuselibrary{xparse}
\tcbuselibrary{breakable}

%%
%% Coloured box "definition" for definitions
%%
\DeclareTColorBox{definition}{ o }				% #1 parameter
{
	colframe=blue!60!green,colback=blue!5!white, % color of the box
	breakable, 
	pad at break* = 0mm, 						% to split the box
	title = {#1},
	after title = {\large \hfill \faBook},
}
%%
%% Coloured box "definition2" for definitions
%%
\DeclareTColorBox{definitionNOHFILL}{ o }				% #1 parameter
{
	colframe=blue!60!green,colback=blue!5!white, % color of the box
	pad at break* = 0mm, 						% to split the box
	title = {#1},
	before title = {\faBook \quad },
	breakable
}


%%
%% Coloured box "algo" for algorithms
%%
\newtcolorbox{algo}[ 1 ]
{
	colback = blue!5!white,
	colframe = blue!75!black,
	title=#1,
	fonttitle = \bfseries,
	breakable
}
%%
%% Coloured box "conceptgen" for points adding to a concept's deifintion
%%
\newtcolorbox{conceptgen}[ 1 ]
{
	breakable,
	colback = beaublue,
	colframe = airforceblue,
	title=#1,
	fonttitle = \bfseries
}
%%
%% Coloured box "probch3" pour formules relatives au 3ème chapitre de prob
%%
\newtcolorbox{probch3}[ 1 ]
{
	colback = ruddypink,
	colframe = burgundy,
	fonttitle = \bfseries,	
	breakable,
	title=#1
}
%%
%% Coloured box "formula" for formulas
%%
\newtcolorbox{formula}[ 1 ]
{
	colback = green!5!white,
	colframe = green!70!black,
	breakable,
	fonttitle = \bfseries,
	title=#1
}
%%
%% Coloured box "formula" for formulas
%%
\DeclareTColorBox{algo2}{ o }
{
	enhanced,
	title = #1,
	colback=blue!5!white,	
	colbacktitle=blue!75!black,
	fonttitle = \bfseries,
	breakable,
	boxed title style={size=small,colframe=arsenic} ,
	attach boxed title to top center = {yshift=-3mm,yshifttext=-1mm},
}
%%
%% Coloured box "examplebox" for formulas
%%
\newtcolorbox{examplebox}[ 1 ]
{
	colback = lightmauve,
	colframe = antiquefuchsia,
	breakable,
	fonttitle = \bfseries,title=#1
}
%%
%% Coloured box "rappel" pour rappel de formules
%%
\newtcolorbox{rappel}[ 1 ]
{
	colback = ashgrey,
	colframe = arsenic,
	breakable,
	fonttitle = \bfseries,title=#1
}
%%
%% Coloured box "rappel" pour rappel de formules
%%
\DeclareTColorBox{rappel_enhanced}{ o }
{
	enhanced,
	title = #1,
	colback=ashgrey, % color of the box
%	colframe=blue(pigment),
%	colframe=arsenic,	
	colbacktitle=arsenic,
	fonttitle = \bfseries,
	breakable,
	boxed title style={size=small,colframe=arsenic} ,
	attach boxed title to top center = {yshift=-3mm,yshifttext=-1mm},
}
%%
%% Coloured box "notation" for notation and terminology
%%
\DeclareTColorBox{distributions}{ o }			% #1 parameter
{
	enhanced,
	title = #1,
	colback=gray(x11gray), % color of the box
%	colframe=blue(pigment),
	colframe=arsenic,	
	colbacktitle=aurometalsaurus,
	fonttitle = \bfseries,
	boxed title style={size=small,colframe=arsenic} ,
	attach boxed title to top center = {yshift=-3mm,yshifttext=-1mm},
	breakable
%	left=0pt,
%  	right=0pt,
%    box align=center,
%    ams align*
%  	top=-10pt
}

%% -----------------------------
%% Graphics and pictures
%% -----------------------------
\usepackage{graphicx}
\usepackage{pict2e}
\usepackage{tikz}

%% -----------------------------
%% insert pdf pages into document
%% -----------------------------
\usepackage{pdfpages}

%% -----------------------------
%% Color configuration
%% -----------------------------
\usepackage{color, soulutf8, colortbl}


%
%	Colour definitions
%
\definecolor{blue(munsell)}{rgb}{0.0, 0.5, 0.69}
\definecolor{blue(matcha)}{rgb}{0.596, 0.819, 1.00}
\definecolor{blue(munsell)-light}{rgb}{0.5, 0.8, 0.9}
\definecolor{bleudefrance}{rgb}{0.19, 0.55, 0.91}
\definecolor{blizzardblue}{rgb}{0.67, 0.9, 0.93}
\definecolor{bondiblue}{rgb}{0.0, 0.58, 0.71}
\definecolor{blue(pigment)}{rgb}{0.2, 0.2, 0.6}
\definecolor{bluebell}{rgb}{0.64, 0.64, 0.82}
\definecolor{airforceblue}{rgb}{0.36, 0.54, 0.66}
\definecolor{beaublue}{rgb}{0.74, 0.83, 0.9}
\definecolor{cobalt}{rgb}{0.0, 0.28, 0.67}	% nice light blue-ish
\definecolor{blue_rectangle}{RGB}{83, 84, 244}		% ACT-2004
\definecolor{indigo(web)}{rgb}{0.29, 0.0, 0.51}	% purple-ish
\definecolor{antiquefuchsia}{rgb}{0.57, 0.36, 0.51}	%	pastel dark purple ish
\definecolor{darkpastelpurple}{rgb}{0.59, 0.44, 0.84}
\definecolor{gray(x11gray)}{rgb}{0.75, 0.75, 0.75}
\definecolor{aurometalsaurus}{rgb}{0.43, 0.5, 0.5}
\definecolor{ruddypink}{rgb}{0.88, 0.56, 0.59}
\definecolor{pastelred}{rgb}{1.0, 0.41, 0.38}		
\definecolor{lightmauve}{rgb}{0.86, 0.82, 1.0}
\definecolor{azure(colorwheel)}{rgb}{0.0, 0.5, 1.0}
\definecolor{darkgreen}{rgb}{0.0, 0.2, 0.13}			
\definecolor{burntorange}{rgb}{0.8, 0.33, 0.0}		
\definecolor{burntsienna}{rgb}{0.91, 0.45, 0.32}		
\definecolor{ao(english)}{rgb}{0.0, 0.5, 0.0}		% ACT-2003
\definecolor{amber(sae/ece)}{rgb}{1.0, 0.49, 0.0} 	% ACT-2004
\definecolor{green_rectangle}{RGB}{131, 176, 84}		% ACT-2004
\definecolor{red_rectangle}{RGB}{241,112,113}		% ACT-2004
\definecolor{amethyst}{rgb}{0.6, 0.4, 0.8}
\definecolor{amethyst-light}{rgb}{0.6, 0.4, 0.8}
\definecolor{ashgrey}{rgb}{0.7, 0.75, 0.71}			% dark grey-black-ish
\definecolor{arsenic}{rgb}{0.23, 0.27, 0.29}			% light green-beige-ish gray
\definecolor{amaranth}{rgb}{0.9, 0.17, 0.31}
\definecolor{brickred}{rgb}{0.8, 0.25, 0.33}
\definecolor{pastelred}{rgb}{1.0, 0.41, 0.38}

%
% Useful shortcuts for coloured text
%
\newcommand{\orange}{\textcolor{orange}}
\newcommand{\red}{\textcolor{red}}
\newcommand{\cyan}{\textcolor{cyan}}
\newcommand{\blue}{\textcolor{blue}}
\newcommand{\green}{\textcolor{green}}
\newcommand{\purple}{\textcolor{magenta}}
\newcommand{\yellow}{\textcolor{yellow}}

%% -----------------------------
%% Enumerate environment configuration
%% -----------------------------
%
% Custum enumerate & itemize Package
%
\usepackage{enumitem}
%
% French Setup for itemize function
%
\frenchbsetup{StandardItemLabels=true}
%
% Change default label for itemize
%
\renewcommand{\labelitemi}{\faAngleRight}


%% -----------------------------
%% Tabular column type configuration
%% -----------------------------
\newcolumntype{C}{>{$}c<{$}} % math-mode version of "l" column type
\newcolumntype{L}{>{$}l<{$}} % math-mode version of "l" column type
\newcolumntype{R}{>{$}r<{$}} % math-mode version of "l" column type
\newcolumntype{f}{>{\columncolor{green!20!white}}p{1cm}}
\newcolumntype{g}{>{\columncolor{green!40!white}}m{1.2cm}}
\newcolumntype{a}{>{\columncolor{red!20!white}$}p{2cm}<{$}}	% ACT-2005
% configuration to force a line break within a single cell
\usepackage{makecell}


%% -----------------------------
%% Fontawesome for special symbols
%% -----------------------------
\usepackage{fontawesome}

%% -----------------------------
%% Section Font customization
%% -----------------------------
\usepackage{sectsty}
\sectionfont{\color{\SectionColor}}
\subsectionfont{\color{\SubSectionColor}}

%% -----------------------------
%% Footer/Header Customization
%% -----------------------------
\usepackage{lastpage}
\usepackage{fancyhdr}
\pagestyle{fancy}

%
% Header
%
\fancyhead{} 	% Reset
\fancyhead[L]{Aide-mémoire pour~ \cours ~(\textbf{\sigle})}
\fancyhead[R]{\auteur}

%
% Footer
%
\fancyfoot{}		% Reset
\fancyfoot[R]{\thepage ~de~ \pageref{LastPage}}
\fancyfoot[L]{\href{https://github.com/ressources-act/Guide_de_survie_en_actuariat}{\faGithub \ ressources-act/Guide de survie en actuariat}}
%
% Page background color
%
\pagecolor{\BackgroundColor}




%% END OF PREAMBLE
% ---------------------------------------------
% ---------------------------------------------
%% -----------------------------
%% Variable definition
%% -----------------------------
\def\cours{Mathématiques actuarielles Vie II}
\def\sigle{ACT-2007}
\def\SectionColor{burntorange}
\def\SubSectionColor{burntsienna}
\def\SubSubSectionColor{burntsienna}

%% Reduce margin space
\setlength{\abovedisplayskip}{-15pt}

\newcommand{\bettershortstack}[2][c]{%
  \begin{tabular}[b]{@{}#1@{}}#2\end{tabular}%
}
\usepackage{stackengine}
\newcommand\cumlaut[2][black]{\stackon[.33ex]{#2}{\textcolor{#1}{\kern-.04ex.\kern-.2ex.}}}
%% -----------------------------
%% Début du document
%% -----------------------------
\begin{document}

\begin{center}
	\textsc{\Large Contributeurs}\\[0.5cm] 
\end{center}
\begin{contrib}{ACT-2007\: Mathématiques actuarielles vie II}
\begin{description}
	\item[aut.] Nicholas Langevin
	\item[aut.] Gabriel Crépeault-Cauchon 
	\item[aut.] Alexandre Turcotte 
	\item[aut.] Alec James van Rassel
	\item[ctb.] Olivier Côté
	\item[src.] Ilie-Radu Mitric
\end{description}
\end{contrib}


\newpage

\raggedcolumns
\begin{multicols*}{2} 

\subsection*{Rappels}

\subsubsection*{Approximation Woolhouse}
\begin{align*}
	\ax**{x\color{airforceblue}{:\angln}}[(m)]	
	&\approx	\ax**{x\textcolor{airforceblue}{:\angln}}	-	
		\frac{m - 1}{2m} \textcolor{airforceblue}{\left( 1 - \Ex[n]{x} \right)}	-	
		\frac{m^{2} - 1}{12m^{2}}(\delta + \mu_{x} \textcolor{airforceblue}{- \Ex[n]{x} (\delta + \mu_{x + n})})		\\
	\mu_{x}
	&\approx		-	\frac{1}{2} \left( \ln \px{x - 1} + \ln \px{x} \right)
\end{align*}	

\subsubsection*{Hypothèse DUD}
\textbf{Mortalité}
\begin{align*}
	\qx[s]{x}
	&=	s \qx{x}, \;	s\in (0, 1)
\end{align*}

\textbf{Assurance}
\begin{align*}
	\Ax*{x\textcolor{airforceblue}{:\angln}}
	&\overset{DUD}{=}	\frac{i}{\delta}\Ax{\nthtop{\textcolor{airforceblue}{1}}{x}\textcolor{airforceblue}{:\angln}}		{\color{airforceblue} + \Ax{x\textcolor{airforceblue}{:\nthtop{1}{\angln}}}}	\\
	\Ax{x:\textcolor{airforceblue}{:\angln}}[\textcolor{pastelred}{(m)}]
	&\overset{DUD}{=}	\frac{i}{i^{\textcolor{pastelred}{(m)}}}\Ax{\nthtop{\textcolor{airforceblue}{1}}{x}\textcolor{airforceblue}{:\angln}}		{\color{airforceblue} + \Ax{x\textcolor{airforceblue}{:\nthtop{1}{\angln}}}}	\\
\end{align*}

\textbf{Rentes}
\begin{align*}
	\ddot{a}_{x}^{(m)}
	&=	\alpha(m)\ddot{a}_{x}	-	\beta(m)		\\
	\ddot{a}_{x\textcolor{cobalt}{:\angln}}^{(m)}
	&=	\alpha(m)\ddot{a}_{x\textcolor{cobalt}{:\angln}}	-	\beta(m)\textcolor{cobalt}{(1 - \Ex[n]{x})}		\\
	\prescript{}{\textcolor{amethyst}{n|}}{\ddot{a}}_{x}^{(m)}
	&=	\alpha(m)\prescript{}{\textcolor{amethyst}{n|}}{\ddot{a}}_{x}	-	\beta(m)\textcolor{amethyst}{\Ex[n]{x}}		\\
	\ddot{a}_{\textcolor{azure(colorwheel)}{\overline{x:\angln}}}^{(m)}
	&=	\alpha(m)\ddot{a}_{\textcolor{azure(colorwheel)}{\overline{x:\angln}}}	-	\beta(m)\textcolor{azure(colorwheel)}{(1 - v^{n} + \Ex[n]{x})}		
\end{align*}
où:
\begin{align*}
	\alpha(m)	
	&=	\frac{id}{i^{(m)}d^{(m)}}	&
	\beta(m)	
	&=	\frac{i - i^{(m)}}{i^{(m)}d^{(m)}}
\end{align*}

\subsubsection*{Relations}
\textbf{Assurance}
\begin{align*}
	\Ax{x}	
	&=	v\qx{x} + v\px{x} \Ax{x + 1}		\\
\end{align*}

\textbf{Rente}
\begin{align*}
	\ax**{x}	
	&=	1 + v\px{x} \ax**{x + 1}		\\
	\ax**[n|]{x}
	&=	\Ex[n]{x} \ax**{x + n}	\\
	&=	\ax**{x} - \ax**{x:\angln}	\\
	\ax**{x:\angln}
	&=	\ax**{x} - \Ex[n]{x} \ax**{x + n}	\\
	\ax**{x:\angln}[\textcolor{airforceblue}{(m)}]
	&=	\frac{1}{\textcolor{airforceblue}{m}}	+
		\underbrace{v^{\textcolor{airforceblue}{\frac{1}{m}}} 
		\px[\textcolor{airforceblue}{\frac{1}{m}}]{x}}_{\Ex[\textcolor{airforceblue}{\frac{1}{m}}]{x}} \:
		\ax**{x + \frac{1}{\textcolor{airforceblue}{m}}:\angl{n - \frac{1}{\color{airforceblue}{m}}}}[\textcolor{airforceblue}{(m)}]	\\
\end{align*}

\textbf{Note rente différée}:	pas faire l'erreur d'oublier de soustraire les $n$ années sans paiements de la rente:
\begin{align*}
	\prescript{}{n|}{\cumlaut[cyan]{Y}}_{x} 
	&= 	\begin{cases}
			0	& , K = 0, 1, \dots, n - 1 \\
			v^{n} \cumlaut[cyan]{a}_{\angl{K \textcolor{cyan}{+ 1} - n}}			& , K = n, n + 1 \\
		\end{cases} 	\\
\end{align*}

\subsubsection*{Mortalité}
\textbf{Tables}
\begin{align*}
	\dx[t]{x}
	&=	\lx{x} - \lx{x + t}	\\
	\px[t]{x}
	&=	\frac{\lx{x + t}}{\lx{x}}	&
	\qx[t]{x}
	&=	\frac{\lx{x} - \lx{x + t}}{\lx{x}}	\\
	\px[t]{x}
	&=	\textrm{e}^{-\int_{0}^{t} \mu_{x + s} ds}
\end{align*}


\textbf{Sélection à l'âge $[x]$}
\begin{align*}
	\Ax*{\nthtop{1}{[x] + h}:\angl{n - h}}
	&=	\int_{0}^{n - h} \textrm{e}^{-\delta t} \px[t]{[x] + h} \mu_{[x] + h + t} dt	\\
	&=	\int_{h}^{n} \textrm{e}^{-\delta (s - h)} \frac{\px[s]{[x]}}{\px[h]{[x]}} \mu_{[x] + s} ds
\end{align*}
\pagebreak

\section{Calcul de réserve}

\subsection*{Notation}
\begin{itemize}[leftmargin = *]
	\item[] 	$\actsymb[h]{L}{}$: Perte nette future sur un contrat d'assurance pour un individu d'âge $(x)$ au temps $h$.
		\begin{itemize}[leftmargin = *]
		\item	Puisque la perte est évaluée au temps $h$, on suppose que l'assuré va décéder par après et conditionne à sa survie:
			\begin{align*}
			\actsymb[h]{L}{}	
			&=	\{ \actsymb[h]{L}{} | T_{x} > h \}
			\end{align*}
		\end{itemize}
	\item[]	$\actsymb[h]{V}{}$: Réserve nette pour un contrat d'assurance pour un individu d'âge $(x)$ au temps $h$.
		\begin{itemize}[leftmargin = *]
		\item	La réserve est basée sur ce qu'on s'attend à avoir comme perte:
			\begin{align*}
			\actsymb[h]{V}{}	
			&=	\text{E}[\actsymb[h]{L}{}]	
			\end{align*}
		\item[]	$\actsymb[h]{V}{}[g]$: Réserve pour contrat avec primes brutes (lorsqu'il y a des frais).
		\item[]	$\actsymb[h]{V}{}[n]$: Réserve pour contrat avec primes pures (lorsqu'il n'y a pas de frais).
		\end{itemize}
	\item[]	$\Vx[h]{}[I]$	Réserve initiale au début de l'année $h$;
		\begin{align*}
		\Vx[h]{}[I]	&=	\Vx[h]{} + \pi
		\end{align*}
	\item[]	$VP_{@h}$: La valeur présente au temps $h$.
	\item[]	$VPA_{@h}$: La valeur présente anticipée au temps $h$.
		\begin{align*}
		VPA_{@t}	
		&=	\text{E}[VP_{@h}]
		\end{align*}
\end{itemize}

\begin{examplebox}{Notation pour un contrat d'assurance vie entière}
\begin{align*}
	\actsymb[h]{L}{}	
	&=	M Z_{x + h} - \pi \ddot{Y}_{x + h}	\\
	\text{Var}(\actsymb[h]{L}{})
	&=	\left(M + \frac{\pi}{d}\right)^{2} \left[ \actsymb[][2]{A}{x + h} - (\Ax{x + h})^{2} \right]	\\
	\actsymb[h]{V}{}[n]	
	&=	M \Ax{x + h} - \pi \ax**{x + h}	\\
	&\equiv	\left(M + \frac{\pi}{d}\right) \Ax{x + h} - \frac{\pi}{d}
\end{align*}

Sous le principe d'équivalence du portefeuille (PEP):
\begin{align*}
	\Vx[h]{}[n]
	&\overset{PEP}{=}	M\left[\frac{\Ax{x + h} - \Ax{x}}{1 - \Ax{x}}\right]	\\
	&\overset{PEP}{=}	M\left[1 - \frac{\ax**{x + h}}{\ax**{x}}\right]	
\end{align*}
\end{examplebox}

\begin{examplebox}{Notation pour un contrat d'assurance avec primes non-nivelées}
\begin{align*}
	\LVx{h}
	&=	b_{\color{burntsienna}K_{x + h} + h + 1} v^{\color{orange}K_{x + h} + 1} - \sumz{K_{x + h}}{i = 0} \pi_{i + h} v^{i}	\\
	\Vx[h]{}[n]
	&=	\sumz{\infty}{j = 0} b_{\textcolor{teal}{j} + h + 1} v^{\textcolor{teal}{j} + 1} \px[\textcolor{teal}{j}]{x + h} \qx{x + h + \textcolor{teal}{j}} - \sumz{\infty}{j = 0} \pi_{i + h} v^{i} \px[\textcolor{teal}{j}]{x + h}
\end{align*}
	\textbf{Note}
\begin{itemize}[leftmargin = *]
	\item	La prestation $b$ est payable au moment ${\color{burntsienna}K_{x + h} + h + 1}$. 
	\item	Cependant, puisqu'on évalue la perte au temps $h$, il y a seulement ${\color{orange}K_{x + h} + 1}$ années à actualiser.
\end{itemize}
\end{examplebox}

\subsection*{Calcul de réserves}
\begin{conceptgen}{Méthodes d'évaluation de la réserve}
\setlength{\mathindent}{-1.5cm}
\begin{minipage}[t]{0.5\columnwidth}
\begin{center}
	\textbf{Prospective}
\end{center}
\begin{align*}
	\actsymb[h]{V}{}[g]
	&=	VPA_{@t}\left(\shortstack{prestations futures\\ à payer}\right)\\	 
		&+	VPA_{@t}\left(\shortstack{frais futurs\\ à payer}\right)\\ 
		&- 	VPA_{@t}\left(\shortstack{primes futures\\ à recevoir}\right)
\end{align*}
\end{minipage}
\setlength{\mathindent}{-0.5cm}
\begin{minipage}[t]{0.5\columnwidth}
\begin{center}
	\textbf{Rétrospective}
\end{center}
\begin{align*}
	\actsymb[h]{V}{}[g]
	&=	\frac{\actsymb[0]{V}{}[g]}{\Ex[h]{x}}	\\
		&+ 	\frac{VPA_{@\textcolor{orange}{0}}\left(\shortstack{primes recues\\ avant $h$}\right)}{\Ex[h]{x}}	\\
		&- 	\frac{VPA_{@\textcolor{orange}{0}}\left(\shortstack{prestations à payer\\ avant $h$}\right)}{\Ex[h]{x}}
\end{align*}
\end{minipage}
\setlength{\mathindent}{1cm}

\tcbline

Exemple pour un contrat d'assurance vie mixte $n$ années:
\begin{description}
	\item[Méthode prospective]	$\actsymb[h]{V}{}[n]	=	M \Ax{x + h: \angl{n - h}} - P \ax**{x + h: \angl{n - h}}$
	\item[Méthode rétrospective]	$\actsymb[h]{V}{}[n]	=	0 + \frac{P \ax**{x:\angl{h}} - M\Ax{\nthtop{\color{orange}\textbf{1}}{x}:\angl{h}}}{\Ex[h]{x}}$
	\item	L'assurance mixte devient une temporaire puisque la méthode rétrospective considère seulement les prestations à payer {\color{orange}\textbf{avant}} $h$.
\end{description}

\end{conceptgen}

\textbf{Relation:} $\{T_{x} - t | T_{x} > t\} \overset{d}{=} T_{x + t}$ où $\overset{d}{=}$ veut dire égale en distribution.

\subsubsection*{Relation récursive pour les réserves (discrètes)}

\begin{align*}
	\Vx[h]{}[n]
	&=	\left[
			\textcolor{ao(english)}{\px{x + h}\Vx[h + 1]{}[n]} + 
			\textcolor{amethyst}{\qx{x + h} b_{h + 1}}
		\right]\textcolor{orange}{v} - 
		\textcolor{burntorange}{\pi_{h}}	\\
	\Vx[h]{}[g]
	&=	\left[
			\textcolor{ao(english)}{\px{x + h}\Vx[h + 1]{}[n]} + 
			\textcolor{amethyst}{\qx{x + h} (b_{h + 1} + E_{h + 1})}
		\right]\textcolor{orange}{v} - 
		\textcolor{burntorange}{(G_{h} - e_{h})}
\end{align*}

La réserve pour l'année $h$ est composée de:
\begin{itemize}
	\item	\textcolor{ao(english)}{La réserve au temps $h + 1$ si l'assuré survie l'année $h$} et
	\item	\textcolor{amethyst}{la prestation payable (et frais encourus) à $h + 1$ si l'assuré décède lors de l'année $h$},
	\item	\textcolor{orange}{actualisés de $h + 1$ à $h$},
	\item	\textcolor{burntorange}{moins la prime (plus les frais) reçus de l'assuré au début de l'année $h$}.
\end{itemize}
où
\begin{description}
	\item[$G_h$]	La prime \textit{(gross premium)} à recevoir à $t = h$;
	\item[$e_h$]	Les frais reliés à la collecte de la prime \textit{(per premium expenses)};
	\item[$E_h$]	Les frais reliés au paiement de la prestation \textit{(settlement expenses)}.
\end{description}

Avec la réserve pour l'année $h + 1$ isolée :
\begin{align*}
	\Vx[h + 1]{}[g]
	&= 	\frac{(\Vx[h]{}[g] + G_h - e_h)(1+i) - (b_{h+1} + E_{h+1}) \qx[]{x+h}[]}{\px[]{x+h}[]}
\end{align*}

Avec le montant net au risque réserve pour l'année $h + 1$ isolé :
\begin{align*}
	\underbrace{(b_{h + 1} + E_{h + 1} - \Vx[h + 1]{}[g])}_{\text{montant net au risque}} \qx[]{x+h}[]
	&= 	(\Vx[h]{}[g] + G_h - e_h)(1 + i) - \Vx[h + 1]{}[g]
\end{align*}

\paragraph*{Note}	Si on a une assurance mixte dont la prestation est fonction de la réserve (e.g., $b_{k} = 1000 + \Vx[k]{}$), on commence de la fin puisqu'on sait que $\Vx[n]{} = M$.

\subsubsection*{Approximation classique pour les réserves à durées fractionnaires}
\begin{align*}
	\Vx[h + s]{}[g]
	&\approx	\left(\Vx[h]{}[g] + G_h - e_h\right)(1 - s) + \left(\Vx[h + 1]{}[g]\right)(s), \, s \in (0, 1)
\end{align*}

\subsection*{Profit de l'assureur}
\begin{distributions}[Notation]
\begin{description}
	\item[]	$N_{h}$: Nombre de contrats d'assurance vie (identiques) du portefeuille en vigueur au temps $h$.
	\item[]	$\Vx[h + 1]{}[E]$: Réserve totale pour l'année $h + 1$ du portefeuille selon l'intérêt ($i$), la mortalité ($\qx{x + h}$) et les frais ($e_{h}$ et $E_{h}$) \textbf{espérés} (\textit{\textbf{E}}xpected) pour l'année $h$.
	\item[]	$\Vx[h + 1]{}[A]$: Réserve totale pour l'année $h + 1$ du portefeuille selon l'intérêt ($i'$), la mortalité ($\qx{x + h}'$) et les frais ($e_{h}'$ et $E_{h}'$) \textbf{réellement} (\textit{\textbf{A}}ctually) encourus lors de l'année $h$.
	\item	Le profit de l'assureur pour l'année $h$ sera donc $\Vx[h + 1]{}[A] - \Vx[h + 1]{}[E]$.
\end{description}
\end{distributions}

Si uniquement $\rule{1cm}{0.15mm}$ change(nt), alors le profit sur $\rule{1cm}{0.15mm}$ pour l'année $h$ est:
\begin{description}
	\item[les frais]	$N_{h}\left[ (e_{h} - e_{h}')(1 + i) + (E_{h + 1} - E_{h + 1}')\qx{x + h} \right]$.
	\item[l'intérêt]	$N_{h} \left( \Vx[h]{}[g] + (G_{h} - e_{h}) \right) (i' - i)$.
	\item[la mortalité]	$\left( b_{h + 1} + E_{h + 1} - \Vx[h + 1]{}[g] \right)(N_{h}\qx{x + h} - N_{h}\qx{x + h}')$
\end{description}

S'il y a des différentes ordre, il suffit de remplacer les composantes par les nouvelles. \\
Par exemple: 
\begin{itemize}
	\item	Si l'ordre est frais-intérêt-mortalité, le profit sur l'intérêt devient $N_{h} \left( \Vx[h]{}[g] + (G_{h} - e_{h}') \right) (i' - i)$.
	\item	Si l'ordre est intérêt-frais-mortalité, le profit sur les frais devient $N_{h}\left[ (e_{h} - e_{h}')(1 + i') + (E_{h + 1} - E_{h + 1}')\qx{x + h} \right]$.
\end{itemize}

%%%	--------------------------------
%%%	NOTE
%%%	+	Pas vu dans le cadre du cours à l'hiver 2020.
%%%
%\subsection*{Quote-Part de l'actif (\emph{Asset shares})}
%Alors que la réserve $\actsymb[t]{V}{}$ nous dit le montant que l'assureur doit avoir de côté, la quote-part de l'actif nous indique plutôt le montant réel que l'assureur a de côté pour le contrat donné.
%\[
%	AS_{K + 1}
%	= 	\frac{(AS_{k} + G_k - e'_k)(1 + i') - (b_{k + 1} + E'_{k + 1}) \qx[]{x + k}[']}
%			 {\px[]{x + k}[']}
%\]
%%%	--------------------------------

\subsection*{Équation de Thiele}
Cette équation permet d'obtenir le \emph{taux instantané d'accroissement} de $\actsymb[t]{V}{}$.
\begin{align*}
	\derivee{t} \Vx[t]{}[g]
	&=	\textcolor{ao(english)}{\delta_t \Vx[t]{}[g]}	+	
		\textcolor{amethyst}{(G_t - e_t)} 	- 
		\textcolor{burntorange}{(b_t + E_t - \Vx[t]{}[g]) \mu_{[x] + t}}
\end{align*}
\begin{itemize}
	\item	\textcolor{ao(english)}{Applique continûment l'intérêt à la réserve au temps $t$}.
	\item	\textcolor{amethyst}{Le montant est fixe et payé au début de l'année $t$}.
	\item	\textcolor{burntorange}{Applique continûment la mortalité au montant payable pour un décès à $t$}.
\end{itemize}

on peut approximer $\Vx[h]{}[g]$ avec la \underline{Méthode d'Euler} : 
\begin{align*}
	\Vx[h]{}[g]
	&=	\frac{
		\Vx[t + h]{}[g]	- 
		h\left[ 
			(G_h - e_h)	- 
			(b_h + E_h)\mu_{[x] + h}
		\right]}
		{1 + h \delta_t + h \mu_{[x] + h}}
\end{align*}

\subsection*{Frais d'acquisition reportés}

\begin{description}
	\item	$\actsymb[h]{V}{}[e]$	Réserve pour les frais d'acquisition reportés (DAC).
		\begin{align*}
		\Vx[h]{}[e] =	DAC_{h}
		&=	VPA_{@t}\left(\shortstack{frais}\right) - VPA_{@t}\left(\shortstack{primes pour les frais futurs}\right)	\\
		&\equiv	\Vx[h]{}[g] - \Vx[h]{}[n]		
		\end{align*}
		\begin{itemize}[leftmargin = *]
		\item	\og \textit{expense reserve} \fg{} ou \og \textit{Deferred Acquisition Costs} \fg{}.
		\item	Si $e_{0} > e_{h}$, c'est une réserve négative.
		\item	Si $e_{0} = e_{h}$ alors $\Vx[h]{}[g] = \Vx[h]{}[n] = 0$ et $DAC_{h} = 0$.
		\end{itemize}
	\item	$P^{g}$:	Prime nivelée pour un contrat avec des frais (alias la prime brute $G$).
	\item	$P^{n}$:	Prime nivelée pour un contrat sans frais (alias la prime nette $P$).
	\item	$P^{e}$:	Prime pour les frais (\og \textit{expense premium} \fg{}).
		\begin{align*}
		P^{e}
		&=	P^{g} - P^{n}
		\end{align*}
\end{description}

\subsection*{FTP}

 
\begin{description}
	\item	$\Vx[h]{}[FTP]$	Réserve de primes FTP.
	\item	$\pi_{0}^{FTP}$	Prime FTP pour la première année.
		\begin{align*}
		\pi_{0}^{FTP}
		&=	\actsymb[1]{P}{[x]}	
		\underset{\scriptsize{\shortstack{contrat\\ vie entière}}}{=}	bv\qx{[x]}
		\end{align*}
	\item	$\pi_{h}^{FTP}$	Prime nivelée FTP pour les $h = 1, 2, \dots$ autres années.
		\begin{align*}
		\pi_{h}^{FTP}
		&=	\actsymb{P}{[x] + 1}	
		\underset{\scriptsize{\shortstack{contrat\\ vie entière}}}{=}	b \frac{\Ax{[x] + 1}}{\ax**{[x] + 1}}
		\end{align*}
\end{description}
\begin{itemize}[leftmargin = *]
	\item	Habituellement, il y a plus de frais au temps d'acquisition;
	\item	Ces frais supplémentaire sont répartis sur la durée du contrat;
	\item	Habituellement, on utilise la prime nette pour faire les calculs puisque c'est plus simple;
	\item	Lorsqu'on établit l'équation pour la perte, utilisée les frais et la prime applicables à partir de la deuxième année et soustraire la différence pour la première;
	\item	\textbf{Note}:	Lorsqu'on calcule la réserve FTP $\Vx[h]{}[\text{FTP}]$ on n'a pas besoin de calculer $\pi_{0}^{\text{FTP}}$, on y va directement avec $\pi_{h}^{\text{FTP}}$.
\end{itemize}

\newpage

\section{Modèles à plusieurs états}


\begin{description}
	\item[$_{k}Q_{t}^{(i, j)}$]	Probabilité de transition de l'état $i$ au temps $t$ à l'état $j$ au temps $t + k$.
		\begin{itemize}[leftmargin = *]
		\item	De façon équivalente, $\px[k]{x + t}^{ij}$.
		\end{itemize}
	\item[$M_{t}$]	État au temps $t$ parmi les $\{1, 2, \dots, r\}$ ou $\{0, 1, \dots, r\}$ états.
		\begin{itemize}[leftmargin = *]
		\item	De façon équivalente, $M(t)$.
		\item	Le processus $M_{t}$ est une "Chaine de Markov" ssi $\forall t = 0, 1, 2, \dots$:
		\begin{align*}
		Q_{t}^{(i, j)}
		&=	\Pr(M_{t + 1} = j | M_{t} = i, M_{t - 1}, \dots, M_{0})	\\
		&=	\Pr(M_{t + 1} = j | M_{t} = i)	
		\end{align*}
		\end{itemize}
	\item[$\bm{Q}_{t}$]	Matrice des probabilités de transition.
		\begin{itemize}[leftmargin = *]
		\item	Les transitions sont en fin d'année.
		\item	Si la matrice :
			\begin{description}
			\item[dépend du temps]	alors $M_{t}$ est une chaîne de Markov \textbf{non-homogène}.
			\item[ne dépend pas du temps]	alors $M_{t}$ est une chaîne de Markov \textbf{homogène}.
			\item	Également, dans ce cas-ci, on dénote $\bm{Q}_{t}$ par $\bm{Q}$ puisque $Q_{t}^{ij} = Q^{ij} \, \forall t \geq 0$
			\end{description}
		\end{itemize}
	\item[$_{k}\bm{Q}_{t}$]	Matrice de $k$-étapes des probabilités de transition.
\end{description}

\begin{align*}
	\actsymb[m + n]{Q}{t}[(i, j)]
	&=	\sumz{r}{k = 1}\actsymb[m]{Q}{t}[(i, k)] \actsymb[n]{Q}{t + m}[(k, j)]
\end{align*}

\columnbreak

\subsection*{En temps continu}
On généralise la notation utilisée auparavant (le \textit{modèle actif-décédé}) pour des modèles à plusieurs états.

\subsubsection*{Notation et hypothèses}
\begin{description}
	\item[$Y_{x}(t)$]	Processus stochastique $\{Y(s); s \ge 0\}$ de l'état dont les transitions peuvent se produire à n'importe quel moment $t \ge 0$ et donc pas seulement en fin d'année;
		\begin{itemize}[leftmargin = *]
		\item	De façon équivalente, $Y(x + t)$;
		\item	$Y_{x}(t) = i$ pour un assuré d'âge $(x)$ dans l'état $i$ au temps $t$ (ou, de façon équivalente, à l'âge $x + t$).
		\end{itemize}
	\item	$\px[k]{x + t}^{ij}$	Probabilité qu'un individu d'âge $x$ dans l'état $i$ au temps $t$ soit dans l'état $j$ (où $j$ peut être égale à $i$) au temps $t + k$.
		\begin{align*}
		\px[k]{x + t}^{ij}
		&=	\Pr(Y_{x}(t) = j | Y_{x} = i)
		\end{align*}
	\item	$\px[k]{x + t}^{\overline{ii}}$	Probabilité qu'un individu d'âge $x$ dans l'état $i$ au temps $t$ reste dans dans l'état $i$ continument jusqu'au temps $t + k$.
		\begin{align*}
		\px[k]{x + t}^{\overline{ij}}
		&=	\Pr(Y_{x}(s) = i, \underbrace{\forall s \in [0, t]}_{\text{sans sortir et revenir}} | Y_{x} = i)
		\end{align*}
		\begin{itemize}[leftmargin = *]
			\item	Il s'ensuit que $\px[k]{x + t}^{ij} \geq \px[k]{x + t}^{\overline{ij}}$ car:
				\begin{align*}
				\px[k]{x + t}^{ij}
				&=	\px[k]{x + t}^{\overline{ij}} + \Pr(Y_{x}(t) = i, \text{après avoir sorti et revenu} | Y_{x} = i)	\\
				\end{align*}
		\end{itemize}
	\item[$\mu_{x}^{ij}$]	\textbf{Force de transition} de l'état $i$ à l'état $j$ (\underline{$i \neq j$}) pour un assuré d'âge $(x)$.
		\begin{align*}
		\mu_{x}^{ij}
		&=	\limz{h}{0^{+}} \frac{\px[h]{x}[ij]}{h}, \: i \neq j
		\end{align*}
		\begin{itemize}[leftmargin = *]
		\item	On trouve que \icbox[red][palechestnut]{pour $i \neq j$}:
			\begin{align*}
			\px[h]{x}[ij]	&=	h\mu_{x}^{ij} + o(h)		&
			&\Rightarrow		\px[h]{x}[ij]	&\approx	h\mu_{x}^{ij}, \\
			&\text{où } h > 0 \text{ est très petit.}
			\end{align*}
		\end{itemize}
\end{description}

\begin{rappel}{Hypothèses du modèle à plusieurs états} 
\begin{enumerate}
	\item	Le processus stochastique $Y_{t}$ est une chaîne de Markov.
		\begin{align*}
		\Pr(Y_{t + s} = j | Y_{t} = i, Y_{u}, 0 \leq u < 1) &= \Pr(Y_{t + s} = j | Y_{t} = i)
		\end{align*}
	\item	Pour tout intervalle de longueur $h$,
		\begin{align*}
		\Pr\left( \shortstack{2, ou plus, transitions\\ pendant une période de longueur $h$} \right)	&=	o(h)	
		\end{align*}
		\begin{itemize}[leftmargin = *]
		\item[\textbf{Note}]	Une fonction $g \in o(h)$ si $\underset{h \rightarrow 0}{\lim} \frac{g(h)}{h} = 0$.
		\end{itemize}
	\item	Pour tous les états $i$ et $j$, et tout âge $x \geq 0$, $\px[t]{x}[ij]$ est différentiable par rapport à $t$.
\end{enumerate}
\begin{itemize}[leftmargin = *]
\item	Cette hypothèse veille au bon déroulement mathématique en assurant:
	\begin{itemize}[leftmargin = *]
	\item	L'existence de la limite dans la définition de $\mu_{x}^{ij}$;
	\item	Que la probabilité d'une transition dans un intervalle de longueur $h$ tend vers 0 lorsque $h$ tends vers 0.
	\end{itemize}
\end{itemize}
\end{rappel}

\begin{conceptgen}{Remarques}
\begin{enumerate}[leftmargin = *]
	\item	\icbox{$\px[h]{x}[ii] = \px[h]{x}[\overline{ij}] + o(h)$} où $o(h)$ est la probabilité de sortir et revenir de l'hypothèse 2.
	\item	
	\begin{align*}
		\px[h]{x}[ij] 
		\geq \px[h]{x}[\overline{ii}] 	
		&= 1 - \sumz{n}{j = 0, j \neq i} \px[h]{x}[ij] + o(h) 	\\
		&\equiv 1 - h \sumz{n}{j = 0, j \neq i} \mu_{x}^{ij} + o(h)
	\end{align*}
\end{enumerate}
\end{conceptgen}

\subsubsection*{Formules}
Nous pouvons exprimer toutes les probabilités en fonction des forces de transitions. 
%On peut relier directement les forces de transition aux probabilités $\px[t]{x}[\overline{ii}]$ et $\px[t]{x}[ii]$. La probabilité que l'assuré d'âge $x$ présentement dans l'état $i$ restera dans l'état $i$ sans jamais en sortir est:
\textbf{Approche directe}:
\begin{align*}
	\px[t]{x}[\overline{ii}]
	&=	\text{exp}\left\{	-\int_{0}^{t}\sumz{n}{j = 0, j \neq i} \mu_{x + s}^{ij} ds\right\}
\end{align*}

Transition d'un état au prochain pour un \textbf{modèle d'invalidité permanente}:
\begin{align*}
	\px[u]{x}[01] 
	&=	\int_{t = 0}^{u} (\px[t]{x}[\overline{00}]) \textcolor{teal}{(\mu_{x + t}^{01})} (\px[u - t]{x + t}[\overline{11}]) \textcolor{teal}{dt}
	\approx	\int_{t = 0}^{u} (\px[t]{x}[\overline{00}]) (\px[u - t]{x + t}[\overline{11}]) \textcolor{teal}{(\px[dt]{x + t}[01])}
\end{align*}

Transition d'un état à un état supérieur:
\begin{align*}
	\px[u]{x}[02] 
	&=	\int_{t = 0}^{u} \left\{(\px[t]{x}[\overline{00}]) (\mu_{x + t}^{01}) (\px[u - t]{x + t}[12])\right\} + \left\{(\px[t]{x}[\overline{00}]) (\mu_{x + t}^{02}) (\px[u - t]{x + t}[\overline{22}])\right\} dt	\\
	&=	1 - \px[u]{x}[\overline{00}] - \px[u]{x}[01]
\end{align*}

\subsubsection*{Approximations}
Pour les modèles où il est possible de sortir et de revenir à un état.

\begin{conceptgen}{Kolmogorov's Forward Equations}
\begin{align*}
	\derivee{t}{} \px[t]{x}[ij]
	&=	\sumz{n}{k = 0, k \neq j} \left( \px[t]{x}[ik] \mu_{x + t}^{kj} - \px[t]{x}[ij] \mu_{x + t}^{jk}  \right)
\end{align*}

Avec la \textbf{notation} $\mu_{x}^{ii} = - \sumz{n}{k = 0, k \neq i} \mu_{x}^{ik}$ on a:
\begin{align*}
	\derivee{t}{} \px[t]{x}[ij]
	&=	\sumz{n}{k = 0} \px[t]{x}[ik] \mu_{x + t}^{kj}
\end{align*}
où $\mu_{x}$ n'est \underline{plus une force de transition} mais \textbf{représente plutôt une notation} pour simplifier l'expression.
\end{conceptgen}

On peut récrire l'expression en forme matricielle:
	\setlength{\mathindent}{-1cm}
\begin{align*}
	\derivee{t}{} \actsymb[t]{\bm{P}}{x}
	&=	\actsymb[t]{\bm{P}}{x} \actsymb{\bm{P}}{x + t}	\\
	\derivee{t}{}
	\begin{pmatrix}
	\px[t]{x}[00]	&	\px[t]{x}[01]	&	\dots	&	\px[t]{x}[0n]	\\
	\px[t]{x}[10]	&	\px[t]{x}[11]	&	\dots	&	\px[t]{x}[1n]	\\
	\vdots			&	\vdots			&	\ddots	&	\vdots	\\
	\px[t]{x}[n0]	&	\px[t]{x}[n1]	&	\dots	&	\px[t]{x}[nn]	
	\end{pmatrix}
	&=	
	\begin{pmatrix}
	\px[t]{x}[00]	&	\px[t]{x}[01]	&	\dots	&	\px[t]{x}[0n]	\\
	\px[t]{x}[10]	&	\px[t]{x}[11]	&	\dots	&	\px[t]{x}[1n]	\\
	\vdots			&	\vdots			&	\ddots	&	\vdots	\\
	\px[t]{x}[n0]	&	\px[t]{x}[n1]	&	\dots	&	\px[t]{x}[nn]	
	\end{pmatrix}
	\begin{pmatrix}
	\mu_{x + t}^{00}	&	\mu_{x + t}^{01}	&	\dots	&	\mu_{x + t}^{0n}	\\
	\mu_{x + t}^{10}	&	\mu_{x + t}^{11}	&	\dots	&	\mu_{x + t}^{1n}	\\
	\vdots			&	\vdots			&	\ddots	&	\vdots	\\
	\mu_{x + t}^{n0}	&	\mu_{x + t}^{n1}	&	\dots	&	\mu_{x + t}^{nn}	
	\end{pmatrix}
\end{align*}
	\setlength{\mathindent}{1cm}
	
\begin{conceptgen}{Méthode d'Euler}
Pour $h > 0$ très petit, on a: 
\begin{align*}
	\derivee{t}{} \px[t]{x}[ij]
	&\approx	\frac{(\px[t + h]{x}[ij] - \px[t]{x}[ij])}{h}
\end{align*}
avec la condition initiale $\forall i \neq j$:
\begin{align*}
	\px[0]{x}[ii]	&=	1	&	&\text{et}	&	\px[0]{x}[ij]	&=	0
\end{align*}
\end{conceptgen}

\subsubsection*{Paiements}
\begin{description}
	\item	$\ax**{x:\angln}[\overline{ii}]$	La VPA d'une rente temporaire payant $1$\$ à une vie dans l'état $i$ seulement lorsqu'elle est dans l'état $i$.
		\begin{align*}
		\ax**{x:\angln}[\overline{ii}]
		&=	\int_{0}^{n} v^{t} \px[t]{x}[\overline{ii}] dt
		\end{align*}
	\item	On a aussi plus généralement:
\begin{align*}
	\ax**{x:\angln}[ij]
	&=	\int_{0}^{n} v^{t} \px[t]{x}[ij] dt	\\
	\Ax*{x}[ij]
	&=	\int_{0}^{\infty} \sumz{}{k \neq j}	v^{t} \px[t]{x}[ik] \mu_{x + t}^{kj} dt
\end{align*}
\end{description}



%\columnbreak

\subsection*{Modèle à plusieurs décroissances}
\begin{itemize}[leftmargin = *]
	\item	en anglais, \og \textit{Multiple Decrement Model} \fg{};
	\item	Précédemment, il y avait résiliation du contrat uniquement en raison d'un décès;
	\item	Cependant, on généralise pour évaluer les primes et réserves de contrats dont les prestations diffèrent en fonction des causes de décroissances;
	\item	Ces modèles sont en fait des cas particuliers des chaînes de Markov.
\end{itemize}

\begin{description}
	\item[$T_{x}$]	Temps de décroissance de $x$ (alias, la \textit{durée de vie} résiduelle de $x$);
	\item[$J$]	Cause de la décroissance;
		\begin{itemize}[leftmargin = *]
		\item	Variable aléatoire discrète avec $J \in \{1, 2, \dots, m\}$ où $m$ est le nombre de causes possibles de décroissance.
		\end{itemize}
	\item	$\qx[t]{x}[(j)]$	Probabilité de décroissance d'ici $t$ années pour un assuré d'âge $x$ en raison de la $j$e cause;
		\begin{align*}
		\px[t]{x}[0j]
		&=	\qx[t]{x}[(j)]
		=	\Pr(T_{x} \leq t, J = j)
		\end{align*}
		\begin{itemize}[leftmargin = *]
		\item	Il s'ensuit de l'équation que $\qx[t]{x}[(j)]$ est une distribution conjointe de $T_{x}$ et $J$.
		\end{itemize}
	\item	$\qx[t]{x}[(\tau)]$	Probabilité de décroissance d'ici $t$ années pour un assuré d'âge $x$ peu importe la cause;
		\begin{align*}
		\qx[t]{x}[(\tau)]
		&=	\Pr(T_{x} \leq t)		\\
		&=	\sumz{m}{j = 1} \Pr(T_{x} \leq t, J = j)	
		=	\sumz{m}{j = 1} \qx[t]{x}[(j)]	
		\end{align*}
		\begin{itemize}[leftmargin = *]
		\item	En parallèle, \icbox{$\px[t]{x}[(\tau)]$} est la probabilité de survivre pendant $t$ années à toutes les causes de décroissance;
		\item	Cependant, $\px[t]{x}[(j)]$ \textbf{n'existe pas} et $\px[t]{x}[(j)] \neq 1 - \qx[t]{x}[(j)]$.
		\end{itemize}
\end{description}

\subsubsection*{Fonctions de densité}
\begin{align*}
	f_{T_{x}, J}(t, j)
%	&=	\derivee{t}{} (\qx[t]{x}[(j)])	\\
	&=	(\px[t]{x}[(\tau)])(\mu_{x + t}^{(j)})	\\
%	\therefore
%	f_{T_{x}}(t)
%	&=	\sumz{m}{j = 1}f_{T_{x}, J}(t, j)	\\
%	=	\derivee{t}{} (\qx[t]{x}[(\tau)])	\\
	f_{J}(j)
	&=	\int_{0}^{\infty} f_{T_{x}, J}(t, j) dt	
	=	\qx[\infty]{x}[(j)]	\\
	\text{E}[T_{x}]
	&=	\int_{0}^{\infty} \px[t]{x}[(\tau)]dt	\\
	f_{J | T_{x}}(J | t)
	&=	\frac{f_{T_{x}, J}(t, j)}{f_{T_{x}}(t)}
	\equiv	\frac{\mu_{x +  t}^{(j)}}{\mu_{x +  t}^{(\tau)}}
\end{align*}

\subsubsection*{Force de décroissance totale}
\begin{align*}
	\mu_{x + t}^{(\tau)}
	&=	\sumz{m}{j = 1} \mu_{x + t}^{(j)}
\end{align*}

\begin{align*}
	f_{T_{x}}(t)
	&=	\sumz{m}{j = 1}f_{T_{x}, J}(t, j)	
	=	(\px[t]{x}[(\tau)])(\mu_{x + t}^{(\tau)})
\end{align*}
\begin{align*}
	\qx[t]{x}[(\tau)]
%	&=	\int_{0}^{t} f_{T_{x}}(s, j)ds
%	=	\int_{0}^{t} \px[s]{x}[(\tau)] \mu_{x + s}^{(\tau)}ds
	&=	\sumz{m}{j = 1} \qx[t]{x}[(j)]
\end{align*}

De plus:
\begin{align*}
	\px[t]{x}[(\tau)]
%	&=	S_{T_{x}}(t)
	=	\textrm{e}^{-\int_{0}^{t} \mu_{x + s}^{(\tau)}ds}
\end{align*}

\subsubsection*{Force de décroissance de la $j^{\text{e}}$ cause}
\begin{description}
	\item[$\mu_{x + t}^{(j)}$]	Force de décroissance de la $j^{\text{e}}$ cause pour un assuré d'âge $x$.
\end{description}
\begin{align*}
	\qx[t]{x}[(j)]
	&=	\int_{0}^{t} f_{T_{x}, J}(s, j)ds
	=	\int_{0}^{t} \px[s]{x}[(\tau)] \mu_{x + s}^{(j)}ds
\end{align*}

\subsubsection*{Incorporation de $K_{x}$}
\begin{align*}
	\qx[k|]{x}[(j)]
	&=	\Pr(k \leq T_{x} < k + 1, J = j)	\\
	&=	\int_{k}^{k + 1} f_{T_{x}, J}(t, j)dt
	\equiv	\int_{0}^{k + 1} f_{T_{x}, J}(t, j)dt - \int_{0}^{k} f_{T_{x}, J}(t, j)dt	\\
	&=	\qx[k + 1]{x}[(j)] - \qx[k]{x}[(j)]
\end{align*}

Aussi
\begin{align*}
	\qx[k|]{x}[(j)]
	&=	\int_{k}^{k + 1} \px[t]{x}[(\tau)] \mu_{x + t}^{(j)}dt
	=	\px[k]{x}[(\tau)] \qx{x + k}[(j)]
\end{align*}
\paragraph{Note}	Développer cette expression s'il y a un manque d'information sur des $\ell{x}$; voir l'exercice 2.10 du cours 8 pour un exemple.

Loi marginale:
\begin{align*}
	\qx[k|]{x}[(\tau)]
	&=	\Pr(K_{x} = k)
	=	\sumz{m}{j = 1} \Pr(K_{x} = k, J = j)
	=	\sumz{m}{j = 1} \qx[k|]{x}[(j)]	\\
	&\equiv	\qx[k + 1]{x}[(\tau)] - \qx[k]{x}[(\tau)]
	=	\px[k]{x}[(\tau)] - \px[k + 1]{x}[(\tau)]	\\
	&\equiv \px[k]{x}[(\tau)] \qx{x + k}[(\tau)]
\end{align*}

\subsubsection*{Tables de mortalité}
\begin{description}
	\item[]	$\dx[r]{x}$	Nombre de décès d'une cohorte de $\ell{x}$ personnes entre les temps 0 et $r$ (alias, entre les âges $x$ et $x + r$);
	\item[]	$\dx[r]{x}[(j)]$	Nombre de décès d'une cohorte de $\ell{x}$ personnes entre les temps 0 et $r$ en cause la décroissance $j$.
\end{description}

\paragraph{Note}	Pour des paiements selon l'état, voir l'exercice 2.12 à la fin du cours 8. 

Si $\mu_{x + t}^{(1)}, \dots, \mu_{x + t}^{(m)}$ sont des constantes $\mu^{(1)}, \dots, \mu^{(m)}$ alors:
\begin{align*}
	\frac{\mu_{x + t}^{(j)}}{\mu_{x + t}^{(\tau)}}
	&=	\frac{\mu_{x + t}^{(j)}}{\mu_{x + t}^{(1)} + \dots + \mu_{x + t}^{(m)}}	
	=	k	
	=	\text{constante}	\\
	\Rightarrow	\qx[t]{x}[(j)]
	&=	k \qx[t]{x}[(\tau)]
\end{align*}

\columnbreak

\subsection*{Modèles à décroissance unique associés}
\begin{description}
	\item[$T_{x}'^{(j)}$]	Temps de décroissance de $x$ (alias, la \textit{durée de vie} résiduelle de $x$) \underline{\textbf{en supposant}} qu'il est uniquement exposé à la cause $j$;
		\begin{itemize}[leftmargin = *]
		\item	C'est une durée de vie théorique, mais utile;
		\item	Comme il y a un seul type de décès, c'est le modèle actif-décédé de vie I;
		\item	Généralement, on suppose que les décroissances $T_{x}'^{(j)}$ pour $j = 1, 2, \dots, m$ sont indépendantes;
		\item	Avec l'indépendance, on trouve que la distribution de $T_{x}$ est la même que la première cause de décès \icbox{$T_{x}	\overset{d}{=}	\min\left\{T_{x}'^{(1)}, T_{x}'^{(2)}, \dots, T_{x}'^{(m)}\right\}$}.
		\end{itemize}
	\item[]	$\qx[t]{x}'^{(j)}$:	Probabilité de décroissance d'ici $t$ années pour un assuré d'âge $x$ en raison de la $j$e cause \underline{\textbf{en supposant}} qu'il est uniquement exposé à la cause $j$;
		\begin{itemize}[leftmargin = *]
		\item	Puisque c'est le modèle actif-décédé, il s'ensuit que \icbox{$\px[t]{x}'^{(j)}	=	1	-	\qx[t]{x}'^{(j)}$}.
		\end{itemize}
\end{description}

On peut relier les 2 modèles:
\begin{align*}
	\mu_{x + s}^{(j)}
	&=	\mu_{x + s}'^{(j)}	\\
	\px[t]{x}[(\tau)]
	&=	\underset{j = 1}{\overset{m}{\prod}} \px[t]{x}'^{(j)}
\end{align*}
\begin{itemize}[leftmargin = *]
	\item	 La multiplication des $\px[t]{x}$ ci-dessus illustre le lien avec la fonction de survie du minimum.
\end{itemize}

Donc:
\begin{align*}
	\qx[t]{x}[(\tau)]
	&=	\qx[t]{x}[(1)]	+	\qx[t]{x}[(2)]	+	\dots	+	\qx[t]{x}[(m	)]	&
	\px[t]{x}[(\tau)]
	&=	\px[t]{x}'^{(1)}	\times	\px[t]{x}'^{(2)}	\times	\dots	\times	\px[t]{x}'^{(m)}
\end{align*}

Il s'ensuit que \icbox[red][palechestnut]{$\px[t]{x}[(\tau)] \leq \px[t]{x}'^{(j)}, \forall j = 1, 2, \dots, m$} et \icbox[red][palechestnut]{$\qx[t]{x}'^{(j)} \geq \qx[t]{x}[(j)], \forall j = 1, 2, \dots, m$}.

\subsection{Interrelations}
\begin{rappel_enhanced}[Hypothèses]
Pour $x \in \mathbb{Z}^{+}, t \in [0, 1], j = 1, 2, \dots, m$,
\begin{description}
	\item[DUD]	$\qx[t]{x}'^{(j)} = t \times \qx[]{x}'^{(j)}$;
	\item[FC]	$\mu_{x + t}^{(j)} = \mu_{x}^{(j)}$.
\end{description}

\begin{itemize}[leftmargin = *]
	\item	Si les mortalités $T_{x}'^{(j)}$ suivent des lois DeMoivre, alors DUD est exact.
\end{itemize}
\end{rappel_enhanced}

\begin{conceptgen}{Trouver $q_{x}^{(j)}$ de $q_{x}'^{(j)}$}
Sachant $\qx[]{x}'^{(1)}, \dots, \qx[]{x}'^{(m)}$,
\begin{enumerate}
	\item	Poser une hypothèse (DUD ou FC) sur les décroissances uniques $T_{x}'^{(j)}$ pour trouver $\qx[s]{x}'^{(j)}$;
	\item	Trouver $\mu_{x + s}'^{(j)}	=	\mu_{x + s}^{(j)}$;
	\item	Trouver $\px[s]{x}^{(\tau)}	=	\sumz{m}{j = 1}\mu_{x + s}^{(j)}$;
	\item	Trouver $\qx[t]{x}^{(j)}	=	\int_{0}^{t}\px[s]{x}^{(\tau)}\mu_{x + s}^{(j)}ds$.
\end{enumerate}

\begin{distributions}[Sous DUD]
\begin{align*}
	\qx{x}[(j)]
	&=	\qx{x}'^{(j)}	\int_{0}^{t} \left[ \underset{k \neq j, k = 1}{\overset{m}{\prod}}\left(1 - t \cdot \qx{x}'^{(k)}\right) \right] dt
\end{align*}

\tcbline

\textbf{Cas particuliers pour $t = 1$}:
\begin{itemize}[leftmargin = *]
	\item	Si $m = 2$, \icbox{$\qx{x}[(1)]	=	\qx{x}'^{(1)} \left( 1 - \frac{\qx{x}'^{(2)}}{2} \right)$} et \icbox{$\qx{x}[(2)]	=	\qx{x}'^{(2)} \left( 1 - \frac{\qx{x}'^{(1)}}{2} \right)$};
	\item	Si $m = 3$, \icbox{$\qx{x}[(1)]	=	\qx{x}'^{(1)} \left[ 1 - \frac{1}{2}\left( \qx{x}'^{(2)} + \qx{x}'^{(3)} \right) + \frac{1}{3} \left( \qx{x}'^{(2)} \qx{x}'^{(3)} \right) \right]$}.
\end{itemize}
\end{distributions}

\begin{distributions}[Sous FC]
\begin{align*}
	\qx[\color{blue(pigment)}t]{x}[(j)]
	&=	\frac{\ln(\px{x}'^{(j)})}{\ln(\px{x}[(\tau)])}\qx[\color{blue(pigment)}t]{x}[(\tau)] 
\end{align*}
\end{distributions}
\end{conceptgen}

\begin{conceptgen}{Trouver $_{t}q_{x}'^{(j)}$ de $_{t}q_{x}^{(j)}$}
Sachant $\qx[t]{x}[(1)], \dots, \qx[t]{x}[(m)]$,
\begin{enumerate}
	\item	Poser une hypothèse (DUD ou FC) sur les décroissances ($T_{x}^{(j)}$) pour trouver $\qx[t]{x}^{(j)}$;
	\item	Trouver $\mu_{x + t}^{(j)}	=	\frac{\deriv{t}{\qx[t]{x}[(j)]}}{\px[t]{x}[(\tau)]}	=	\mu_{x + t}'^{(j)}$;
	\item	Trouver $\px[t]{x}'^{(j)}	=	\textrm{e}^{\int_{0}^{t}\mu_{x + s}'^{(j)}ds}$.
\end{enumerate}

\begin{distributions}[Sous DUD]
\begin{align*}
	\qx[t]{x}'^{(j)}
	&=	1 - \left( 1 - t \times \qx{x}[(\tau)] \right)^{\qx{x}[(j)]/\qx{x}[(\tau)]}	\\
	\px[\color{blue(pigment)}t]{x}'^{(j)}
	&=	\left( \px[\color{blue(pigment)}t]{x}[(\tau)] \right)^{\qx{x}[(j)]/\qx{x}[(\tau)]}	
\end{align*}
\end{distributions}

\begin{distributions}[Sous FC]
\begin{align*}
	\px[\color{blue(pigment)}t]{x}'^{(j)}
	&=	\left( \px[\color{blue(pigment)}t]{x}[(\tau)] \right)^{\qx{x}[(j)]/\qx{x}[(\tau)]}	
\end{align*}
\end{distributions}
\end{conceptgen}

\end{multicols*}
\end{document}

