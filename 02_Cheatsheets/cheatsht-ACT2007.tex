\documentclass[10pt, french]{article}

%% -----------------------------
%% Préambule
%% -----------------------------
% !TEX encoding = UTF-8 Unicode
% LaTeX Preamble for all cheatsheets
% Author : Gabriel Crépeault-Cauchon

% HOW-TO : copy-paste this file in the same directory as your .tex file, and add in your preamble the next command right after you have specified your documentclass : 
% \input{preamble-cheatsht.tex}
% ---------------------------------------------
% ---------------------------------------------

% Extra note : this preamble creates document that are meant to be used inside the multicols environment. See the documentation on internet for further information.

%% -----------------------------
%% Encoding packages
%% -----------------------------
\usepackage[utf8]{inputenc}
\usepackage[T1]{fontenc}
\usepackage{babel}
\usepackage{lmodern}

%% -----------------------------
%% Variable definition
%% -----------------------------
\def\auteur{Gabriel Crépeault-Cauchon / Nicholas Langevin}
\def\BackgroundColor{white}

%% -----------------------------
%% Margin and layout
%% -----------------------------
% Determine the margin for cheatsheet
\usepackage[landscape, hmargin=1cm, vmargin=1.7cm]{geometry}
\usepackage{multicol}

% Remove automatic indentation after section/subsection title.
\setlength{\parindent}{0cm}

% Save space in cheatsheet by removing space between align environment and normal text.
\usepackage{etoolbox}
\newcommand{\zerodisplayskips}{%
  \setlength{\abovedisplayskip}{0pt}%
  \setlength{\belowdisplayskip}{0pt}%
  \setlength{\abovedisplayshortskip}{0pt}%
  \setlength{\belowdisplayshortskip}{0pt}}
\appto{\normalsize}{\zerodisplayskips}
\appto{\small}{\zerodisplayskips}
\appto{\footnotesize}{\zerodisplayskips}

%% -----------------------------
%% URL and links
%% -----------------------------
\usepackage{hyperref}
\hypersetup{colorlinks = true, urlcolor = gray!70!white, linkcolor = black}

%% -----------------------------
%% Document policy (uncomment only one)
%% -----------------------------
%	\usepackage{concrete}
	\usepackage{mathpazo}
%	\usepackage{frcursive} %% permet d'écrire en lettres attachées
%	\usepackage{aeguill}
%	\usepackage{mathptmx}
%	\usepackage{fourier} 

%% -----------------------------
%% Math configuration
%% -----------------------------
\usepackage[fleqn]{amsmath}
\usepackage{amsthm,amssymb,latexsym,amsfonts}
\usepackage{empheq}
\usepackage{numprint}
\usepackage{dsfont} % Pour avoir le symbole du domaine Z

% Mathematics shortcuts

\newcommand{\reels}{\mathbb{R}}
\newcommand{\entiers}{\mathbb{Z}}
\newcommand{\naturels}{\mathbb{N}}
\newcommand{\eval}{\biggr \rvert}
\usepackage{cancel}
\newcommand{\derivee}[1]{\frac{\partial}{\partial #1}}
\newcommand{\prob}[1]{\Pr \left( #1 \right)}
\newcommand{\esp}[1]{\mathrm{E} \left[ #1 \right]} % espérance
\newcommand{\variance}[1]{\mathrm{Var} \left( #1   \right)}
\newcommand{\covar}[1]{\mathrm{Cov} \left( #1   \right)}
\newcommand{\laplace}{\mathcal{L}}
\newcommand{\deriv}[2][]{\frac{\partial^{#1}}{\partial #2^{#1}}}
\newcommand{\e}[1]{\mathrm{e}^{#1}}
\newcommand{\te}[1]{\text{exp}\left\{#1\right\}}
\DeclareMathSymbol{\shortminus}{\mathbin}{AMSa}{"39}



% To indicate equation number on a specific line in align environment
\newcommand\numberthis{\addtocounter{equation}{1}\tag{\theequation}}

%
% Actuarial notation packages
%
\usepackage{actuarialsymbol}
\usepackage{actuarialangle}

%
% Matrix notation for math symbols (\bm{•})
%
\usepackage{bm}
% Matrix notation variable (bold style)
\newcommand{\matr}[1]{\mathbf{#1}}



%% -----------------------------
%% tcolorbox configuration
%% -----------------------------
\usepackage[most]{tcolorbox}
\tcbuselibrary{xparse}
\tcbuselibrary{breakable}

%%
%% Coloured box "definition" for definitions
%%
\DeclareTColorBox{definition}{ o }				% #1 parameter
{
	colframe=blue!60!green,colback=blue!5!white, % color of the box
	breakable, 
	pad at break* = 0mm, 						% to split the box
	title = {#1},
	after title = {\large \hfill \faBook},
}
%%
%% Coloured box "definition2" for definitions
%%
\DeclareTColorBox{definitionNOHFILL}{ o }				% #1 parameter
{
	colframe=blue!60!green,colback=blue!5!white, % color of the box
	pad at break* = 0mm, 						% to split the box
	title = {#1},
	before title = {\faBook \quad },
	breakable
}


%%
%% Coloured box "algo" for algorithms
%%
\newtcolorbox{algo}[ 1 ]
{
	colback = blue!5!white,
	colframe = blue!75!black,
	title=#1,
	fonttitle = \bfseries,
	breakable
}
%%
%% Coloured box "conceptgen" for points adding to a concept's deifintion
%%
\newtcolorbox{conceptgen}[ 1 ]
{
	breakable,
	colback = beaublue,
	colframe = airforceblue,
	title=#1,
	fonttitle = \bfseries
}
%%
%% Coloured box "probch3" pour formules relatives au 3ème chapitre de prob
%%
\newtcolorbox{probch3}[ 1 ]
{
	colback = ruddypink,
	colframe = burgundy,
	fonttitle = \bfseries,	
	breakable,
	title=#1
}
%%
%% Coloured box "formula" for formulas
%%
\newtcolorbox{formula}[ 1 ]
{
	colback = green!5!white,
	colframe = green!70!black,
	breakable,
	fonttitle = \bfseries,
	title=#1
}
%%
%% Coloured box "formula" for formulas
%%
\DeclareTColorBox{algo2}{ o }
{
	enhanced,
	title = #1,
	colback=blue!5!white,	
	colbacktitle=blue!75!black,
	fonttitle = \bfseries,
	breakable,
	boxed title style={size=small,colframe=arsenic} ,
	attach boxed title to top center = {yshift=-3mm,yshifttext=-1mm},
}
%%
%% Coloured box "examplebox" for formulas
%%
\newtcolorbox{examplebox}[ 1 ]
{
	colback = lightmauve,
	colframe = antiquefuchsia,
	breakable,
	fonttitle = \bfseries,title=#1
}
%%
%% Coloured box "rappel" pour rappel de formules
%%
\newtcolorbox{rappel}[ 1 ]
{
	colback = ashgrey,
	colframe = arsenic,
	breakable,
	fonttitle = \bfseries,title=#1
}
%%
%% Coloured box "rappel" pour rappel de formules
%%
\DeclareTColorBox{rappel_enhanced}{ o }
{
	enhanced,
	title = #1,
	colback=ashgrey, % color of the box
%	colframe=blue(pigment),
%	colframe=arsenic,	
	colbacktitle=arsenic,
	fonttitle = \bfseries,
	breakable,
	boxed title style={size=small,colframe=arsenic} ,
	attach boxed title to top center = {yshift=-3mm,yshifttext=-1mm},
}
%%
%% Coloured box "notation" for notation and terminology
%%
\DeclareTColorBox{distributions}{ o }			% #1 parameter
{
	enhanced,
	title = #1,
	colback=gray(x11gray), % color of the box
%	colframe=blue(pigment),
	colframe=arsenic,	
	colbacktitle=aurometalsaurus,
	fonttitle = \bfseries,
	boxed title style={size=small,colframe=arsenic} ,
	attach boxed title to top center = {yshift=-3mm,yshifttext=-1mm},
	breakable
%	left=0pt,
%  	right=0pt,
%    box align=center,
%    ams align*
%  	top=-10pt
}

%% -----------------------------
%% Graphics and pictures
%% -----------------------------
\usepackage{graphicx}
\usepackage{pict2e}
\usepackage{tikz}

%% -----------------------------
%% insert pdf pages into document
%% -----------------------------
\usepackage{pdfpages}

%% -----------------------------
%% Color configuration
%% -----------------------------
\usepackage{color, soulutf8, colortbl}


%
%	Colour definitions
%
\definecolor{blue(munsell)}{rgb}{0.0, 0.5, 0.69}
\definecolor{blue(matcha)}{rgb}{0.596, 0.819, 1.00}
\definecolor{blue(munsell)-light}{rgb}{0.5, 0.8, 0.9}
\definecolor{bleudefrance}{rgb}{0.19, 0.55, 0.91}
\definecolor{blizzardblue}{rgb}{0.67, 0.9, 0.93}
\definecolor{bondiblue}{rgb}{0.0, 0.58, 0.71}
\definecolor{blue(pigment)}{rgb}{0.2, 0.2, 0.6}
\definecolor{bluebell}{rgb}{0.64, 0.64, 0.82}
\definecolor{airforceblue}{rgb}{0.36, 0.54, 0.66}
\definecolor{beaublue}{rgb}{0.74, 0.83, 0.9}
\definecolor{cobalt}{rgb}{0.0, 0.28, 0.67}	% nice light blue-ish
\definecolor{blue_rectangle}{RGB}{83, 84, 244}		% ACT-2004
\definecolor{indigo(web)}{rgb}{0.29, 0.0, 0.51}	% purple-ish
\definecolor{antiquefuchsia}{rgb}{0.57, 0.36, 0.51}	%	pastel dark purple ish
\definecolor{darkpastelpurple}{rgb}{0.59, 0.44, 0.84}
\definecolor{gray(x11gray)}{rgb}{0.75, 0.75, 0.75}
\definecolor{aurometalsaurus}{rgb}{0.43, 0.5, 0.5}
\definecolor{ruddypink}{rgb}{0.88, 0.56, 0.59}
\definecolor{pastelred}{rgb}{1.0, 0.41, 0.38}		
\definecolor{lightmauve}{rgb}{0.86, 0.82, 1.0}
\definecolor{azure(colorwheel)}{rgb}{0.0, 0.5, 1.0}
\definecolor{darkgreen}{rgb}{0.0, 0.2, 0.13}			
\definecolor{burntorange}{rgb}{0.8, 0.33, 0.0}		
\definecolor{burntsienna}{rgb}{0.91, 0.45, 0.32}		
\definecolor{ao(english)}{rgb}{0.0, 0.5, 0.0}		% ACT-2003
\definecolor{amber(sae/ece)}{rgb}{1.0, 0.49, 0.0} 	% ACT-2004
\definecolor{green_rectangle}{RGB}{131, 176, 84}		% ACT-2004
\definecolor{red_rectangle}{RGB}{241,112,113}		% ACT-2004
\definecolor{amethyst}{rgb}{0.6, 0.4, 0.8}
\definecolor{amethyst-light}{rgb}{0.6, 0.4, 0.8}
\definecolor{ashgrey}{rgb}{0.7, 0.75, 0.71}			% dark grey-black-ish
\definecolor{arsenic}{rgb}{0.23, 0.27, 0.29}			% light green-beige-ish gray
\definecolor{amaranth}{rgb}{0.9, 0.17, 0.31}
\definecolor{brickred}{rgb}{0.8, 0.25, 0.33}
\definecolor{pastelred}{rgb}{1.0, 0.41, 0.38}

%
% Useful shortcuts for coloured text
%
\newcommand{\orange}{\textcolor{orange}}
\newcommand{\red}{\textcolor{red}}
\newcommand{\cyan}{\textcolor{cyan}}
\newcommand{\blue}{\textcolor{blue}}
\newcommand{\green}{\textcolor{green}}
\newcommand{\purple}{\textcolor{magenta}}
\newcommand{\yellow}{\textcolor{yellow}}

%% -----------------------------
%% Enumerate environment configuration
%% -----------------------------
%
% Custum enumerate & itemize Package
%
\usepackage{enumitem}
%
% French Setup for itemize function
%
\frenchbsetup{StandardItemLabels=true}
%
% Change default label for itemize
%
\renewcommand{\labelitemi}{\faAngleRight}


%% -----------------------------
%% Tabular column type configuration
%% -----------------------------
\newcolumntype{C}{>{$}c<{$}} % math-mode version of "l" column type
\newcolumntype{L}{>{$}l<{$}} % math-mode version of "l" column type
\newcolumntype{R}{>{$}r<{$}} % math-mode version of "l" column type
\newcolumntype{f}{>{\columncolor{green!20!white}}p{1cm}}
\newcolumntype{g}{>{\columncolor{green!40!white}}m{1.2cm}}
\newcolumntype{a}{>{\columncolor{red!20!white}$}p{2cm}<{$}}	% ACT-2005
% configuration to force a line break within a single cell
\usepackage{makecell}


%% -----------------------------
%% Fontawesome for special symbols
%% -----------------------------
\usepackage{fontawesome}

%% -----------------------------
%% Section Font customization
%% -----------------------------
\usepackage{sectsty}
\sectionfont{\color{\SectionColor}}
\subsectionfont{\color{\SubSectionColor}}

%% -----------------------------
%% Footer/Header Customization
%% -----------------------------
\usepackage{lastpage}
\usepackage{fancyhdr}
\pagestyle{fancy}

%
% Header
%
\fancyhead{} 	% Reset
\fancyhead[L]{Aide-mémoire pour~ \cours ~(\textbf{\sigle})}
\fancyhead[R]{\auteur}

%
% Footer
%
\fancyfoot{}		% Reset
\fancyfoot[R]{\thepage ~de~ \pageref{LastPage}}
\fancyfoot[L]{\href{https://github.com/ressources-act/Guide_de_survie_en_actuariat}{\faGithub \ ressources-act/Guide de survie en actuariat}}
%
% Page background color
%
\pagecolor{\BackgroundColor}




%% END OF PREAMBLE
% ---------------------------------------------
% ---------------------------------------------
%% -----------------------------
%% Variable definition
%% -----------------------------
\def\cours{Mathématiques actuarielles Vie II}
\def\sigle{ACT-2007}
\def\auteur{Gabriel Crépeault-Cauchon // Nicholas Langevin // Alexandre Turcotte}
\def\SectionColor{burntorange}
\def\SubSectionColor{burntsienna}

%% Reduce margin space
\setlength{\abovedisplayskip}{-15pt}

%% -----------------------------
%% Début du document
%% -----------------------------
\begin{document}
\begin{multicols*}{2} 

\section{Calcul de réserve}

\subsection*{Notation}
\begin{itemize}
	\item[] 	$\actsymb[t]{L}{}$: Perte prospective de l'assuré au temps $t$;
		\begin{itemize}
		\item	Le symbole représente la perte pour un assuré d'âge $x$ à partir du temps $t$ et peut donc être réécrit comme:
			\begin{align*}
			\actsymb[t]{L}{}	
			&=	\{ \actsymb[t]{L}{} | T_{x} > t \}
			\end{align*}
		\end{itemize}
	\item[]	$\actsymb[t]{V}{}$: Réserve de l'assureur au temps $t$;
		\begin{itemize}
		\item	Le symbole représente la réserve pour un contrat d'assurance d'un assuré d'âge $x$ à partir du temps $t$ et peut donc être réécrit comme:
			\begin{align*}
			\actsymb[t]{V}{}	
			&=	\text{E}[\{ \actsymb[t]{L}{} | T_{x} \ge t \}]
			\end{align*}
		\end{itemize}
	\item[]	$VP_{@t}$: La valeur présente au temps $t$;
	\item[]	$VPA_{@t}$: La valeur présente actuarielle au temps $t$;
		\begin{align*}
		VPA_{@t}	
		&=	\text{E}[VP_{@t}]
		\end{align*}
\end{itemize}

\subsection*{Termes}
\begin{description}
	\item[endowment:] Mixte;
\end{description}

\subsection*{Calcul de réserves}

\paragraph{Perte prospective:} la perte prospective, $\actsymb[t]{L}{}$, actualise les transactions qui vont arriver dans le futur:
\setlength{\mathindent}{-1cm}
\begin{align*}
	\actsymb[t]{L}{} 
%	&= 	\{ \actsymb[t]{L}{} | T_x > t \} 	\\
	&= 	VP_{@t}(\text{prestations à payer}) - VP_{@t}(\text{primes à reçevoir}) \textcolor{ao(english)}{+ VP_{@t}(\text{frais à payer})} 	
%	&= 	Z - Y
\end{align*}
\setlength{\mathindent}{1cm}
\textcolor{ao(english)}{S'il y a des frais pour les contrats, il suffit de l'ajouter à la perte.}

\textbf{Relation:} $\{T_{x} - t | T_{x} > t\} \overset{d}{=} T_{x + t}$ où $\overset{d}{=}$ veut dire égale en distribution.

\paragraph{Réserve:} La réserve, $\actsymb[t]{V}{}$, est l'espérance du montant que l'assureur devra payer dans le futur---alias, l'espérance de la perte. Il y a donc plusieurs façons de calculer ces réserves mais on utilise surtout la méthode prospective.

Selon la méthode \textbf{prospective},
\setlength{\mathindent}{-1cm}
\begin{align*}
	\actsymb[t]{V}{} 
	&= 	\esp{\actsymb[t]{L}{}} \\
	&= 	VPA_{@t}(\text{prestations à payer}) - VPA_{@t}(\text{primes à reçevoir}) \\ 
	&\quad \textcolor{ao(english)}{+ VPA_{@t}(\text{frais à payer})} 	
%	= \esp{Z} - \esp{Y}
\end{align*}
\setlength{\mathindent}{1cm}

%%%	--------------------------
%%%	Pas vu dans à l'hiver 2020
%%%
%Selon la méthode \textbf{rétrospective},
%\setlength{\mathindent}{-1cm} 
%\begin{align*}
%	\actsymb[t]{V}{} 
%	&= 	\frac{\text{VPA}_{@t}(\text{primes ($\pi$) reçues avant $h$}) - \text{VPA}_{@t}(\text{prestations payables avant $h$})}{g}		
%\end{align*}
%\setlength{\mathindent}{1cm}
%%%	--------------------------

\paragraph{Remarque:} Pour des primes \textbf{nivelées} établies selon le principe d'équivalence du portefeuille, on pose $\actsymb[0]{V}{} = 0$. Donc:
\setlength{\mathindent}{-1cm}
\begin{align*}
	VPA_{@t}(\text{primes à reçevoir})
	=	VPA_{@t}(\text{prestations à payer}) \textcolor{ao(english)}{+ VPA_{@t}(\text{frais à payer})} 
\end{align*}
\setlength{\mathindent}{1cm}

\paragraph{Relation récursive pour les réserves (discrètes)}

\begin{align*}
	\textcolor{burntorange}{(\actsymb[h]{V}{} + \pi_{h})}\textcolor{burntsienna}{(1 + i)}
	&=	\textcolor{amethyst}{\qx{x + h} b_{h + 1}} + \textcolor{ao(english)}{\px{x + h} \actsymb[h + 1]{V}{}}
\end{align*}
On ajoute \textcolor{burntorange}{la prime $P$ à la réserve $\actsymb[h]{V}{}$ au temps $h$} et \textcolor{burntsienna}{accumule pour un an}. \\
Ceci est équivalent à soit \textcolor{amethyst}{décéder à l'âge $x + h$ et payer la prestation en cas de décès $b_{h + 1}$} ou \textcolor{ao(english)}{survive et ajouter à la réserve $\actsymb[h + 1]{V}{}$ au temps $x + h$}.

\textbf{Formule générale}\footnote{Si les frais ne sont pas applicables pour le problème, simplement poser $G_h = E_h = 0$.} : 
\begin{align*}
	\actsymb[h+1]{V}{} 
	&= 	\frac{(\actsymb[h]{V}{} + G_h - e_h)(1+i) - (b_{h+1} + E_{h+1}) \qx[]{x+h}[]}{\px[]{x+h}[]}
\end{align*}
où
\begin{description}
	\item[$G_h$]	La prime \textit{(gross premium)} à recevoir à $t = h$;
	\item[$e_h$]	Les frais relié à la collecte de la prime \textit{(per premium expenses)};
	\item[$E_h$]	Les frais reliés aux paiement de la prestation \textit{(settlement expenses)}.
\end{description}

Alternativement, on peut récrire:
\begin{align*}
	\underbrace{(b_{h+1} + E_{h + 1} - \actsymb[h + 1]{V}{})}_{\text{montant net au risque}} \qx[]{x+h}[]
	&= 	(\actsymb[h]{V}{} + G_h - e_h)(1+i)
\end{align*}


\paragraph{Formules alternatives pour Contrat d'assurance-vie entière (si $\pi^{PE}$)}
\begin{align*}
\actsymb[h]{V}{} 
	&= 	M \Ax{x+h} - \pi \ax**{x+h} 
	= 	M \left( 1 - \frac{\ax**{x+h}}{\ax**{x}} \right) 
	= 	M \left( \frac{\Ax{x+h} - \Ax{x}}{1 - \Ax{x}} \right)
\end{align*}
 Remarque : ces formules fonctionnent aussi dans le cas d'un contrat d'assurance-vie entière continu.
 
 \paragraph{Approximation classique pour les réserves à durées fractionnaires}
 \[\actsymb[h+s]{V}{} = (\actsymb[h]{V}{} + G_h - e_h)(1-s) + (\actsymb[h+1]{V}{})(s) \]



\subsection*{Profit de l'assureur}
\paragraph{Profit de l'assureur en changeant les 3 composantes}
\begin{flalign*}
\actsymb[k+1]{V}{}^A - \actsymb[k+1]{V}{}^E	& = N_k (\actsymb[k]{V}{} + G - e_k')(1+i') - (b_{k+1} + E_{k+1}' - \actsymb[k+1]{V}{}) N_k \qx[]{x+k}['] \\
& - \left[N_k(\actsymb[k]{V}{} + G = e_k)(1+i) - (b_{k+1} + E_{k+1} - \actsymb[k+1]{V}{}) N_k \qx[]{x+k}[] \right]
\end{flalign*}

\paragraph{Profit de l'assureur en changeant une seule composante} : 
\\

\begin{tabular}{|c|C|}
\hline 
Intérêt ($i$) & N_k(\actsymb[k]{V}{} + G - e_k)(i' - i) \\ 
\hline 
Frais $e_k$ ou $E_k$ & N_k(e_k - e'_k)(1 + i) + (E_{k+1} - E'_{k+1})N_k \qx[]{k+1}[] \\ 
\hline 
Mortalité $\qx{x+k}$ & (b_{k+1} + E_{k+1} - \actsymb[k+1]{V}{})(N_k \qx[]{x+k}[] - N_k \qx[]{x+k}['])\\ 
\hline 
\end{tabular} 

\subsection*{Quote-Part de l'actif (\emph{Asset shares})}
Alors que la réserve $\actsymb[t]{V}{}$ nous dit le montant que l'assureur doit avoir de côté, la quote-part de l'actif nous indique plutôt le montant réel que l'assureur a de côté pour le contrat donné.
\[AS_{K+1} = \frac{(AS_{k} + G_k - e'_k)(1 + i') - (b_{k+1} + E'_{k+1}) \qx[]{x+k}[']     }{\px[]{x+k}[']}\]


\subsection*{Équation de Thiele}
Cette équation permet d'obtenir le \emph{taux instantanné d'accroissement} de $\actsymb[t]{V}{}$.
\[\derivee{t} \left( \actsymb[t]{V}{} \right) =\delta_t \actsymb[t]{V}{} + G_t - e_t - (b_t + E_t) - \actsymb[t]{V}{} \mu_{[x] + t}  \]
on peut approximer $\actsymb[t]{V}{}$ avec la \underline{Méthode d'Euler} : 
\[\actsymb[t]{V}{} = \frac{\actsymb[t+h]{V}{} - h(G_t - e_t - (b_t + E_t)\mu_{[x] + t})}{1 + h \delta_t + h \mu_{[x] + t}}   \]

\subsection*{Modification de contrat}
\subsubsection*{Valeur de rachat (\emph{Cash value at surrender})}









% -------------------------------------------------------------
% Fin de la feuille aide-mémoire
\end{multicols*}
\end{document}
