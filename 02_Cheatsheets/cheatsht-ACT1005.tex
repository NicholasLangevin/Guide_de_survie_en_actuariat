\documentclass[10pt, french]{article}
%% -----------------------------
%% Préambule
%% -----------------------------
% !TEX encoding = UTF-8 Unicode
% LaTeX Preamble for all cheatsheets
% Author : Gabriel Crépeault-Cauchon

% HOW-TO : copy-paste this file in the same directory as your .tex file, and add in your preamble the next command right after you have specified your documentclass : 
% \input{preamble-cheatsht.tex}
% ---------------------------------------------
% ---------------------------------------------

% Extra note : this preamble creates document that are meant to be used inside the multicols environment. See the documentation on internet for further information.

%% -----------------------------
%% Encoding packages
%% -----------------------------
\usepackage[utf8]{inputenc}
\usepackage[T1]{fontenc}
\usepackage{babel}
\usepackage{lmodern}

%% -----------------------------
%% Variable definition
%% -----------------------------
\def\auteur{Gabriel Crépeault-Cauchon / Nicholas Langevin}
\def\BackgroundColor{white}

%% -----------------------------
%% Margin and layout
%% -----------------------------
% Determine the margin for cheatsheet
\usepackage[landscape, hmargin=1cm, vmargin=1.7cm]{geometry}
\usepackage{multicol}

% Remove automatic indentation after section/subsection title.
\setlength{\parindent}{0cm}

% Save space in cheatsheet by removing space between align environment and normal text.
\usepackage{etoolbox}
\newcommand{\zerodisplayskips}{%
  \setlength{\abovedisplayskip}{0pt}%
  \setlength{\belowdisplayskip}{0pt}%
  \setlength{\abovedisplayshortskip}{0pt}%
  \setlength{\belowdisplayshortskip}{0pt}}
\appto{\normalsize}{\zerodisplayskips}
\appto{\small}{\zerodisplayskips}
\appto{\footnotesize}{\zerodisplayskips}

%% -----------------------------
%% URL and links
%% -----------------------------
\usepackage{hyperref}
\hypersetup{colorlinks = true, urlcolor = gray!70!white, linkcolor = black}

%% -----------------------------
%% Document policy (uncomment only one)
%% -----------------------------
%	\usepackage{concrete}
	\usepackage{mathpazo}
%	\usepackage{frcursive} %% permet d'écrire en lettres attachées
%	\usepackage{aeguill}
%	\usepackage{mathptmx}
%	\usepackage{fourier} 

%% -----------------------------
%% Math configuration
%% -----------------------------
\usepackage[fleqn]{amsmath}
\usepackage{amsthm,amssymb,latexsym,amsfonts}
\usepackage{empheq}
\usepackage{numprint}
\usepackage{dsfont} % Pour avoir le symbole du domaine Z

% Mathematics shortcuts

\newcommand{\reels}{\mathbb{R}}
\newcommand{\entiers}{\mathbb{Z}}
\newcommand{\naturels}{\mathbb{N}}
\newcommand{\eval}{\biggr \rvert}
\usepackage{cancel}
\newcommand{\derivee}[1]{\frac{\partial}{\partial #1}}
\newcommand{\prob}[1]{\Pr \left( #1 \right)}
\newcommand{\esp}[1]{\mathrm{E} \left[ #1 \right]} % espérance
\newcommand{\variance}[1]{\mathrm{Var} \left( #1   \right)}
\newcommand{\covar}[1]{\mathrm{Cov} \left( #1   \right)}
\newcommand{\laplace}{\mathcal{L}}
\newcommand{\deriv}[2][]{\frac{\partial^{#1}}{\partial #2^{#1}}}
\newcommand{\e}[1]{\mathrm{e}^{#1}}
\newcommand{\te}[1]{\text{exp}\left\{#1\right\}}
\DeclareMathSymbol{\shortminus}{\mathbin}{AMSa}{"39}



% To indicate equation number on a specific line in align environment
\newcommand\numberthis{\addtocounter{equation}{1}\tag{\theequation}}

%
% Actuarial notation packages
%
\usepackage{actuarialsymbol}
\usepackage{actuarialangle}

%
% Matrix notation for math symbols (\bm{•})
%
\usepackage{bm}
% Matrix notation variable (bold style)
\newcommand{\matr}[1]{\mathbf{#1}}



%% -----------------------------
%% tcolorbox configuration
%% -----------------------------
\usepackage[most]{tcolorbox}
\tcbuselibrary{xparse}
\tcbuselibrary{breakable}

%%
%% Coloured box "definition" for definitions
%%
\DeclareTColorBox{definition}{ o }				% #1 parameter
{
	colframe=blue!60!green,colback=blue!5!white, % color of the box
	breakable, 
	pad at break* = 0mm, 						% to split the box
	title = {#1},
	after title = {\large \hfill \faBook},
}
%%
%% Coloured box "definition2" for definitions
%%
\DeclareTColorBox{definitionNOHFILL}{ o }				% #1 parameter
{
	colframe=blue!60!green,colback=blue!5!white, % color of the box
	pad at break* = 0mm, 						% to split the box
	title = {#1},
	before title = {\faBook \quad },
	breakable
}


%%
%% Coloured box "algo" for algorithms
%%
\newtcolorbox{algo}[ 1 ]
{
	colback = blue!5!white,
	colframe = blue!75!black,
	title=#1,
	fonttitle = \bfseries,
	breakable
}
%%
%% Coloured box "conceptgen" for points adding to a concept's deifintion
%%
\newtcolorbox{conceptgen}[ 1 ]
{
	breakable,
	colback = beaublue,
	colframe = airforceblue,
	title=#1,
	fonttitle = \bfseries
}
%%
%% Coloured box "probch3" pour formules relatives au 3ème chapitre de prob
%%
\newtcolorbox{probch3}[ 1 ]
{
	colback = ruddypink,
	colframe = burgundy,
	fonttitle = \bfseries,	
	breakable,
	title=#1
}
%%
%% Coloured box "formula" for formulas
%%
\newtcolorbox{formula}[ 1 ]
{
	colback = green!5!white,
	colframe = green!70!black,
	breakable,
	fonttitle = \bfseries,
	title=#1
}
%%
%% Coloured box "formula" for formulas
%%
\DeclareTColorBox{algo2}{ o }
{
	enhanced,
	title = #1,
	colback=blue!5!white,	
	colbacktitle=blue!75!black,
	fonttitle = \bfseries,
	breakable,
	boxed title style={size=small,colframe=arsenic} ,
	attach boxed title to top center = {yshift=-3mm,yshifttext=-1mm},
}
%%
%% Coloured box "examplebox" for formulas
%%
\newtcolorbox{examplebox}[ 1 ]
{
	colback = lightmauve,
	colframe = antiquefuchsia,
	breakable,
	fonttitle = \bfseries,title=#1
}
%%
%% Coloured box "rappel" pour rappel de formules
%%
\newtcolorbox{rappel}[ 1 ]
{
	colback = ashgrey,
	colframe = arsenic,
	breakable,
	fonttitle = \bfseries,title=#1
}
%%
%% Coloured box "rappel" pour rappel de formules
%%
\DeclareTColorBox{rappel_enhanced}{ o }
{
	enhanced,
	title = #1,
	colback=ashgrey, % color of the box
%	colframe=blue(pigment),
%	colframe=arsenic,	
	colbacktitle=arsenic,
	fonttitle = \bfseries,
	breakable,
	boxed title style={size=small,colframe=arsenic} ,
	attach boxed title to top center = {yshift=-3mm,yshifttext=-1mm},
}
%%
%% Coloured box "notation" for notation and terminology
%%
\DeclareTColorBox{distributions}{ o }			% #1 parameter
{
	enhanced,
	title = #1,
	colback=gray(x11gray), % color of the box
%	colframe=blue(pigment),
	colframe=arsenic,	
	colbacktitle=aurometalsaurus,
	fonttitle = \bfseries,
	boxed title style={size=small,colframe=arsenic} ,
	attach boxed title to top center = {yshift=-3mm,yshifttext=-1mm},
	breakable
%	left=0pt,
%  	right=0pt,
%    box align=center,
%    ams align*
%  	top=-10pt
}

%% -----------------------------
%% Graphics and pictures
%% -----------------------------
\usepackage{graphicx}
\usepackage{pict2e}
\usepackage{tikz}

%% -----------------------------
%% insert pdf pages into document
%% -----------------------------
\usepackage{pdfpages}

%% -----------------------------
%% Color configuration
%% -----------------------------
\usepackage{color, soulutf8, colortbl}


%
%	Colour definitions
%
\definecolor{blue(munsell)}{rgb}{0.0, 0.5, 0.69}
\definecolor{blue(matcha)}{rgb}{0.596, 0.819, 1.00}
\definecolor{blue(munsell)-light}{rgb}{0.5, 0.8, 0.9}
\definecolor{bleudefrance}{rgb}{0.19, 0.55, 0.91}
\definecolor{blizzardblue}{rgb}{0.67, 0.9, 0.93}
\definecolor{bondiblue}{rgb}{0.0, 0.58, 0.71}
\definecolor{blue(pigment)}{rgb}{0.2, 0.2, 0.6}
\definecolor{bluebell}{rgb}{0.64, 0.64, 0.82}
\definecolor{airforceblue}{rgb}{0.36, 0.54, 0.66}
\definecolor{beaublue}{rgb}{0.74, 0.83, 0.9}
\definecolor{cobalt}{rgb}{0.0, 0.28, 0.67}	% nice light blue-ish
\definecolor{blue_rectangle}{RGB}{83, 84, 244}		% ACT-2004
\definecolor{indigo(web)}{rgb}{0.29, 0.0, 0.51}	% purple-ish
\definecolor{antiquefuchsia}{rgb}{0.57, 0.36, 0.51}	%	pastel dark purple ish
\definecolor{darkpastelpurple}{rgb}{0.59, 0.44, 0.84}
\definecolor{gray(x11gray)}{rgb}{0.75, 0.75, 0.75}
\definecolor{aurometalsaurus}{rgb}{0.43, 0.5, 0.5}
\definecolor{ruddypink}{rgb}{0.88, 0.56, 0.59}
\definecolor{pastelred}{rgb}{1.0, 0.41, 0.38}		
\definecolor{lightmauve}{rgb}{0.86, 0.82, 1.0}
\definecolor{azure(colorwheel)}{rgb}{0.0, 0.5, 1.0}
\definecolor{darkgreen}{rgb}{0.0, 0.2, 0.13}			
\definecolor{burntorange}{rgb}{0.8, 0.33, 0.0}		
\definecolor{burntsienna}{rgb}{0.91, 0.45, 0.32}		
\definecolor{ao(english)}{rgb}{0.0, 0.5, 0.0}		% ACT-2003
\definecolor{amber(sae/ece)}{rgb}{1.0, 0.49, 0.0} 	% ACT-2004
\definecolor{green_rectangle}{RGB}{131, 176, 84}		% ACT-2004
\definecolor{red_rectangle}{RGB}{241,112,113}		% ACT-2004
\definecolor{amethyst}{rgb}{0.6, 0.4, 0.8}
\definecolor{amethyst-light}{rgb}{0.6, 0.4, 0.8}
\definecolor{ashgrey}{rgb}{0.7, 0.75, 0.71}			% dark grey-black-ish
\definecolor{arsenic}{rgb}{0.23, 0.27, 0.29}			% light green-beige-ish gray
\definecolor{amaranth}{rgb}{0.9, 0.17, 0.31}
\definecolor{brickred}{rgb}{0.8, 0.25, 0.33}
\definecolor{pastelred}{rgb}{1.0, 0.41, 0.38}

%
% Useful shortcuts for coloured text
%
\newcommand{\orange}{\textcolor{orange}}
\newcommand{\red}{\textcolor{red}}
\newcommand{\cyan}{\textcolor{cyan}}
\newcommand{\blue}{\textcolor{blue}}
\newcommand{\green}{\textcolor{green}}
\newcommand{\purple}{\textcolor{magenta}}
\newcommand{\yellow}{\textcolor{yellow}}

%% -----------------------------
%% Enumerate environment configuration
%% -----------------------------
%
% Custum enumerate & itemize Package
%
\usepackage{enumitem}
%
% French Setup for itemize function
%
\frenchbsetup{StandardItemLabels=true}
%
% Change default label for itemize
%
\renewcommand{\labelitemi}{\faAngleRight}


%% -----------------------------
%% Tabular column type configuration
%% -----------------------------
\newcolumntype{C}{>{$}c<{$}} % math-mode version of "l" column type
\newcolumntype{L}{>{$}l<{$}} % math-mode version of "l" column type
\newcolumntype{R}{>{$}r<{$}} % math-mode version of "l" column type
\newcolumntype{f}{>{\columncolor{green!20!white}}p{1cm}}
\newcolumntype{g}{>{\columncolor{green!40!white}}m{1.2cm}}
\newcolumntype{a}{>{\columncolor{red!20!white}$}p{2cm}<{$}}	% ACT-2005
% configuration to force a line break within a single cell
\usepackage{makecell}


%% -----------------------------
%% Fontawesome for special symbols
%% -----------------------------
\usepackage{fontawesome}

%% -----------------------------
%% Section Font customization
%% -----------------------------
\usepackage{sectsty}
\sectionfont{\color{\SectionColor}}
\subsectionfont{\color{\SubSectionColor}}

%% -----------------------------
%% Footer/Header Customization
%% -----------------------------
\usepackage{lastpage}
\usepackage{fancyhdr}
\pagestyle{fancy}

%
% Header
%
\fancyhead{} 	% Reset
\fancyhead[L]{Aide-mémoire pour~ \cours ~(\textbf{\sigle})}
\fancyhead[R]{\auteur}

%
% Footer
%
\fancyfoot{}		% Reset
\fancyfoot[R]{\thepage ~de~ \pageref{LastPage}}
\fancyfoot[L]{\href{https://github.com/ressources-act/Guide_de_survie_en_actuariat}{\faGithub \ ressources-act/Guide de survie en actuariat}}
%
% Page background color
%
\pagecolor{\BackgroundColor}




%% END OF PREAMBLE
% ---------------------------------------------
% ---------------------------------------------
%% -----------------------------
%% Redefine from template
%% -----------------------------
\def\auteur{Alec James van Rassel}
%% -----------------------------
%% Variable definition
%% -----------------------------
\def\cours{Analyse et traitement collectif du risque}
\def\sigle{ACT-1005}
%% -----------------------------
%% Colour setup for sections
%% -----------------------------
\def\SectionColor{brickred}
\def\SubSectionColor{amaranth}
\def\SubSubSection{pastelred}


%\setcounter{section}{1}

%% -----------------------------
%% Début du document
%% -----------------------------
\begin{document}

\begin{multicols*}{3} 
\section{Le risque et l'assurance}
\begin{definitionNOHFILL}[Risque]
\begin{description}
	\item[\textbf{Un} risque:] Un événement dont l'occurrence est (habituellement) aléatoire pouvant causer un dommage à des personnes et/ou des biens;
	\item[\textbf{Le} risque:] La probabilité de survenance de l'événement et l'ampleur de ses conséquences;
\end{description}

Il y \textbf{deux composantes} aux risques:
\begin{itemize}
	\item	La \textbf{probabilité d'occurrence} d'un événement accidentel;
	\item	La \textbf{gravité} des effets (ou conséquences) \textit{financière} de l'événement;
\end{itemize}

Donc du point de vue d'un assureur, le risque est l'\textbf{exposition} à un \textit{événement} dommageable inhérent à une situation (ou activité). 

Quelques \textbf{exemples} d'événements dont l'exposition peut être prise en charge par une compagnie:
\begin{itemize}
	\item	Une compagnie d'assurance auto assure une personne contre le risque d'un accident automobile;
	\item	Une compagnie d'assurance de voyage assure une personne contre le \textbf{danger} (toujours sous forme de conséquence \textit{financière}, que ce soit au niveau de la responsibilité civile, des frais médicaux etc.) d'aller au Mexique;
%%%	-----
%%%	NOTE:
%%%	+	Danger de se faire voler? Tuer? Je ne suis pas certain;
%%% + J'ai émis mon hypothèse, à revoir avec Isabelle ~ OC ~
%%%	-----
\end{itemize}
\end{definitionNOHFILL}

\begin{definitionNOHFILL}[Aversion]
L'aversion au risque est la \textit{peur} d'un investisseur d'un risque qu'il juge trop important.
(L'antonyme de l'\textit{aversion} au risque serait la \textit{tolérance} de celui-ci)

L'aversion au risque se caractérise par une personne qui:
\begin{itemize}
	\item	Ne souhaite pas courir le risque et va vouloir le \textbf{transférer};
	\item[]	Pour exemple, assurer sa maison contre le risque d'inondation;
	\item	Ne juge pas d'être en mesure de supporter le risque et \textbf{refuse} de s'y exposer;
	\item[]	Pour exemple, ne pas faire de parachutisme;
\end{itemize}
	
Le \textbf{degré d'aversion} au risque est \textbf{variable} selon l'intervenant (tous on une aversion au risque, seul le \textit{degré d'aversion} diffère). Par exemple, même les compagnies d'assurance se \textit{\textbf{ré}}assurent. 

Habituellement, elles ont moins d'aversion au risque qu'un individu en raison de leur:
\begin{itemize}
	\item	\textbf{capacité financière};
	\item	La \textbf{mise en commun} des risques;
\end{itemize} 

Lorsqu'un individu souhaite \textit{transférer} son risque, il échange au preneur de risque une \textbf{prime de risque}. En assurance, c'est donc une \textit{prime d'assurance} qu'un \textbf{assuré} va payer à sa \textbf{compagnie d'assurance}.
\end{definitionNOHFILL}

\begin{algo}{Gérer du risque}
Différentes \textbf{méthodes} existent pour gérer un risque, pour exemple:
\begin{itemize}[leftmargin = *]
	\item	Évitement (Ex : Éviter d'avoir une voiture);
	\item	Prévention (Conserver le risque réduit grâce à la prévention);
	\item	Prise de risque (\textbf{rétention}) (intentionnelle ou non);
	\item	Transfert (Principe fondamental de l'assurance);
	\item	Diversification des risques (Ne pas tous mettre ces oeufs dans le même panier);
	\item	\textcolor{amaranth}{Couverture des risques (\textbf{hedging}) (Non-Couvert dans le cadre du cours)};
	\item	\textcolor{amaranth}{La titrisation (Non-Couvert dans le cadre du cours)};
\end{itemize}
%
Face à un risque, différents \textbf{comportements} peuvent survenir selon:
\begin{itemize}
	\item	La \textit{perception} du risque;
	\item	L'\textit{aversion} au risque;
	\item	La disponibilité d'\textit{outils} pour gérer des risques;
%%%	-----
%%%	NOTE:
%%%	+	Outils c'est vague, trouver un exemple plus concret;
%%%	+	Genre "outils" dans le sens d'argent ou outils dans le sens de marteaux, etc.?
%%%	+	Je ne suis pas certain de comprendre ce qu'on veut dire;
%%%	-----
\end{itemize}
\end{algo}

\begin{definitionNOHFILL}[L'assurance]
L'assurance est un \textit{système} qui permet de \textit{protéger} un assuré (individu, association, entreprise) contre les \textbf{conséquences \textbf{financières}} découlant de la survenance d'un risque \textit{spécifique}.
%%%	-----
%%%	NOTE:
%%%	+	Je dis "un risque spécifique" au lieu "d'un risque particulier" pour éviter toute confusion avec l'assurance de particuliers / personnes
%%%
%%%	-----

Les assureurs est en mesure de protéger les individus contre un risque grâce à \textit{loi des grands nombres}:
\begin{itemize}
	\item	On associe un assuré à une \textbf{communauté} de personnes---l'ensemble des assurés;
	\item	On \textbf{rassemble} \textit{(pool)} les primes; 
\end{itemize}
Lorsque des risques se réalisent, on \textbf{indemnise} les membres ayant subi des dommages. Ce faisant, la communauté prend \textbf{matériellement} en charge les dommages de ses membres.

On définit donc l'assurance comme un système de gestion des risques basé sur la notion de \textit{solidarité}. Ce \textbf{mécanisme} de l'assurance :
\begin{itemize}
	\item	Ne modifie ni la \textit{fréquence} du risque ni sa \textit{sévérité};
	\item	\textit{Transfère} le risque d'un assuré à un, ou plusieurs, autres;
	\item	\textit{Protège} un assuré contre le risque de survenance d'événements qu'il ne peut pas supporter seul;
	\item	\textit{Permet} à un assuré de réaliser des activités comportant des risques qu'il n'aurait pas autrement pu supporter;
\end{itemize}
\end{definitionNOHFILL}
Lorsque des risques se réalisent, on \textbf{indemnise} les membres ayant subi des dommages. Ce faisant, la communauté prend \textbf{matériellement} en charge les dommages de ses membres.

\begin{algo}{\textbf{Revenu} de l'assureur}
\begin{itemize}
	\item	L'assureur reçoit les \textbf{primes} d'assurance;
	\item	L'assureur \textbf{place l'argent} des assurés, excédentaire des paiements qu'il doit faire, en bourse;
\end{itemize}
Ainsi, il obtient une deuxième source de revenus (primordiale dans le cas d'assurances d'une potentielle longue durée, comme l'\textit{assurance vie}).
\end{algo}

\begin{conceptgen}{Types d'Assurances}
\textbf{1. Assurance de personnes}  

Exemples : 
%\begin{multicols*}{2}
\begin{itemize}[leftmargin = *]
	\item	Décès et longévité;
	\item	Invalidité;
	\item	Perte d'emploi;
	\item	Autres soins de santé (médical et paramédical, dentaire, lunettes);
\end{itemize}  

Certains de ces \textit{risques} sont couverts par l'État, alors que les autres pourront l'être par des compagnies privées.   
%\end{multicols*}

\textbf{2. Assurance IARD} (\textbf{I}ncendie, \textbf{A}ccidents et \textbf{R}isques \textbf{D}ivers)  \textcolor{amaranth}{(Couvert dans le cours \textit{Introduction à l'actuariat I})}  

Exemples pour les \textcolor{amethyst}{individus} et \textcolor{amaranth}{entreprises}:
\begin{itemize}[leftmargin = *]
	\item	\textcolor{amethyst}{Biens (auto, habitation)};
	\item	\textcolor{amaranth}{Biens (auto, bâtiment)};
	\item	\textcolor{amaranth}{Opérations};
\end{itemize}
\end{conceptgen}


\section{La sécurité sociale}
%\subsection{Définition}
%\subsection{Deux différents conceptions}
%\subsection{Bismarck}
%\subsection{Beveridge}
%\subsection{Déclaration universelle de droits de l'homme (UN) (1948)}
%\subsection{Sécurité sociale au Canada et au Québec}
%\subsubsection*{Historique}
%\subsubsection*{Types de programmes}
%\subsubsection*{Catégories de programmes}
%\subsubsection*{Assistance sociale}
%\subsubsection*{Assurance sociale}
%\subsubsection*{Classement des régimes existants}
%\subsection{Régimes sociaux sous tension}

\end{multicols*}
%% -----------------------------
%% Fin du document
%% -----------------------------
\end{document}
