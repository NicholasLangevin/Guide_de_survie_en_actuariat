\documentclass[10pt, french]{article}
%% -----------------------------
%% Préambule
%% -----------------------------
% !TEX encoding = UTF-8 Unicode
% LaTeX Preamble for all cheatsheets
% Author : Gabriel Crépeault-Cauchon

% HOW-TO : copy-paste this file in the same directory as your .tex file, and add in your preamble the next command right after you have specified your documentclass : 
% \input{preamble-cheatsht.tex}
% ---------------------------------------------
% ---------------------------------------------

% Extra note : this preamble creates document that are meant to be used inside the multicols environment. See the documentation on internet for further information.

%% -----------------------------
%% Encoding packages
%% -----------------------------
\usepackage[utf8]{inputenc}
\usepackage[T1]{fontenc}
\usepackage{babel}
\usepackage{lmodern}

%% -----------------------------
%% Variable definition
%% -----------------------------
\def\auteur{Gabriel Crépeault-Cauchon / Nicholas Langevin}
\def\BackgroundColor{white}

%% -----------------------------
%% Margin and layout
%% -----------------------------
% Determine the margin for cheatsheet
\usepackage[landscape, hmargin=1cm, vmargin=1.7cm]{geometry}
\usepackage{multicol}

% Remove automatic indentation after section/subsection title.
\setlength{\parindent}{0cm}

% Save space in cheatsheet by removing space between align environment and normal text.
\usepackage{etoolbox}
\newcommand{\zerodisplayskips}{%
  \setlength{\abovedisplayskip}{0pt}%
  \setlength{\belowdisplayskip}{0pt}%
  \setlength{\abovedisplayshortskip}{0pt}%
  \setlength{\belowdisplayshortskip}{0pt}}
\appto{\normalsize}{\zerodisplayskips}
\appto{\small}{\zerodisplayskips}
\appto{\footnotesize}{\zerodisplayskips}

%% -----------------------------
%% URL and links
%% -----------------------------
\usepackage{hyperref}
\hypersetup{colorlinks = true, urlcolor = gray!70!white, linkcolor = black}

%% -----------------------------
%% Document policy (uncomment only one)
%% -----------------------------
%	\usepackage{concrete}
	\usepackage{mathpazo}
%	\usepackage{frcursive} %% permet d'écrire en lettres attachées
%	\usepackage{aeguill}
%	\usepackage{mathptmx}
%	\usepackage{fourier} 

%% -----------------------------
%% Math configuration
%% -----------------------------
\usepackage[fleqn]{amsmath}
\usepackage{amsthm,amssymb,latexsym,amsfonts}
\usepackage{empheq}
\usepackage{numprint}
\usepackage{dsfont} % Pour avoir le symbole du domaine Z

% Mathematics shortcuts

\newcommand{\reels}{\mathbb{R}}
\newcommand{\entiers}{\mathbb{Z}}
\newcommand{\naturels}{\mathbb{N}}
\newcommand{\eval}{\biggr \rvert}
\usepackage{cancel}
\newcommand{\derivee}[1]{\frac{\partial}{\partial #1}}
\newcommand{\prob}[1]{\Pr \left( #1 \right)}
\newcommand{\esp}[1]{\mathrm{E} \left[ #1 \right]} % espérance
\newcommand{\variance}[1]{\mathrm{Var} \left( #1   \right)}
\newcommand{\covar}[1]{\mathrm{Cov} \left( #1   \right)}
\newcommand{\laplace}{\mathcal{L}}
\newcommand{\deriv}[2][]{\frac{\partial^{#1}}{\partial #2^{#1}}}
\newcommand{\e}[1]{\mathrm{e}^{#1}}
\newcommand{\te}[1]{\text{exp}\left\{#1\right\}}
\DeclareMathSymbol{\shortminus}{\mathbin}{AMSa}{"39}



% To indicate equation number on a specific line in align environment
\newcommand\numberthis{\addtocounter{equation}{1}\tag{\theequation}}

%
% Actuarial notation packages
%
\usepackage{actuarialsymbol}
\usepackage{actuarialangle}

%
% Matrix notation for math symbols (\bm{•})
%
\usepackage{bm}
% Matrix notation variable (bold style)
\newcommand{\matr}[1]{\mathbf{#1}}



%% -----------------------------
%% tcolorbox configuration
%% -----------------------------
\usepackage[most]{tcolorbox}
\tcbuselibrary{xparse}
\tcbuselibrary{breakable}

%%
%% Coloured box "definition" for definitions
%%
\DeclareTColorBox{definition}{ o }				% #1 parameter
{
	colframe=blue!60!green,colback=blue!5!white, % color of the box
	breakable, 
	pad at break* = 0mm, 						% to split the box
	title = {#1},
	after title = {\large \hfill \faBook},
}
%%
%% Coloured box "definition2" for definitions
%%
\DeclareTColorBox{definitionNOHFILL}{ o }				% #1 parameter
{
	colframe=blue!60!green,colback=blue!5!white, % color of the box
	pad at break* = 0mm, 						% to split the box
	title = {#1},
	before title = {\faBook \quad },
	breakable
}


%%
%% Coloured box "algo" for algorithms
%%
\newtcolorbox{algo}[ 1 ]
{
	colback = blue!5!white,
	colframe = blue!75!black,
	title=#1,
	fonttitle = \bfseries,
	breakable
}
%%
%% Coloured box "conceptgen" for points adding to a concept's deifintion
%%
\newtcolorbox{conceptgen}[ 1 ]
{
	breakable,
	colback = beaublue,
	colframe = airforceblue,
	title=#1,
	fonttitle = \bfseries
}
%%
%% Coloured box "probch3" pour formules relatives au 3ème chapitre de prob
%%
\newtcolorbox{probch3}[ 1 ]
{
	colback = ruddypink,
	colframe = burgundy,
	fonttitle = \bfseries,	
	breakable,
	title=#1
}
%%
%% Coloured box "formula" for formulas
%%
\newtcolorbox{formula}[ 1 ]
{
	colback = green!5!white,
	colframe = green!70!black,
	breakable,
	fonttitle = \bfseries,
	title=#1
}
%%
%% Coloured box "formula" for formulas
%%
\DeclareTColorBox{algo2}{ o }
{
	enhanced,
	title = #1,
	colback=blue!5!white,	
	colbacktitle=blue!75!black,
	fonttitle = \bfseries,
	breakable,
	boxed title style={size=small,colframe=arsenic} ,
	attach boxed title to top center = {yshift=-3mm,yshifttext=-1mm},
}
%%
%% Coloured box "examplebox" for formulas
%%
\newtcolorbox{examplebox}[ 1 ]
{
	colback = lightmauve,
	colframe = antiquefuchsia,
	breakable,
	fonttitle = \bfseries,title=#1
}
%%
%% Coloured box "rappel" pour rappel de formules
%%
\newtcolorbox{rappel}[ 1 ]
{
	colback = ashgrey,
	colframe = arsenic,
	breakable,
	fonttitle = \bfseries,title=#1
}
%%
%% Coloured box "rappel" pour rappel de formules
%%
\DeclareTColorBox{rappel_enhanced}{ o }
{
	enhanced,
	title = #1,
	colback=ashgrey, % color of the box
%	colframe=blue(pigment),
%	colframe=arsenic,	
	colbacktitle=arsenic,
	fonttitle = \bfseries,
	breakable,
	boxed title style={size=small,colframe=arsenic} ,
	attach boxed title to top center = {yshift=-3mm,yshifttext=-1mm},
}
%%
%% Coloured box "notation" for notation and terminology
%%
\DeclareTColorBox{distributions}{ o }			% #1 parameter
{
	enhanced,
	title = #1,
	colback=gray(x11gray), % color of the box
%	colframe=blue(pigment),
	colframe=arsenic,	
	colbacktitle=aurometalsaurus,
	fonttitle = \bfseries,
	boxed title style={size=small,colframe=arsenic} ,
	attach boxed title to top center = {yshift=-3mm,yshifttext=-1mm},
	breakable
%	left=0pt,
%  	right=0pt,
%    box align=center,
%    ams align*
%  	top=-10pt
}

%% -----------------------------
%% Graphics and pictures
%% -----------------------------
\usepackage{graphicx}
\usepackage{pict2e}
\usepackage{tikz}

%% -----------------------------
%% insert pdf pages into document
%% -----------------------------
\usepackage{pdfpages}

%% -----------------------------
%% Color configuration
%% -----------------------------
\usepackage{color, soulutf8, colortbl}


%
%	Colour definitions
%
\definecolor{blue(munsell)}{rgb}{0.0, 0.5, 0.69}
\definecolor{blue(matcha)}{rgb}{0.596, 0.819, 1.00}
\definecolor{blue(munsell)-light}{rgb}{0.5, 0.8, 0.9}
\definecolor{bleudefrance}{rgb}{0.19, 0.55, 0.91}
\definecolor{blizzardblue}{rgb}{0.67, 0.9, 0.93}
\definecolor{bondiblue}{rgb}{0.0, 0.58, 0.71}
\definecolor{blue(pigment)}{rgb}{0.2, 0.2, 0.6}
\definecolor{bluebell}{rgb}{0.64, 0.64, 0.82}
\definecolor{airforceblue}{rgb}{0.36, 0.54, 0.66}
\definecolor{beaublue}{rgb}{0.74, 0.83, 0.9}
\definecolor{cobalt}{rgb}{0.0, 0.28, 0.67}	% nice light blue-ish
\definecolor{blue_rectangle}{RGB}{83, 84, 244}		% ACT-2004
\definecolor{indigo(web)}{rgb}{0.29, 0.0, 0.51}	% purple-ish
\definecolor{antiquefuchsia}{rgb}{0.57, 0.36, 0.51}	%	pastel dark purple ish
\definecolor{darkpastelpurple}{rgb}{0.59, 0.44, 0.84}
\definecolor{gray(x11gray)}{rgb}{0.75, 0.75, 0.75}
\definecolor{aurometalsaurus}{rgb}{0.43, 0.5, 0.5}
\definecolor{ruddypink}{rgb}{0.88, 0.56, 0.59}
\definecolor{pastelred}{rgb}{1.0, 0.41, 0.38}		
\definecolor{lightmauve}{rgb}{0.86, 0.82, 1.0}
\definecolor{azure(colorwheel)}{rgb}{0.0, 0.5, 1.0}
\definecolor{darkgreen}{rgb}{0.0, 0.2, 0.13}			
\definecolor{burntorange}{rgb}{0.8, 0.33, 0.0}		
\definecolor{burntsienna}{rgb}{0.91, 0.45, 0.32}		
\definecolor{ao(english)}{rgb}{0.0, 0.5, 0.0}		% ACT-2003
\definecolor{amber(sae/ece)}{rgb}{1.0, 0.49, 0.0} 	% ACT-2004
\definecolor{green_rectangle}{RGB}{131, 176, 84}		% ACT-2004
\definecolor{red_rectangle}{RGB}{241,112,113}		% ACT-2004
\definecolor{amethyst}{rgb}{0.6, 0.4, 0.8}
\definecolor{amethyst-light}{rgb}{0.6, 0.4, 0.8}
\definecolor{ashgrey}{rgb}{0.7, 0.75, 0.71}			% dark grey-black-ish
\definecolor{arsenic}{rgb}{0.23, 0.27, 0.29}			% light green-beige-ish gray
\definecolor{amaranth}{rgb}{0.9, 0.17, 0.31}
\definecolor{brickred}{rgb}{0.8, 0.25, 0.33}
\definecolor{pastelred}{rgb}{1.0, 0.41, 0.38}

%
% Useful shortcuts for coloured text
%
\newcommand{\orange}{\textcolor{orange}}
\newcommand{\red}{\textcolor{red}}
\newcommand{\cyan}{\textcolor{cyan}}
\newcommand{\blue}{\textcolor{blue}}
\newcommand{\green}{\textcolor{green}}
\newcommand{\purple}{\textcolor{magenta}}
\newcommand{\yellow}{\textcolor{yellow}}

%% -----------------------------
%% Enumerate environment configuration
%% -----------------------------
%
% Custum enumerate & itemize Package
%
\usepackage{enumitem}
%
% French Setup for itemize function
%
\frenchbsetup{StandardItemLabels=true}
%
% Change default label for itemize
%
\renewcommand{\labelitemi}{\faAngleRight}


%% -----------------------------
%% Tabular column type configuration
%% -----------------------------
\newcolumntype{C}{>{$}c<{$}} % math-mode version of "l" column type
\newcolumntype{L}{>{$}l<{$}} % math-mode version of "l" column type
\newcolumntype{R}{>{$}r<{$}} % math-mode version of "l" column type
\newcolumntype{f}{>{\columncolor{green!20!white}}p{1cm}}
\newcolumntype{g}{>{\columncolor{green!40!white}}m{1.2cm}}
\newcolumntype{a}{>{\columncolor{red!20!white}$}p{2cm}<{$}}	% ACT-2005
% configuration to force a line break within a single cell
\usepackage{makecell}


%% -----------------------------
%% Fontawesome for special symbols
%% -----------------------------
\usepackage{fontawesome}

%% -----------------------------
%% Section Font customization
%% -----------------------------
\usepackage{sectsty}
\sectionfont{\color{\SectionColor}}
\subsectionfont{\color{\SubSectionColor}}

%% -----------------------------
%% Footer/Header Customization
%% -----------------------------
\usepackage{lastpage}
\usepackage{fancyhdr}
\pagestyle{fancy}

%
% Header
%
\fancyhead{} 	% Reset
\fancyhead[L]{Aide-mémoire pour~ \cours ~(\textbf{\sigle})}
\fancyhead[R]{\auteur}

%
% Footer
%
\fancyfoot{}		% Reset
\fancyfoot[R]{\thepage ~de~ \pageref{LastPage}}
\fancyfoot[L]{\href{https://github.com/ressources-act/Guide_de_survie_en_actuariat}{\faGithub \ ressources-act/Guide de survie en actuariat}}
%
% Page background color
%
\pagecolor{\BackgroundColor}




%% END OF PREAMBLE
% ---------------------------------------------
% ---------------------------------------------
%% -----------------------------
%% Variable definition
%% -----------------------------
\def\cours{Statistiques mathématiques avancées}
\def\sigle{STT-7115}

%% -----------------------------
%% Colour setup for sections
%% -----------------------------
\def\SectionColor{burntorange}
\def\SubSectionColor{burntsienna}
\def\SubSubSectionColor{burntsienna}

%% -----------------------------
%% Colour setup for prestations
%%	Ajoute couleurs sur les trêmas des signes de prestations
%% -----------------------------
\usepackage{stackengine}
\newcommand\cumlaut[2][black]{\stackon[.33ex]{#2}{\textcolor{#1}{\kern-.04ex.\kern-.2ex.}}}
%
% Save more space than default
%
\setlength{\abovedisplayskip}{-15pt}
\setlist{leftmargin=*}
%
%	Extra math symbols
%
\usepackage{mathrsfs}
\usetikzlibrary{matrix}
%
% thin space, limits underneath in displays
%

%% -----------------------------
%% 	Colour setup for sections
%% -----------------------------
\def\SectionColor{cobalt}
\def\SubSectionColor{azure(colorwheel)}
\def\SubSubSection{azure(colorwheel)}
%% -----------------------------
\setcounter{secnumdepth}{0}

%% -----------------------------
%% Color definitions
%% -----------------------------
\definecolor{indigo(web)}{rgb}{0.29, 0.0, 0.51}
\definecolor{cobalt}{rgb}{0.0, 0.28, 0.67}
\definecolor{azure(colorwheel)}{rgb}{0.0, 0.5, 1.0}
%% -----------------------------
%% Variable definition
%% -----------------------------
%%
%% Matrix notation variable (bold style)
%%
\newcommand\cololine[2]{\colorlet{temp}{.}\color{#1}\bar{\color{temp}#2}\color{temp}}
\newcommand\colbar[2]{\colorlet{temp}{.}\color{#1}\bar{\color{temp}#2}\color{temp}}

\begin{document}

\begin{center}
	\textsc{\Large Contributeurs}\\[0.5cm] 
\end{center}
%\begin{contrib}{ACT-1XXX\: Cours de première année}
\begin{description}
	\item[aut., cre.] Alec James van Rassel
\end{description}
\end{contrib}

\newpage
\raggedcolumns
\begin{multicols*}{2}
\section{Variables aléatoires}
\subsection{Notions aléatoires}
\begin{definitionGENERAL}{Notion d'expérience aléatoire}[\circled{1}{trueblue}]
Cadre dans lequel on observe différentes actions dues au hasard.

\begin{distributions}[Notation]
\begin{description}
	\item[$\omega$]	Le \textbf{\textit{résultat}} d'une expérience aléatoire, alias \textit{épreuve} ou \textit{issue}.
	\item[$\Omega$]	L'\textbf{\textit{ensemble} des résultats possibles}.
		\begin{itemize}
		\item	Il s'ensuit que $\omega \in \Omega$.
		\item	Par exemple, pour le lancer d'un dé où l'on désire savoir le résultat $\Omega = \{\text{pile}, \text{face}\}$.
		\item	On dénote par $\mathcal{P}(\Omega)$ l'\textbf{ensemble de toutes les parties} de $\Omega$.
		\end{itemize}
\end{description}
\end{distributions}
\end{definitionGENERAL}

\begin{definitionGENERAL}{Notion d'événement aléatoire}[\circled{2}{trueblue}]
Événement lié à une certain expérience aléatoire. 

\bigskip	

Un événement est tout \textbf{sous-ensemble} de $\Omega$. Par exemple, pour l'expérience aléatoire de jeter un dé on a que l'ensemble des résultats possibles $\Omega = \{1, 2, 3, 4, 5, 6\}$. L'événement $A$ \og obtenir un nombre pair \fg{} s'écrit $A = \{2, 4, 6\}$. De ceci on déduit qu'à toute propriété définie sur $\Omega$, on associe un sous-ensemble de $\Omega$ composé de tous les $\omega$ qui vérifient la propriété. 
\end{definitionGENERAL}

\subsubsection{Algèbre de Boole des événements}
\begin{definitionNOHFILL}[Algèbre de Boole (\og \textit{boolean algebra} \fg{}) des événements]
La classe $\mathcal{E}$ des événements est l'\textbf{algèbre de Boole de parties de $\Omega$}, si elle contient $\Omega$ et est stable par intersection, réunion et complémentation.

\bigskip

\paragraph{Note}	On dit habituellement algèbre plutôt qu'algèbre de Boole.
\end{definitionNOHFILL}


\begin{definitionNOHFILLprop}[Opérations logiques]
Les opérations logiques que l'on peut effectuer sur les événements sont : 
\begin{enumerate}
	\item	Soit les événements $A \subset \Omega$ et $B \subset \Omega$, alors : 
		\begin{itemize}
		\item	$A \cup B$ est un événement réalisé ssi \textbf{au moins un} des deux est réalisé.
		\item	$A \cap B$ est un événement réalisé ssi \textbf{les deux} sont réalisés simultanément.
		\end{itemize}
	\item	$\emptyset$ est un événement qui ne peut être réalisé appelé l'\textbf{événement impossible}. À chaque expérience, $\Omega$ est toujours réalisé et appelé l'\textbf{événement certain}.
	\item	$A \subset \Omega$ est un événement.
		\begin{itemize}
		\item	Le complément $A^{c}$ ou $\overline{A}$ est appelé \textbf{événement contraire de $A$} et se réalise si $\omega \notin A$.
		\end{itemize}
	\item	La \textbf{différence de deux événements} $A$ et $B$ est \lfbox[formula]{$A \smallsetminus B = A \cap B^{c}$} se réalise si $A$ est réalisé mais pas $B$.
	\item	La \textbf{différence symétrique} de $A$ et $B$ est \lfbox[formula]{$A \Delta B = (A \smallsetminus B) \cup (B \smallsetminus A)$} se réalise si l'un des deux événements est réalisé mais pas l'autre.
	\item	Si, $\forall n \in \mathbb{N}$, l'événement $A_{n}$ représente \og \textbf{gagner $n$ matchs} \fg{}, alors 
		\begin{itemize}
		\item	$\displaystyle \cup_{n = 1}^{\infty} A_{n}$ représente \og \textbf{gagner au moins un match}\fg{}.
		\item	$\displaystyle \cap_{n = 1}^{\infty} A_{n}^{c}$ représente \og \textbf{ne pas gagner de matchs}\fg{}.
		\end{itemize}
	\item	Deux événements sont \textbf{incompatibles} si \lfbox[formula]{$A_{1} \cap A_{2} = \emptyset$}. 
		\begin{itemize}
		\item	On peut aussi dire que les parties de $\Omega$ représentées par $A_{1}$ et $A_{2}$ sont disjointes.
		\item	Si deux événements sont incompatibles, on a une somme au lieu d'une réunion avec \lfbox[formula]{$A_{1} \cup A_{2} = A_{1} + A_{2}$} si $A_{1} \cap A_{2} = \emptyset$.
		\end{itemize}
	\item	Si les événements de la suite $(A_{i})_{i \in \mathbb{I}}$ forment une \textbf{partition} de $\Omega$, on dit que ses événements $(A_{i})_{i \in \mathbb{I}}$ forment un \textbf{système exhaustif} de $\Omega$.
	\item	La suite d'événements $(A_{n})_{n \in \mathbb{N}^{\ast}}$ est : 
		\begin{description}
		\item[croissante]	ssi $A_{1} \subset A_{2} \subset \hdots$
		\item[décroissante]	ssi $A_{1} \supset A_{2} \supset \hdots$
		\end{description}
	\item	Si la suite $(A_{n})_{n \in \mathbb{N}^{\ast}}$ est une suite d'événements d'un ensemble $\Omega$, on représente que :
		\begin{description}
		\item[une infinité de $A_{n}$ est réalisé]	en écrivant que, quel que soit le rang $k \in \mathbb{N}^{\ast}$, il existe des événements de rang supérieur (à $k$) qui sont réalisés : \lfbox[formula]{$\cap_{k = 1}^{\infty}\cup_{n = k}^{\infty} A_{n}$}.
		\item[un nombre fini de $A_{n}$ est réalisé]	en écrivant qu'il existe un rang tel qu'à partir de ce rang, tous les événements réalisés sont les contraires des événements $A_{n}$ : \lfbox[formula]{$\cup_{k = 1}^{\infty}\cap_{n = k}^{\infty} A_{n}^{c}$}.
		\end{description}
		
\end{enumerate}
\end{definitionNOHFILLprop}

\begin{definitionNOHFILLsub}[Limites de suite d'événements]
Soit $(A_{n})_{n \in \mathbb{N}^{+}}$ une suite d'événements de $\Omega$. On défini les limites $\inf$ et $\sup$ d'événements par : 
\begin{align*}
	A_{\ast} 
	&=	\lim \inf A_{n} 
	=	\bigcup_{k = 1}^{\infty} \bigcap_{n = k}^{\infty} A_{n}	\\
	A^{\ast} 
	&=	\lim \sup A_{n} 
	=	\bigcap_{k = 1}^{\infty} \bigcup_{n = k}^{\infty} A_{n}
\end{align*}

De plus, si les ensembles $A_{\ast}$ et $A^{\ast}$ coïncident, alors on écrit $A = A_{\ast} = A^{\ast} = \lim_{n \rightarrow \infty} A_{n}$.

\begin{definitionNOHFILLprop}[Propositions]
Soit $(A_{n})_{n \in \mathbb{N}^{+}}$ une suite d'événements de $\Omega$.
\begin{enumerate}[label = \roman*)]
	\item	Si $A_{1} \subset A_{2} \subset \hdots$ alors \lfbox[formula]{$\displaystyle \lim_{n \rightarrow \infty} A_{n} = \bigcup_{n = 1}^{\infty} A_{n}$}.
	\item	Si $A_{1} \supset A_{2} \supset \hdots$ alors \lfbox[formula]{$\displaystyle \lim_{n \rightarrow \infty} A_{n} = \bigcap_{n = 1}^{\infty} A_{n}$}.
\end{enumerate}
\end{definitionNOHFILLprop}
\end{definitionNOHFILLsub}



\columnbreak
\subsection{Espaces probabilisables}
\begin{definitionNOHFILL}[Tribu d'événements]
La tribu (ou $\sigma$-algèbre) sur un ensemble $\Omega$ est un ensemble $\mathcal{A}$ de parties de $\Omega$ tel que : 
\begin{enumerate}[label = \roman*)]
	\item	$\Omega \in \mathcal{A}$.
	\item	Si $A \in \mathcal{A}$, alors $A^{c} \in \mathcal{A}$.
	\item	$\forall (A_{n})_{n \in \mathbb{N}^{\ast}}$ une suite d'éléments de $\mathcal{A}$, alors l'événement $\displaystyle \bigcup_{n = 1}^{\infty} A_{n} \in \mathcal{A}$.
\end{enumerate}
\end{definitionNOHFILL}

Il y a une multitude de façons de choisir une tribu. Par exemple : 
\begin{itemize}
	\item	la tribu la plus "grossière" est $\mathcal{A} = \{\emptyset, \Omega\}$.
	\item	la tribu la plus "grosse" est $\mathcal{A} = \mathcal{P}(\Omega)$.
\end{itemize}


\begin{definitionNOHFILL}[Espace probabilisable (ou mesurable)]
Le couple $(\Omega, \mathcal{A})$ composé d'un ensemble $\Omega$ et une tribu $\mathcal{A}$ sur $\Omega$.

\bigskip

Les éléments de $\Omega$ sont appelés \textbf{\textit{éventualités}} et les éléments de $\mathcal{A}$ \textbf{\textit{événements}}.
\end{definitionNOHFILL}

\begin{definitionNOHFILLprop}[Propriétés de la tribu]
Soit $\mathcal{A}$ une tribu sur $\Omega$. Alors : 
\begin{enumerate}[label = \alph*)]
	\item	\lfbox[formula]{$\emptyset \in \mathcal{A}$}.
	\item	$\forall A_{1}, \dots, A_{k} \in \mathcal{A}$, alors \lfbox[formula]{$\displaystyle \bigcup_{i = 1}^{k} A_{i} \in \mathcal{A}$} et \lfbox[formula]{$\displaystyle \bigcap_{i = 1}^{k} A_{i} \in \mathcal{A}$}.
	\item	$\forall (A_{n})_{n \in \mathbb{N}^{\ast}}$ suite d'événements de $\mathcal{A}$, alors \lfbox[formula]{$\displaystyle \bigcap_{n \in \mathbb{N}^{\ast}} A_{n} \in \mathcal{A}$}.
	\item	$\forall (A_{n})_{n \in \mathbb{N}^{\ast}}$ suite d'événements de $\mathcal{A}$, alors \lfbox[formula]{$\lim \inf A_{ n} \in \mathcal{A}$}.
	\item	$\forall (A_{n})_{n \in \mathbb{N}^{\ast}}$ suite d'événements de $\mathcal{A}$, alors \lfbox[formula]{$\lim \sup A_{n} \in \mathcal{A}$}.
\end{enumerate}

\paragraph{Note}	Voir la page 19 des notes de cours du chapitre 1 pour les preuves.
\end{definitionNOHFILLprop}



\columnbreak
\subsection{Variables aléatoires}
\begin{rappel_enhanced}[Contexte]
Souvent, un événement s'énonce de façon numérique (p. ex. : \og rouler un 5 \fg{}, \og pluie de 2mm \fg{}, etc.). Donc, à toute expérience $\omega$, on associe un nombre $X(\omega)$ ou un $n$-uple de nombres $(X_{1}(\omega), \dots, X_{n}(\omega))$ mesurant un caractère, ou un ensemble de $n$ caractères, du résultat de l'expérience. 

\bigskip

On suppose que $X$ désigne une application : $\Omega \rightarrow \mathbb{R}$ et que $(X_{1}, \dots, X_{n})$ désigne une application $\Omega \rightarrow \mathbb{R}^{n}$. On dénote les événements les plus simples comme \lfbox[formula]{$\{X \in I\} = \{\omega \in \Omega : X(\omega) \in I\} = X^{-1}(I)$} où $I$ est un intervalle réel.
\end{rappel_enhanced}

\begin{definitionNOHFILL}[Variable aléatoire réelle]
Tout application à valeurs réelles $X: \Omega \rightarrow \mathbb{R}$ telle que, $\forall$ intervalle $I$ de $\mathbb{R}$, $\{X \in I\}$ soit un événement de la tribu $\mathcal{A}$.
\end{definitionNOHFILL}

\begin{definitionNOHFILLsub}[Tribu borélienne]
La tribu borélienne $\mathcal{B}_{\mathbb{R}}$ est la plus petite tribu de $\mathbb{R}$ qui contient tous ses intervalles. Les éléments de $\mathcal{B}_{\mathbb{R}}$ sont appelés les \textbf{\textit{boréliens}} de $\mathbb{R}$.

\bigskip

Pour une v.a. réelle $X$, alors $\forall B \in \mathcal{B}_{\mathbb{R}}$ on a $\{X \in B\} \in \mathcal{A}$. 

\bigskip

Bref, \lfbox[formula]{$(\Omega, \mathcal{A}) \overset{X}{\rightarrow} (\mathbb{R}, \mathcal{B}_{\mathbb{R}})$}. On dit que la tribu $X^{-1}(\mathcal{B}_{\mathbb{R}})$ sur $\Omega$ est la \textbf{\textit{tribu des événements engendrés par $X$}}.
\end{definitionNOHFILLsub}



\columnbreak
\subsection{Probabilités}
\begin{definitionNOHFILL}[Notion de Probabilité]
Soit un espace probabilisable $(\Omega, \mathcal{A})$. On appelle \textbf{\textit{probabilité}} sur $(\Omega, \mathcal{A})$ toute application $P: \mathcal{A} \rightarrow [0, 1]$ telle que :
\begin{enumerate}[label = \roman*)]
	\item	\lfbox[formula]{$P(\Omega) = 1$}.
	\item	$\forall (A_{n})_{n \in \mathbb{N}^{\ast}}$ d'événements deux à deux disjoints, \lfbox[formula]{$\displaystyle P\left(\bigcup_{n \in \mathbb{N}^{\ast}} A_{n}\right) = \sum_{n \in \mathbb{N}^{\ast}} P(A_{n})$}.
\end{enumerate}
\end{definitionNOHFILL}

\begin{definitionNOHFILL}[Espace probabilisé]
Le triplet $(\Omega, \mathcal{A}, P)$ s'appelle un espace probabilisé.
\end{definitionNOHFILL}

On complète la notion précédente sur l'espace borélien avec $(\Omega, \mathcal{A}, P) \overset{X}{\rightarrow} (\mathbb{R}, \mathcal{B}_{\mathbb{R}}, P_{X})$ où $P_{X}$ est appelée loi de probabilité de $X$. On définit $P_{X}(B) = P(X \in B) = P\left(X^{-1}(B)\right)$.


\begin{definitionNOHFILLprop}[Propriétés des probabilités]
Soit $(\Omega, \mathcal{A}, P)$ un espace probabilisé. Alors : 
\begin{enumerate}[label = \alph*)]
	\item	$P(\emptyset) = 0$.
	\item	Si $A$ et $B$ sont des événements disjoints, alors \lfbox[formula]{$P(A \cup B) = P(A) + P(B)$}.
	\item	Si $A$ et $B$ sont des événements quelconques, alors \lfbox[formula]{$P(A \cup B) = P(A) + P(B) - P(A \cap B)$}.
	\item	Si $A$ et $B$ sont des événements tels que $A \subset B$, alors \lfbox[formula]{$P(A \smallsetminus B) = P(B) - P(A)$} et \lfbox[formula]{$P(A) \leq P(B)$}.
	\item	$\forall A \in \mathcal{A}$, \lfbox[formula]{$P(A^{c}) = 1 - P(A)$}.
	\item	Si $(A_{n})_{n \in \mathbb{N}^{\ast}}$ est une suite d'événements quelconques, alors \lfbox[formula]{$\displaystyle P\left(\cup_{n \in \mathbb{N}^{\ast}} A_{n}\right) \leq \sum_{n \in \mathbb{N}^{\ast}} P(A_{n})$}.
	\item	Si $(A_{n})_{n \in \mathbb{N}^{\ast}}$ est une suite d'événements tels que $A_{n} \downarrow \emptyset$, alors \lfbox[formula]{$P(A_{n}) \downarrow 0$}.
	\item	Si $(A_{n})_{n \in \mathbb{N}^{\ast}}$ est une suite d'événements tels que $A_{n} \downarrow A$, alors \lfbox[formula]{$P(A_{n}) \downarrow P(A)$}.
	\item	Si $(A_{n})_{n \in \mathbb{N}^{\ast}}$ est une suite d'événements tels que $A_{n} \uparrow A$, alors \lfbox[formula]{$P(A_{n}) \uparrow P(A)$}.
\end{enumerate}
\end{definitionNOHFILLprop}

\end{multicols*}

\end{document}
