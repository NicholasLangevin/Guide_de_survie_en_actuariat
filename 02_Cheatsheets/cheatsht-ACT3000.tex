\documentclass[10pt, french]{article}
%% -----------------------------
%% Préambule
%% -----------------------------
% !TEX encoding = UTF-8 Unicode
% LaTeX Preamble for all cheatsheets
% Author : Gabriel Crépeault-Cauchon

% HOW-TO : copy-paste this file in the same directory as your .tex file, and add in your preamble the next command right after you have specified your documentclass : 
% \input{preamble-cheatsht.tex}
% ---------------------------------------------
% ---------------------------------------------

% Extra note : this preamble creates document that are meant to be used inside the multicols environment. See the documentation on internet for further information.

%% -----------------------------
%% Encoding packages
%% -----------------------------
\usepackage[utf8]{inputenc}
\usepackage[T1]{fontenc}
\usepackage{babel}
\usepackage{lmodern}

%% -----------------------------
%% Variable definition
%% -----------------------------
\def\auteur{Gabriel Crépeault-Cauchon / Nicholas Langevin}
\def\BackgroundColor{white}

%% -----------------------------
%% Margin and layout
%% -----------------------------
% Determine the margin for cheatsheet
\usepackage[landscape, hmargin=1cm, vmargin=1.7cm]{geometry}
\usepackage{multicol}

% Remove automatic indentation after section/subsection title.
\setlength{\parindent}{0cm}

% Save space in cheatsheet by removing space between align environment and normal text.
\usepackage{etoolbox}
\newcommand{\zerodisplayskips}{%
  \setlength{\abovedisplayskip}{0pt}%
  \setlength{\belowdisplayskip}{0pt}%
  \setlength{\abovedisplayshortskip}{0pt}%
  \setlength{\belowdisplayshortskip}{0pt}}
\appto{\normalsize}{\zerodisplayskips}
\appto{\small}{\zerodisplayskips}
\appto{\footnotesize}{\zerodisplayskips}

%% -----------------------------
%% URL and links
%% -----------------------------
\usepackage{hyperref}
\hypersetup{colorlinks = true, urlcolor = gray!70!white, linkcolor = black}

%% -----------------------------
%% Document policy (uncomment only one)
%% -----------------------------
%	\usepackage{concrete}
	\usepackage{mathpazo}
%	\usepackage{frcursive} %% permet d'écrire en lettres attachées
%	\usepackage{aeguill}
%	\usepackage{mathptmx}
%	\usepackage{fourier} 

%% -----------------------------
%% Math configuration
%% -----------------------------
\usepackage[fleqn]{amsmath}
\usepackage{amsthm,amssymb,latexsym,amsfonts}
\usepackage{empheq}
\usepackage{numprint}
\usepackage{dsfont} % Pour avoir le symbole du domaine Z

% Mathematics shortcuts

\newcommand{\reels}{\mathbb{R}}
\newcommand{\entiers}{\mathbb{Z}}
\newcommand{\naturels}{\mathbb{N}}
\newcommand{\eval}{\biggr \rvert}
\usepackage{cancel}
\newcommand{\derivee}[1]{\frac{\partial}{\partial #1}}
\newcommand{\prob}[1]{\Pr \left( #1 \right)}
\newcommand{\esp}[1]{\mathrm{E} \left[ #1 \right]} % espérance
\newcommand{\variance}[1]{\mathrm{Var} \left( #1   \right)}
\newcommand{\covar}[1]{\mathrm{Cov} \left( #1   \right)}
\newcommand{\laplace}{\mathcal{L}}
\newcommand{\deriv}[2][]{\frac{\partial^{#1}}{\partial #2^{#1}}}
\newcommand{\e}[1]{\mathrm{e}^{#1}}
\newcommand{\te}[1]{\text{exp}\left\{#1\right\}}
\DeclareMathSymbol{\shortminus}{\mathbin}{AMSa}{"39}



% To indicate equation number on a specific line in align environment
\newcommand\numberthis{\addtocounter{equation}{1}\tag{\theequation}}

%
% Actuarial notation packages
%
\usepackage{actuarialsymbol}
\usepackage{actuarialangle}

%
% Matrix notation for math symbols (\bm{•})
%
\usepackage{bm}
% Matrix notation variable (bold style)
\newcommand{\matr}[1]{\mathbf{#1}}



%% -----------------------------
%% tcolorbox configuration
%% -----------------------------
\usepackage[most]{tcolorbox}
\tcbuselibrary{xparse}
\tcbuselibrary{breakable}

%%
%% Coloured box "definition" for definitions
%%
\DeclareTColorBox{definition}{ o }				% #1 parameter
{
	colframe=blue!60!green,colback=blue!5!white, % color of the box
	breakable, 
	pad at break* = 0mm, 						% to split the box
	title = {#1},
	after title = {\large \hfill \faBook},
}
%%
%% Coloured box "definition2" for definitions
%%
\DeclareTColorBox{definitionNOHFILL}{ o }				% #1 parameter
{
	colframe=blue!60!green,colback=blue!5!white, % color of the box
	pad at break* = 0mm, 						% to split the box
	title = {#1},
	before title = {\faBook \quad },
	breakable
}


%%
%% Coloured box "algo" for algorithms
%%
\newtcolorbox{algo}[ 1 ]
{
	colback = blue!5!white,
	colframe = blue!75!black,
	title=#1,
	fonttitle = \bfseries,
	breakable
}
%%
%% Coloured box "conceptgen" for points adding to a concept's deifintion
%%
\newtcolorbox{conceptgen}[ 1 ]
{
	breakable,
	colback = beaublue,
	colframe = airforceblue,
	title=#1,
	fonttitle = \bfseries
}
%%
%% Coloured box "probch3" pour formules relatives au 3ème chapitre de prob
%%
\newtcolorbox{probch3}[ 1 ]
{
	colback = ruddypink,
	colframe = burgundy,
	fonttitle = \bfseries,	
	breakable,
	title=#1
}
%%
%% Coloured box "formula" for formulas
%%
\newtcolorbox{formula}[ 1 ]
{
	colback = green!5!white,
	colframe = green!70!black,
	breakable,
	fonttitle = \bfseries,
	title=#1
}
%%
%% Coloured box "formula" for formulas
%%
\DeclareTColorBox{algo2}{ o }
{
	enhanced,
	title = #1,
	colback=blue!5!white,	
	colbacktitle=blue!75!black,
	fonttitle = \bfseries,
	breakable,
	boxed title style={size=small,colframe=arsenic} ,
	attach boxed title to top center = {yshift=-3mm,yshifttext=-1mm},
}
%%
%% Coloured box "examplebox" for formulas
%%
\newtcolorbox{examplebox}[ 1 ]
{
	colback = lightmauve,
	colframe = antiquefuchsia,
	breakable,
	fonttitle = \bfseries,title=#1
}
%%
%% Coloured box "rappel" pour rappel de formules
%%
\newtcolorbox{rappel}[ 1 ]
{
	colback = ashgrey,
	colframe = arsenic,
	breakable,
	fonttitle = \bfseries,title=#1
}
%%
%% Coloured box "rappel" pour rappel de formules
%%
\DeclareTColorBox{rappel_enhanced}{ o }
{
	enhanced,
	title = #1,
	colback=ashgrey, % color of the box
%	colframe=blue(pigment),
%	colframe=arsenic,	
	colbacktitle=arsenic,
	fonttitle = \bfseries,
	breakable,
	boxed title style={size=small,colframe=arsenic} ,
	attach boxed title to top center = {yshift=-3mm,yshifttext=-1mm},
}
%%
%% Coloured box "notation" for notation and terminology
%%
\DeclareTColorBox{distributions}{ o }			% #1 parameter
{
	enhanced,
	title = #1,
	colback=gray(x11gray), % color of the box
%	colframe=blue(pigment),
	colframe=arsenic,	
	colbacktitle=aurometalsaurus,
	fonttitle = \bfseries,
	boxed title style={size=small,colframe=arsenic} ,
	attach boxed title to top center = {yshift=-3mm,yshifttext=-1mm},
	breakable
%	left=0pt,
%  	right=0pt,
%    box align=center,
%    ams align*
%  	top=-10pt
}

%% -----------------------------
%% Graphics and pictures
%% -----------------------------
\usepackage{graphicx}
\usepackage{pict2e}
\usepackage{tikz}

%% -----------------------------
%% insert pdf pages into document
%% -----------------------------
\usepackage{pdfpages}

%% -----------------------------
%% Color configuration
%% -----------------------------
\usepackage{color, soulutf8, colortbl}


%
%	Colour definitions
%
\definecolor{blue(munsell)}{rgb}{0.0, 0.5, 0.69}
\definecolor{blue(matcha)}{rgb}{0.596, 0.819, 1.00}
\definecolor{blue(munsell)-light}{rgb}{0.5, 0.8, 0.9}
\definecolor{bleudefrance}{rgb}{0.19, 0.55, 0.91}
\definecolor{blizzardblue}{rgb}{0.67, 0.9, 0.93}
\definecolor{bondiblue}{rgb}{0.0, 0.58, 0.71}
\definecolor{blue(pigment)}{rgb}{0.2, 0.2, 0.6}
\definecolor{bluebell}{rgb}{0.64, 0.64, 0.82}
\definecolor{airforceblue}{rgb}{0.36, 0.54, 0.66}
\definecolor{beaublue}{rgb}{0.74, 0.83, 0.9}
\definecolor{cobalt}{rgb}{0.0, 0.28, 0.67}	% nice light blue-ish
\definecolor{blue_rectangle}{RGB}{83, 84, 244}		% ACT-2004
\definecolor{indigo(web)}{rgb}{0.29, 0.0, 0.51}	% purple-ish
\definecolor{antiquefuchsia}{rgb}{0.57, 0.36, 0.51}	%	pastel dark purple ish
\definecolor{darkpastelpurple}{rgb}{0.59, 0.44, 0.84}
\definecolor{gray(x11gray)}{rgb}{0.75, 0.75, 0.75}
\definecolor{aurometalsaurus}{rgb}{0.43, 0.5, 0.5}
\definecolor{ruddypink}{rgb}{0.88, 0.56, 0.59}
\definecolor{pastelred}{rgb}{1.0, 0.41, 0.38}		
\definecolor{lightmauve}{rgb}{0.86, 0.82, 1.0}
\definecolor{azure(colorwheel)}{rgb}{0.0, 0.5, 1.0}
\definecolor{darkgreen}{rgb}{0.0, 0.2, 0.13}			
\definecolor{burntorange}{rgb}{0.8, 0.33, 0.0}		
\definecolor{burntsienna}{rgb}{0.91, 0.45, 0.32}		
\definecolor{ao(english)}{rgb}{0.0, 0.5, 0.0}		% ACT-2003
\definecolor{amber(sae/ece)}{rgb}{1.0, 0.49, 0.0} 	% ACT-2004
\definecolor{green_rectangle}{RGB}{131, 176, 84}		% ACT-2004
\definecolor{red_rectangle}{RGB}{241,112,113}		% ACT-2004
\definecolor{amethyst}{rgb}{0.6, 0.4, 0.8}
\definecolor{amethyst-light}{rgb}{0.6, 0.4, 0.8}
\definecolor{ashgrey}{rgb}{0.7, 0.75, 0.71}			% dark grey-black-ish
\definecolor{arsenic}{rgb}{0.23, 0.27, 0.29}			% light green-beige-ish gray
\definecolor{amaranth}{rgb}{0.9, 0.17, 0.31}
\definecolor{brickred}{rgb}{0.8, 0.25, 0.33}
\definecolor{pastelred}{rgb}{1.0, 0.41, 0.38}

%
% Useful shortcuts for coloured text
%
\newcommand{\orange}{\textcolor{orange}}
\newcommand{\red}{\textcolor{red}}
\newcommand{\cyan}{\textcolor{cyan}}
\newcommand{\blue}{\textcolor{blue}}
\newcommand{\green}{\textcolor{green}}
\newcommand{\purple}{\textcolor{magenta}}
\newcommand{\yellow}{\textcolor{yellow}}

%% -----------------------------
%% Enumerate environment configuration
%% -----------------------------
%
% Custum enumerate & itemize Package
%
\usepackage{enumitem}
%
% French Setup for itemize function
%
\frenchbsetup{StandardItemLabels=true}
%
% Change default label for itemize
%
\renewcommand{\labelitemi}{\faAngleRight}


%% -----------------------------
%% Tabular column type configuration
%% -----------------------------
\newcolumntype{C}{>{$}c<{$}} % math-mode version of "l" column type
\newcolumntype{L}{>{$}l<{$}} % math-mode version of "l" column type
\newcolumntype{R}{>{$}r<{$}} % math-mode version of "l" column type
\newcolumntype{f}{>{\columncolor{green!20!white}}p{1cm}}
\newcolumntype{g}{>{\columncolor{green!40!white}}m{1.2cm}}
\newcolumntype{a}{>{\columncolor{red!20!white}$}p{2cm}<{$}}	% ACT-2005
% configuration to force a line break within a single cell
\usepackage{makecell}


%% -----------------------------
%% Fontawesome for special symbols
%% -----------------------------
\usepackage{fontawesome}

%% -----------------------------
%% Section Font customization
%% -----------------------------
\usepackage{sectsty}
\sectionfont{\color{\SectionColor}}
\subsectionfont{\color{\SubSectionColor}}

%% -----------------------------
%% Footer/Header Customization
%% -----------------------------
\usepackage{lastpage}
\usepackage{fancyhdr}
\pagestyle{fancy}

%
% Header
%
\fancyhead{} 	% Reset
\fancyhead[L]{Aide-mémoire pour~ \cours ~(\textbf{\sigle})}
\fancyhead[R]{\auteur}

%
% Footer
%
\fancyfoot{}		% Reset
\fancyfoot[R]{\thepage ~de~ \pageref{LastPage}}
\fancyfoot[L]{\href{https://github.com/ressources-act/Guide_de_survie_en_actuariat}{\faGithub \ ressources-act/Guide de survie en actuariat}}
%
% Page background color
%
\pagecolor{\BackgroundColor}




%% END OF PREAMBLE
% ---------------------------------------------
% ---------------------------------------------
%% -----------------------------
%% Variable definition
%% -----------------------------
\def\cours{Théorie du risque}
\def\sigle{ACT-3000}
%% -----------------------------
%% Colour setup for sections
%% -----------------------------
\def\SectionColor{darkpastelpurple}
\def\SubSectionColor{darkpastelpurple}
\def\SubSubSectionColor{darkpastelpurple}

% 
% Débute numérotation des sections  à 1
% 
\setcounter{section}{0}

%% -----------------------------
%% Début du document
%% -----------------------------
\begin{document}

\begin{center}
	\textsc{\Large Contributeurs}\\[0.5cm] 
\end{center}
\begin{contrib}{ACT-3000\: Théorie du risque}
\begin{description}
	\item[aut.] Gabriel Crépeault-Cauchon 
	\item[src.] Étienne Marceau 
\end{description}
\end{contrib}


\newpage

\begin{multicols*}{2} 

\section{Rappels d'intro 2}
\subsection{Mesures de risques}
Voir les preuves de TVaR en annexe. Il est pratique de se rappeler des 3 formes de la TVaR.

\paragraph{Value-at-risk}
$VaR_{\kappa}(X) = F_{X}^{-1}(\kappa)$. De plus, pour $\varphi$ une fonction strictement croissante, on a que
\begin{align*}
VaR_{\kappa}(\varphi(X)) = \varphi \left ( VaR_{\kappa}(X) \right )
\end{align*}



\section{Distribution multivariées}
\subsection{Classes de Fréchet}
Soit $F_1, \dots, F_n$ des fonction de répartition univariées et $F_{\matr{x}} = F_{X_1, \dots, X_n}$ la fonction de répartition du vecteur $\matr{X}$.

On définit la classe de Fréchet $CF(F_1, \dots, F_n)$ par l'ensemble des fonctions de répartition $F_{\matr{X}}$ dont les marginales sont $F_1, \dots, F_n$.

\subsubsection{Bornes d'une classe de Fréchet}
Si $F_{\matr{X}} \in CF(F_1, \dots, F_n)$, alors
\begin{align*}
W(x_1, \dots, x_n) \leq F_{\matr{X}}(x_1, \dots, x_n) \leq M(x_1, \dots, x_n)
\end{align*}
où
\begin{equation}
\label{eq:cf-born-inf}
W(x_1, \dots, x_n) = \max \left (\sum_{i=1}^{n} F_i(x_i) - (n-1) ; 0 \right )
\end{equation}
et
\begin{equation}
\label{eq:cf-born-sup}
M(x_1, \dots, x_n)	 = \min \left ( F_1(x_1), \dots, F_n(x_n) \right)
\end{equation}


\hl{Preuve des bornes à savoir!}


\subsection{Comonotonicité}
Les composantes de $\matr{X}$ sont dites comonotones si $X_i = F_{X_i}(U)$, $i = 1, \dots, n$ et $U \sim U(0,1)$.

\subsubsection{Algorithme}
\begin{enumerate}
\item Simuler $U^{(j)}$ de la v.a. $U \sim U(0,1)$
\item Calculer $X_i^{(j)} = F_{X_i}(U^{(j)})$, $i = 1, \dots, n$
\end{enumerate}

\begin{definition}[variable comonotone et la borne supérieure de Fréchet]
Le vecteur $\matr{X}$ a des composantes comonotones ssi
\begin{align*}
F_{\matr{X}(x_1, \dots, x_n)} = M(x_1, \dots, x_n)
\end{align*}
\hl{Preuve à savoir}
\end{definition}

\begin{definition}[Additivité des $VaR$ et $TVaR$]
On définit $S = \sum_{i=1}^{n} X_i = \sum_{i=1}^{n} F_{X_i}(U) = \varphi(U)$, où $\varphi$ est une fonction croissante pour $y \in (0,1)$. Alors, on a
\begin{align*}
VaR_{\kappa}(S) & = \sum_{i=1}^{n} VaR_{\kappa}(X_i) \\
TVaR_{\kappa}(S) & = \sum_{i=1}^{n} TVaR_{\kappa}(X_i)
\end{align*}
\hl{Preuve à savoir}
\end{definition}


\subsection{Antimonotonicité}
Un couple de v.a.\footnote{L'antimonotonicité est seulement définie pour $n=2$.} $\matr{X} = (X_1, X_2)$ dont les composantes sont définies par $X_1 = F_{X_1}(U)$ et $X_2 = F_{X_2}(1-U)$ est antimonotone par définition.

\subsubsection{Algorithme}
\begin{enumerate}
\item Simuler $U_{(j)}$ de la v.a. $U \sim U(0,1)$
\item Calculer $X_1^{(j)} = F_{X_1}(U^{(j)})$ et $X_2^{(j)} = F_{X_2}(1 - U^{(j)})$
\end{enumerate}

\begin{definition}[variable antimonotone et la borne inférieure de Fréchet]
Le vecteur $\matr{X} = (X_1, X_2)$ a des composantes antimonotone ssi
\begin{align*}
F_{\matr{X}(x_1,x_2)} = W(x_1, x_2)
\end{align*}
\hl{Preuve à savoir}
\end{definition}

\subsection{Loi de Poisson bivariée Teicher}
\begin{itemize}
\item Couple de v.a. $(M_1, M_2)$ dont les marginales sont $Pois(\lambda_1)$ $Pois(\lambda_2)$
\item paramètre de dépendance $\alpha_0$ avec $0 \leq \alpha_0 \leq \min(\lambda_1, \lambda_2)$
\item $\alpha_1 = \lambda - \alpha_0$ et $\alpha_2 = \lambda_2 - \alpha_0$
\item On définit les v.a. $M_1$ et $M_2$ telles que (avec $K_i \sim Pois(\alpha_i)$)
\begin{align*}
M_1 = K_1 + K_0 \text{ et } M_2 = K_2 + K_0
\end{align*}
avec $M_i \sim Pois(\lambda_i)$
\end{itemize}

\subsubsection{Fonction de masse de probabilité (fmp)}

\begin{align*}
f_{M_1, M_2}(m_1, m_2) = e^{-\lambda_i - \lambda_2 + \alpha_0} \sum_{j=0}^{\min(m_1, m2)} \frac{\alpha_0^{j}}{j!} \frac{(\lambda_1 - \alpha_0)^{m_1 - j}}{(m_1 - j)!} \frac{(\lambda_2 - \alpha_0)^{m_2 - j}}{(m_2 - j)!}
\end{align*}



\hl{Preuve à savoir}

\subsubsection{Fonction génératrice des probabilités (fgp)}
\begin{align*}
P_{M_1, M_2}(t_1, t_2) = e^{(\lambda_1 - \alpha_0)(t_1 -1)} e^{(\lambda_2 - \alpha_0)(t_2 - 1)} e^{\alpha_0(t_1 t_2 - 1)}
\end{align*}
\hl{Preuve à savoir}

\paragraph{Covariance de $M_1$ et $M_2$} $\covar{M_1, M_2} = \variance{K_0} = \alpha_0$
\hl{Preuve à savoir}

\subsubsection{Connaître la loi de $N = M_1 + M_2$}
\red{À terminer}




\subsection{Loi exponentielle bivariée EFGM}
\paragraph{fonction de répartition}
La fonction de répartition est
\begin{align*}
F_{X_1, X_2}(x_1, x_2)	& = (1 - e^{-\beta_1 x_1})(1- e^{-\beta_2 x_2}) \\
& + \theta (1 - e^{-\beta_1 x_1})(1 - e^{-\beta_2 x_2}) e^{-\beta_1 x_1} e^{-\beta_2 x_2}
\end{align*}

\paragraph{fgm}
Il \hl{faut savoir prouver} que la fgm est
\begin{align*}
M_{X_1, X_2}(t_1, t_2) & = (1 +\theta) \left ( \frac{\beta_1}{\beta_1 - t_1} \right  )  \left ( \frac{\beta_2}{\beta_2 - t_2} \right  ) \\
& - \theta \left ( \frac{2 \beta_1}{2 \beta_1 - t_1} \right  )  \left ( \frac{\beta_2}{\beta_2 - t_2} \right  ) - \theta \left ( \frac{\beta_1}{\beta_1 - t_1} \right  )  \left ( \frac{2 \beta_2}{2 \beta_2 - t_2} \right  ) \\
& + \theta \left ( \frac{2 \beta_1}{2 \beta_1 - t_1} \right  )  \left ( \frac{2 \beta_2}{2 \beta_2 - t_2} \right  )
\end{align*}

\paragraph{Coefficient de corrélation}
\hl{Il faut savoir prouver que} la coefficient de corrélation est
\begin{align*}
\rho_P(X_1, X_2) = \frac{\theta}{4}
\end{align*}

\paragraph{Fonction de densité} On peut obtenir la fonction de densité de la loi exponentielle bivariée en dérivant 2 fois
\begin{align*}
f_{X_1, X_2}(x_1, x_2) = \frac{\partial^2}{\partial x_1 \partial x_2} F_{X_1, X_2}(x_1, x_2)
\end{align*}



\section{Problématiques d'un rapport du BSIF}
Un extrait d'un rapport du BSIF \footnote{Étude d'impact quantitative No4 du BSIF, 2012.}  présente \textbf{plusieurs incohérences}, notamment : 

\begin{itemize}
\item La corrélation est la formule ou la méthode utilisée dans le présent document pour mesurer l'association entre des variables ; 
\item L'intervalle de confiance est de -1  à 1
\item Une corrélation de 1 (corrélation positive parfaite) sous-entend que l'augmentation d'une variable donnée équivaudra à la hausse d'une autre variable ;
\item Une corrélation de -1 (corrélation négative parfaite) sous-entend que l'augmentation d'une variable donnée se traduira par une baisse correspondante d'une autre variable
\item Une corrélation zéro sous-entend l'absence de relation, ou indépendance, entre deux variables.
\end{itemize}



% Théorie des copules
\section{Théorie des copules}

\begin{definition}[Définition d'une copule]
Une copule $C$ est la fonction de répartition d'un vecteur de va $\matr{U} = (U_1, U_2)$ dont les composantes $U_i \sim U(0,1)$, $i=1,2$. Une copule satisfait les propriétés suivantes : 
\begin{enumerate}[label=(\arabic*)]
\item $C(u_1, u_2)$ est non-décroissante sur $(0,1)^2$
\item $C(u_1, u_2)$ est continue à droite sur $(0,1)^2$
\item $\lim_{u_i \to 0} C(u_1, u_2) = 0$, $i=1,2$
\item $\lim_{u_{3-1} \to 1} C(u_1, u_2) = u_i$, $i=1,2$
\item Inégalité du rectangle : $\forall a_i \leq b_i$, $i=1,2$, on a
\begin{align*}
C(b_1, b_2) - C(b_1, a_2) - C(a_1, b_2) + c(a_1, a_2) \geq 0.
\end{align*}
Cette égalité sera satisfaite si $\frac{\partial^2}{\partial u_1 \partial u_2} C(u_1, u_2) \geq 0$.
\end{enumerate}
\red{On peut généraliser ces définitions pour une copule multivariée.}
\end{definition}

\subsection{Théorème de Sklar}
\begin{definition}[Théorème de Sklar]
Soit $F_\matr{X} \in \mathcal{CF}(F_1, F_2)$ ayant les fonctions de répartition $F_1$ et $F_2$. Il y a 2 volets au théorème : 
\begin{description}
\item[Volet \# 1] Il existe une copule $C$ telle que, $\forall x \in \reels^n$,
\begin{align*}
F(x_1, \dots, x_n) = C(F_1(x_1), \dots, F_n(x_n))
\end{align*}

\item[Volet \# 2] Inversement, si $C$ est une copule de $F_1, \dots, F_n$ sont des fonctions de répartition, alors la fonction définie par
\begin{align*}
F(x_1, \dots, x_n) = C(F_1(x_1), \dots, F_n(x_n))
\end{align*}
est une fonction de répartition multivariée avec les fonctions de répartition marginales $F_1, \dots, F_n$.
\end{description}
\hl{À prouver}
\end{definition}

\subsection{Comment extraire une copule?}
Soit un vecteur de v.a. continues avec fonction de répartition $F_\matr{X} \in \mathcal{CF}(F_1, \dots, F_n)$. Alors, la copule $C$ associée à $F_\matr{X}$ est donnée par
\begin{equation}
C(u_1, \dots, u_n) = F_\matr{X} \left (F_1^{-1}(u_1), \dots, F_n^{-1}(u_n) \right)
\end{equation}

\subsection{Bornes de Fréchet}
Puisque $C$ est une fonction de répartition, on a
\begin{align*}
W(u_1, \dots, u_n) \leq C(u_1, \dots, u_n) \leq M(u_1, \dots, u_n)
\end{align*}
où $W$ et $M$ sont les bornes inférieures (voir Éq. \ref{eq:cf-born-inf}) et supérieures (voir Éq.  \ref{eq:cf-born-sup}), respectivement .

\subsection{Fonction de densité d'une copule}
\begin{equation}
c(u_1, \dots, u_n) = \frac{\partial^2}{\partial u_1, \dots, \partial u_n} C(u_1, \dots, u_n)
\end{equation}
De plus, on a
\begin{align*}
f_{\matr{X}}(x_1, \dots, x_n)	& =  \frac{\partial^2}{\partial u_1, \dots, \partial u_n} F_\matr{X}(x_1, \dots, x_n) \\
& =   \frac{\partial^2}{\partial u_1, \dots, \partial u_n} C\left( F_1(x_1), \dots, F_n(x_n) \right) \\
& = c(u_1, \dots, u_n) f_{X_1}(x_1) \dots f_{X_n}(x_n)
\end{align*}

\subsection{Fonction de répartition conditionnelle d'une copule}
\begin{equation}
\label{eq:repart-cond-copule}
C_{2|1}(u_2 | u_1) = \derivee{u_1} C(u_1, u_2)
\end{equation}
Une relation similaire existe pour $C_{1|2}$. On peut obtenir, par exemple, la fonction de répartition conditionnelle $F_{X_2 | X_1 = x_1}(x_2)$ avec
\begin{align*}
F_{X_2 | X_1 = x_1}(x_2) = C_{2|1} \left( F_{X_2}(x_2) | F_{X_1}(x_1) \right)
\end{align*}


\subsection{Construction d'une copule archimédienne}
Une copule archimédienne est construite à partir de 2 v.a  $Y_i|\Theta \sim \text{Exp}(\Theta)$, $i=1,2$ (et $\Theta$ qui suit une certaine distribution)  peut être construite  avec
\begin{equation}
C(u_1, u_2) = \overline{F}_{\matr{Y}} \left ( \overline{F}_{Y_1}^{-1}(u_1), \overline{F}_{Y_2}^{-1}(u_1) \right )
\end{equation}
où l'on déduit que 
\begin{align*}
F_{Y_i}(x_i) = \esp{F_{Y|\Theta}(x_i | \theta)} = \esp{e^{-\Theta x}} = \laplace_\Theta(x)
\end{align*}

Plusieurs copules possibles : 
\begin{description}
\item[Clayton] $\Theta \sim \Gamma(\frac{1}{\alpha}, 1)$
\item[AMH] $\Theta \sim \text{BinNég}$
\item[Frank] $\Theta \sim \text{Logarithmique}(\gamma = 1 - e^{-\alpha})$
\end{description}


\subsection{Méthode des rectangles}
On peut approximer $F_S(s)$ avec la méthode des rectangles\footnote{Plutôt que d'apprendre les formules ci-dessous, il est mieux de se dessiner les rectangles par rapport à la diagonale puis déduire les $F_{X_1, X_2}$ à additionner et soustraire.}
%TODO : ajouter des Tikz Picture pour montrer comment faire le calcul avec le dessin.

\subsubsection{Méthode \textit{lower}}
On additionne les masses de probabilité associées aux $2^m-1$ rectangles se trouvant \underline{en-dessous} de la diagonale $x_1 + x_2 = s$.

\begin{equation}
\label{eq:rectangle-lower}
A_{S}^{(l, m)}(s) = \sum_{i=1}^{2^m-1} \left [ F_{X_1, X_2} \left( \frac{i}{2^m} s, \frac{2^m - i}{2^m} s \right) -  F_{X_1, X_2} \left( \frac{i-1}{2^m} s, \frac{2^m - i}{2^m} s \right) \right ]
\end{equation}

\subsubsection{Méthode \textit{upper}}
On additionne les masses de probabilité associées aux $2^m$ rectangles se trouvant \underline{au-dessus} de la diagonale $x_1 + x_2 = s$

\begin{equation}
\label{eq:rectangle-upper}
A_{S}^{(u, m)}(s) = \sum_{i=1}^{2^m} \left [ F_{X_1, X_2} \left( \frac{i}{2^m} s, \frac{2^m +1- i}{2^m} s \right) -  F_{X_1, X_2} \left( \frac{i-1}{2^m} s, \frac{2^m +1 - i}{2^m} s \right) \right ]
\end{equation}

On a que
\begin{align*}
A_S^{(l, m)}(s)  \leq A_S^{(l, m+k)}(s) \leq F_S(s)  \leq A_S^{(u, m+k)}(s) \leq A_S^{(u, m)}(s).
\end{align*}










% Annexe dans la cheatsheet
\newpage
\section{Annexe}
\subsection{Les 3 formes explicites de la $TVaR$	}
\label{sec:preuve}

Pour la $TVaR$, il y a 3 preuves à bien connaître : 
\begin{equation*}
TVaR_\kappa(X) = \frac{1}{1 - \kappa} \pi_X(VaR_\kappa(X)) + VaR_\kappa(X)
\end{equation*}


\begin{proof}
\label{preuve:tvar_stoploss}
\begin{align*}
TvaR_\kappa(X)  & = \frac{1}{1 - \kappa} \int_\kappa^1 VaR_u(X) du \\
    & = \frac{1}{1 - \kappa} \int_\kappa^1 (VaR_u(X) - VaR_\kappa(X) + VaR_\kappa(X)) du \\
    & = \frac{1}{1 - \kappa} \int_\kappa^1 (\underbrace{VaR_u(X)}_{\substack{\text{fonction} \\ \text{quantile}}} - VaR_\kappa(X)) du + \underbrace{\int_\kappa^1 VaR_\kappa(X) du}_{\text{intégration d'une constante}} \\
    & = \frac{1}{1 - \kappa} \int_\kappa^1 (F_X^{-1}(u) - VaR_\kappa(X)) \underbrace{f_U(u)}_{U \backsim Unif(0,1)} du  \\
    & + \frac{1}{\cancel{1 - \kappa}} VaR_\kappa(X) (\cancel{1 - \kappa}) \\
    & = \frac{1}{1 - \kappa} E[\max(\underbrace{F_X^{-1}(U)}_{F_X^{-1} \backsim X} - VaR_\kappa(X);0)] + VaR_\kappa(X) \\
    & = \frac{1}{1 - \kappa} E[\max(X - VaR_\kappa(X) ; 0)] + VaR_\kappa(X) \\
    & = \frac{1}{1 - \kappa} \pi_X(VaR_\kappa(X)) + VaR_\kappa(X) \\
\end{align*}
\end{proof}

à partir de la preuve ci-dessus, on peut démontrer celle-ci : 
$$
TVaR_\kappa(X) = \frac{E[X \times 1_{\{X > VaR_\kappa(X) \}}] + VaR_\kappa(X)(F_X(VaR_\kappa(X)) - \kappa)}{1-\kappa} 
$$

\begin{proof}
\begin{align*}
TVaR_\kappa(X)  & = \frac{1}{1 - \kappa} \pi_X(VaR_\kappa(X)) + VaR_\kappa(X) \\
    & = \frac{1}{1 - \kappa} E[\max(X - VaR_\kappa(X); 0)] + VaR_\kappa(X) \\
    & = \frac{1}{1 - \kappa} E[(X - VaR_\kappa(X)) \times 1_{\{X > VaR_\kappa(X) \}}] + VaR_\kappa(X) \\
    & = \frac{1}{1 - \kappa} E[X \times 1_{\{X > VaR_\kappa(X) \}}] - \frac{1}{1 - \kappa} E[VaR_\kappa(X) \times \underbrace{1_{\{X > VaR_\kappa(X) \}}}_{= S_X(VaR_\kappa(X))}] \\
    & + VaR_\kappa(X) \\
    & = \frac{1}{1 - \kappa} E[X \times 1_{\{X > VaR_\kappa(X) \}}] - \frac{1}{1 - \kappa} VaR_\kappa(X)(1 - F_X(VaR_\kappa(X))) \\
    & + \frac{1 - \kappa}{1 - \kappa}VaR_\kappa(X) \\
    & = \frac{E[X \times 1_{\{X > VaR_\kappa(X) \}}] + VaR_\kappa(X)(-1 + F_X(VaR_\kappa(X)) + 1 - \kappa)}{1 - \kappa}  \\
    & = \frac{E[X \times 1_{\{X > VaR_\kappa(X) \}}] + VaR_\kappa(X)(F_X(VaR_\kappa(X)) - \kappa)}{1 - \kappa}  \\
\end{align*}
\end{proof}

Une dernière preuve fortement utilisée pour la $TVaR$, qui découle directement de la dernière : 
\begin{align*}
TVaR_\kappa(X) = \frac{E[X \times 1_{\{X > VaR_\kappa(X) \}}]}{1 - \kappa}  \\
\end{align*}


\begin{proof}
Étant donné que cette formule ne fonctionne seulement que pour une v.a. continue, elle est très facile à prouver : 
\begin{align*}
\text{si $X$ est continue}, \quad \forall x, F_X(VaR_\kappa(X))) = \kappa \\
\end{align*}
Alors, on peut enlever la partie de droite de l'équation.
\end{proof}

\end{multicols*}




%% -----------------------------
%% Fin du document
%% -----------------------------
\end{document}
